\section{Solving power equations (and roots)}

%SU consider adding more homework problems that include formulas from geometry?

% SU do you want to have some solving inversely proportional to a power here?

There's an old saying -- ``when life gives you lemons, make lemonade.''  But how many lemons do you need? It turns out a reasonable equation describing the yield of lemonade from a single lemon is given by $$J = 0.0185C^3$$ where $J$ is the juice, measured in tablespoons, and $C$ is the circumference of the lemon, measured in inches.  (In case you've forgotten, the circumference is the distance \emph{around} the lemon. Think of taking a piece of string and wrapping it around the middle part of the lemon. Then lay the string on a rule to see how long it is.)

A small lemon might measure 6 inches in circumference.  According to our equation, it would yield $$J = 0.0185 (6)^3 = 0.0185 \times \underline{6} \wedge 3 = 3.996 \approx 4\text{ tablespoons}$$ A regular-sized lemon, say 8 inches in circumference, would yield $$J = 0.0185 (8)^3 = 0.0185 \times \underline{8} \wedge 3 = 9.472 \approx 9.5\text{ tablespoons}$$ %And a giant lemon, say 10 inches in circumference, would yield $$J = 0.0185 (10)^3 = 0.0185 \times \underline{10} \wedge 3 = 18.5\text{ tablespoons}$$

%By the way,16 tablespoons equals one cup of liquid so our giant lemon yields over a cup of juice.  Well, at least according to this equation.  

Let's make a table of values and look at a graph of this function.  We've added some values (including some unrealistic ones) to see the shape better.
\begin{center}
\begin{tabular} {|c |c |c |c |c |c |c |c |c|}\hline
$C$ & 0 & 2 & 4 & 6 & 7 & 8 & 9 & 10 \\ \hline
$J$ & 0 & 0.148 & 1.184 & 3.996 & 6.3455 & 9.472 & 13.4865 & 18.5 \\ \hline
\end{tabular}
\end{center}

SU GRAPH

\vspace{2in}

How large a lemon would yield half a cup of juice?  Remember 1 cup is 16 tablespoons so $$\frac{1}{2} \text{ cup} \ast \frac{16 \text{ tablespoons}}{1 \text{ cup}} = 16 \div 2 = 8 \text{ tablespoons}$$ From our graph it look like 7.5 inches in circumference should be pretty close.  Let's use successive approximation to get the answer to two decimal places. We'll round our value of $J$ to two decimal places to fit easily into the table.

\begin{center}
\begin{tabular} {|c |c |c |c |c |c |c |c |c|}\hline
$C$ & 7 & 8 &7.5 & 7.6 & 7.55 & 7.56 & 7.57 \\ \hline
$J$ & 6.35 & 9.47 & 7.80 & 8.12 & 7.96 & 7.99 & 8.03 \\ \hline
vs.\ 8 & low & high & low & high & low &not quite & just over \\ \hline
\end{tabular}
\end{center}

\noindent So a lemon with circumference of approximately 7.57 inches should yield about half a cup of juice.

That didn't take too much work.  But this chapter is all about solving equations.   Much as we have learned to love successive approximation (and for good reason as it works on any type of equation), you might be happy to know that there is a way to solve power equations exactly.  Here's how.  %SU this moves back to solving expl equations if you switch order.

Start with what we're looking for $$J = 8$$ Next, use our equation $J=0.0185C^3$ to get $$0.0185C^3=8$$ We want to find the value of $C$, so we can divide both sides by 0.0185 to get $$\frac{\cancel{0.0185}C^3}{\cancel{0.0185}}=\frac{8}{0.0185}=432.432\ldots$$  Thus we have $$C^3 = 432.432\ldots$$ 

How can we find $C$ just knowing $C^3$?  We take cube roots of both sides to get $$\sqrt[3]{C^3} = \sqrt[3]{432.432\ldots}= 432.432\ldots^{1/3}=432.432\ldots \wedge ( 1 \div 3)= 7.56204\ldots $$   As before we see a lemon of about 7.57 inches should yield half a cup of juice.

Perhaps a brief digression on roots is in order.  (And, that mysterious $\wedge (1\div3)$ business needs some explanation, I expect.) For example, we know $$2^3=2 \wedge 3=8$$ which tells us $$\sqrt[3]{8}=2$$ The \emph{\textbf{cube root}} of a number is whatever you would cube to get that number.  Similarly $$\sqrt[3]{1000} = 10$$ because $$10^3=10 \wedge 3 = 1000$$ 

Some calculators have a key, or sequence of keys, that will take the cube root.  For example, if your calculator has a key labeled $\sqrt[x]{y}$, then you can use it to find the cuberoot by doing $$\sqrt[3]{8}=3\text{ }\sqrt[x]{y}\text{ }8=2$$  Be careful if the key is just labelled $\sqrt{x}$ because that only takes square roots and can't be used for cube roots.

Otherwise you need to know that cube root is equivalent to raising to the 1/3rd power.  So you can calculate $$\sqrt[3]{8}=8^{1/3}=8\wedge(1\div3)=2$$ as we did in our example.  Notice how the 1/3 is in parentheses.  That's because we want the division before the exponent but the usual order of operations is the other way.  If your calculator has a fraction entering key, you might be able to do $$\sqrt[3]{8}=8^{1/3}=8\wedge1/3=2$$ instead, but don't forget to use the fraction key $/$ not the divides key $\div$.

In general, the \emph{\textbf{$n$th root}} of a number is whatever number you would raise to the $n$ power to get the number.  The formula is
\begin{center}
\textsc{The Root Formula} \\ ~\\
The equation $C^n=v$ has solution $C= \sqrt[n]{v} = v^{1/n}$
\end{center}

SU -- decide which comes first: solving expl or solving power, then have 2nd one refer to the 1st.

Back to our lemonade example.  Let's practice solving this type of equation.  If we wanted a lemon that yields 10 tablespoons of juice we would solve $$J=10$$ by putting in our equation we get $$0.0185C^3=10$$  Next, divide each side by 0.0185 to get $$\frac{\cancel{0.0185}C^3}{\cancel{0.0185}}=\frac{10}{0.0185}$$ so that $$C^3 = 10 \div 0.0185 = 540.540\ldots$$ Then take cuberoots to get $$C=\sqrt[3]{ 540.540\ldots} = 540.540\ldots\wedge(1 \div 3) = 8.14596\ldots \approx 8.1 \text{ inches}$$

As before we see that we solve in the reverse order of operations.
\begin{quote} If evaluating, first do $\wedge$ (power), then do $\times, \div$

If solving, first do $\times,\div$, then do $\sqrt[n]{\text{ }}$ (root)\end{quote}

A few pages of calculations into our example and we have only half a cup of lemonade to show for it. How about if we're trying to make a 1/2 gallon pitcher of lemonade?  Suppose the store sells lemons by the bag, where all the lemons in the bag are just about the same size (all small, all medium, etc.).  How many lemons of a fixed size will it take to make 1/2 gallon of lemonade?  We can write a new equation to describe this situation.

First, convert 1/2 gallon into tablespoons. $$\frac{1}{2} \text{ gallon } \ast \frac{4 \text{ quarts}}{1 \text{ gallon}} \ast \frac{4 \text{ cups}}{1 \text{ quart}} \ast \frac{16 \text{ tablespoons}}{1 \text{ cup}} = 4 \times 4 \times 16 \div 2 = 128 \text{ tablespoons}$$

Let $L$ be the number of lemons (of a fixed size) and suppose each lemon yields $J$ tablespoons of juice. For example, if $J$ = 10 tablespoons, then $$\frac{128 \text{ tablespoons}}{10 \frac{\text{tablespoons}}{\text{lemon}}} = 128 \div 10 = 12.8 \text{ lemons}$$ so we would need 13 lemons.  In general the number of lemons we need is given by the equation $$L=\frac{128}{J}$$ 

Look at a table of a few values and graph for this function.  Notice that it's decreasing because the larger the lemons, the fewer we need to use.  That makes sense, doesn't it?

\begin{center}
\begin{tabular} {|c |c |c |c |c |c|}\hline
$J$ & 4 & 6 & 10 & 12 &16 \\ \hline
$L$ & 32.0 & 21.3 & 12.8 & 10.7 & 8 \\ \hline
\end{tabular}
\end{center}

SU NEED GRAPH \vspace{2in}

Last time I made a pitcher of lemonade it took 9 lemons. How much juice did each yield (assuming they're all about the same size)?  Let's solve our equation to find the answer.  Start with what we want $$L=9$$ and use our equation to get $$\frac{128}{L}=9$$  We haven't seen an equation like this before, where the independent variable in in the denominator (bottom) of the fraction, but not to worry. Remembering that $\frac{128}{L}$ means $128 \div L$, we can multiply both sides of the equation by $L$ to get $$\cancel{L}\ast\frac{128}{\cancel{L}} = 9\ast L$$ so that $$9L=128$$  (We switched the variable onto the left-hand side for convenience.)  Now the equation looks much more familiar.  Dividing both sides by 9 gives us $$\frac{\cancel{9}L}{\cancel{9}}= \frac{128}{9}$$ so $$L=\frac{128}{9} = 128\div 9 = 14.22\ldots \approx 14.2 $$  Each lemon must have yielded around 14.2 tablespoons of juice.

And what goes better with lemonade than lemon cake.  For that we're going to need some grated lemon peel.  As with juice, the amount of lemon peel depends on the size of the lemon.  One equation is $$P=0.061C^2$$ where $C$ is the circumference of the lemon, measured in inches, as before and $P$ is the amount of lemon peel, measured in tablespoons.  For example, a lemon of circumference 7 inches will produce about 3 tablespoons of grated lemon peel because $$0.061\ast7^2 = 0.061 \times \underline{7} \wedge 2 = 2.989 \approx 3$$

As before, we can look at a table of select values (not all of which are realistic).

\begin{center}
\begin{tabular} {|c |c |c |c |c |c |c |c |c|}\hline
$C$ & 0 & 2 & 4 & 6 & 7 & 8 & 9 & 10 \\ \hline
$P$ & 0 & 0.244 & 0.976 & 2.196 & 2.989 & 3.904 & 4.941 & 6.1 \\ \hline
\end{tabular}
\end{center}

SU NEED GRAPH \vspace{2in}

What size lemon would give 4 tablespoons of grated lemon peel?  From the graph it looks like just over 8 inches in circumference should do.  A quick successive approximation shows it's around 8.1 inches in circumference.

\begin{center}
\begin{tabular} {|c |c |c |c |c |c |c |c |c|}\hline
$C$ & 8 & 9 &  8.1\\ \hline
$P$ & 3.904 & 4.941& 4.00221 \\ \hline
vs.\ 4 & bit low & high & close \\ \hline
\end{tabular}
\end{center}

To solve exactly we begin with what we want $$P=4$$ and use our equation to get $$0.061C^2=4$$ Then divide both sides by 0.061 to get $$\frac{\cancel{0.061}C^2}{\cancel{0.061}}= \frac{4}{0.061}$$ which simplifies to $$C^2 = \frac{4}{0.061} = 4 \div 0.061 = 65.57\ldots $$ Now to undo the square we use regular square roots.  There is most likely a special key on your calculator for square roots, but we'll do it the same as any other root here. $$\sqrt{C^2} = \sqrt{65.57\ldots} $$ so $$C = \sqrt{65.57\ldots} = 65.57\ldots^{1/2} = 65.57\ldots \wedge ( 1 \div 2) = 8.097\dots \approx 8.1$$ as before. 

%\newpage

%
%\section{Solving power equations (and roots)}

 \begin{center}
\line(1,0){300} %\line(1,0){250}
\end{center}

\section*{Homework}

\noindent \textbf{Start by doing Practice exercises \#1-4 in the workbook.}

\bigskip

\noindent \textbf{Do you know \ldots}

\begin{itemize} 
\item What a ``power'' equation is? 
\item What we mean by square root, cube root, and $n$th root? 
\item How to calculate square roots, cube roots, and $n$th roots on your calculator? 
\item How to evaluate the \textsc{Root Formula} on your calculator?
\item When to use the \textsc{Root Formula}?  \emph{Ask your instructor if you need to remember the \textsc{Root Formula} or it will be provided during the exam.} 
\item How to solve a power equation? 
\item What the graph of a power function looks like? 
\item[~] \textbf{If you're not sure, work the rest of exercises and then return to these questions.  Or, ask your instructor or a classmate for help.} 
\end{itemize}

\subsection*{Exercises}

\begin{enumerate} 
\setcounter{enumi}{4}

\item Recall our lemon zest formula $Z=0.018C^2$ where $C$ is the circumference of the lemon, in inches, and $Z$ is the amount of lemon zest, in tablespoons.
\begin{enumerate}
\item Use the information we found earlier to draw a graph of the function.  Include values $0 \le C \le 10$. 
\item Set up and solve an equation to find the size lemon needed for 1 tablespoon of zest.
\item Suppose the formula holds for grapefruit too.  I don't know of any recipe that calls for grapefruit zest; it is very bitter!  But grapefruit is notorious for interacting with certain medications, and so we're collecting some zest for an experiment.  Let's say we need \nicefrac{1}{4} cup of zest.  How large a grapefruit will we need?  Set up and solve an equation to answer.  Use that $1 \text{ cup} = 16 \text{ tablespoons}$.
\end{enumerate}

\item Wind turbines are used to generate electricity.  For a particular wind turbine, the equation $$W = 2.4 S^3$$ can be used to calculate the amount of electricity generated ($W$ watts) for a given wind speed ($S$ mph), over a fixed period of time.

\hfill \emph{Story also appears in 1.1, 1.3, and 2.4 Exercises}
\begin{enumerate}
\item Set up and solve an equation to determine the wind speed that will generate 12,500 watts of electricity. 
\item Repeat for 45,000 watts.
\end{enumerate}

\item Mom always said to sit close to the lamp when I was reading.  The intensity of light $L$, measured in percentage (\%) that you see from a lamp depends on your distance from the lamp, $F$ feet as described by the formula $$L=\frac{100}{F^2}$$  
\hfill \emph{Story also appears in 1.1 and 2.3 Exercises}
\begin{enumerate}
\item I am most comfortable reading in good light, say 70\% intensity.  According to the equation, how far away can I sit from the lamp?  Use successive approximation to guess the answer to the nearest \nicefrac{1}{10} foot.  Then set up and solve an equation.   Answer to the nearest inch.  
\item For reading a magazine 35\% intensity is enough light. According to the equation, how far away can I sit from the lamp?  Use successive approximation to guess the answer to the nearest \nicefrac{1}{10} foot.  Then set up and solve an equation.   Answer to the nearest inch.  
\end{enumerate}

\item The lake by Rodney's condo was stocked with bass (fish) 10 years ago.  There were initially 400 bass introduced.  Rodney wonders what the annual percent increase of the bass has been and realizes he can calculate it from the number of fish now.  He will use the equation $$B=400 g^{10}$$
where  $B$ is the number of bass in the lake now and $g$ is the annual growth factor.  For each number of bass, first solve for $g$ using the \textsc{Root Formula}, then calculate $r=g-1$.  The percent increase is $r$ written as a percent.

\hfill \emph{Story also appears in 5.5 Exercises}
\begin{enumerate}
\item Find the annual percent increase if there are $B=3,000$ bass now.
\item Find the annual percent increase if there are $B=4,000$ bass now.
\end{enumerate}

\item If you drop a rock from a high place, it falls $R$ feet in $T$ seconds where 
$$R = 16T^2$$
\begin{enumerate}
\item How far does the rock fall in 2 seconds? In 4 seconds?
\item Is the rock falling faster during the first two seconds ($T=0$ to $T=2$) or during the second two seconds ($T=2$ to $T=4$)?   Calculate the rate of change to decide.
\item Tia dropped a rock from her apartment window that's 
 300 feet above ground. Will the rock have hit the ground by 4 seconds after it was dropped?
\item If you evaluate at $T=5$, what value of $R$ do you get and what does it mean in the story, again assuming the rock is dropped from 300 feet up.
\item When does the rock hit the ground?  Set up and solve an equation.  \emph{Hint: what value of $R$ do you solve for?}
\item Now suppose we have a new variable, $H$, which represents the height of the rock Tia dropped after $T$ seconds, write a new equation for $H$ as a function of $T$.  
\item Show how to set up and solve an equation using this new equation to find when the rock hits the ground.  \emph{Hint:  what value of $H$ do you solve for now?}
\end{enumerate}

\item Wynter has a pretty decent job. He is paid a salary of \$780 per week but his hours vary week-to-week. Even though Wynter is not paid by the hour, he can figure out what his hourly wage would be depending on the number of hours he works using the equation $$E = \frac{780}{H}$$ where if he works $H$ hours, then he's earning the equivalent of \$$E$/hour.

\hfill \emph{Story also appears in 2.3 Exercises}
\begin{enumerate}
\item Make a table showing Wynter's equivalent hourly wage if he works 40, 50, or 60 hours a week.
\item Wynter was complaining that things have been so busy lately at work that he's earning the equivalent of only \$9.25/hr.  How many hours a week does that correspond to?
\item Wynter was hoping to earn the equivalent of \$14/hour.  How many hours a week does that correspond to?  
\item Draw a graph illustrating how Wynter's equivalent hourly wage decreases as a function of the number of hours he works.  Include a few extreme values such as 10 hours and 100 hours to better see the shape of the graph.
\end{enumerate}


\end{enumerate}
