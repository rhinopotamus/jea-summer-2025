%\section{Slopes}

\begin{center}
\line(1,0){300} %\line(1,0){250}
\end{center}

\section*{Homework}

\noindent \textbf{Start by doing Practice exercises \#1-4 in the workbook.}

\bigskip

\noindent \textbf{Do you know \ldots}

\begin{itemize} 
\item Which types of situations are linear? 
\item What the slope of a linear function means in the story and what it tells us about the graph? 
\item How to calculate the slope between two points? 
\item What is means if the slope is negative? 
\item How to find the equation of a line through two points? 
\item How to find a linear function given two examples in a story? 
\item If both the slope and intercept are unknown, which is easier to calculate first? 
\item[~] \textbf{If you're not sure, work the rest of exercises and then return to these questions.  Or, ask your instructor or a classmate for help.}
\end{itemize}

\subsection*{Exercises}

\begin{enumerate} 
\setcounter{enumi}{4}

\item  I just saw an advertisement for the same paper we use at the office for only  \$4.25 per ream at a supply store.  (Ream?  Yes.  That's 500 sheets of paper, usually wrapped in paper.) Is that a good deal?
\begin{enumerate}
\item There are 10 reams in a case.  What is the advertised price come to per case?
\item I'm not sure I want to go get a case of paper myself because a case of paper is pretty heavy to lift.  Paper is sold by the weight.  Thick, heavier paper is considered fancier than lighter paper.  The office uses a multipurpose paper called ``92'' meaning it weighs 92 grams per square meter which comes out to around 5 grams per sheet.  How much does a case weigh? Use $1 \text{ kilogram} \approx 2.2 \text{ pounds}$. % Go grams/sheet, sheet/ream, ream/case.
\item But, at the office we pay a delivery charge.  Compare the cost of having just one case delivered versus my buying one case at the store.  Recall that the office pays \$15 delivery fee and \$39.99 per case.
\item Write new equation for paper cost assuming I pick it up at the store. Use $N$ for the number of cases of paper and $C$ is the total cost, in dollars.
 \emph{Hint: this equation is a direct proportionality.}
\item Compare total cost if get 4 cases either delivered or from store.  Repeat 13 cases.  Recall that the equation for delivered paper is $C=15+39.99N$.
\item Graph both functions together on the same axes.
\item Set up and solve inequality for when delivered is cheaper.  
\end{enumerate}

\item The amount of garbage generated in the United States has increased steadily, from 88.1 million tons in 1960 to 254.2 million tons in 2006.  

\hfill \begin{footnotesize}  Source:  Environmental Protection Agency \end{footnotesize}
%  Link to data: http://www.epa.gov/osw/nonhaz/municipal/msw99.htm  Go to http://www.epa.gov/osw/nonhaz/municipal/pubs/msw_2010_rev_factsheet.pdf  for results, and data tables follow.  Added a bit about guess for 2010.

\hfill \emph{Story also appears in 4.5 Exercises}
\begin{enumerate}
\item Assume the amount of garbage increases linearly, by how much has garbage increased each year? 
\item Name the variables, including units, and write a linear equation relating them.
\item According to your equation, how much garbage was projected for 2010?  For 2020?
\item If this trend continues, when will the amount of garbage generated exceed 300 million tons?   Show how to set up and solve an inequality to find the answer.  Be sure to state the actual year.
\item A 2010 report listed the amount of garbage at 249 million tons.  Compare this information to your previous answer.  What are some possible explanations for why this amount was less than expected (and actually decreased from 2006)?
\end{enumerate} 

\item Now that he is retired, Elmer gets a pension check from the Railroad Company each month.  There's a set amount he gets each month but the company deducts a fixed percentage of whatever outside income he earns.  Elmer works part-time at the local hardware store.  In February he earned \$444.10 at the hardware store and his pension check that month was \$886.23.  In March he worked much less,  earning only \$179.30 at the hardware store; his pension check that month was \$912.71
\begin{enumerate}
\item What percentage of his income from the hardware store is deducted from his pension check?  \emph{Calculate the fraction of a dollar deducted for each dollar earned.  Convert your answer to percent.}
\item If Elmer doesn't work in April, how much will his pension check be?  
\item Write an equation showing how Elmer's pension check is affected by his income from the hardware store.  Use $H$ for his income from the hardware store and $P$ for his pension check, both in dollars.  
\item Elmer would like to earn enough at the hardware store to make at least \$1,200 total per month.  Using $T$ for the total Elmer earns in a month (in dollars), write an equation for $T$ as a function of $H$.  \emph{Hint:  start with $T=H+P$, then use your equation for $P$ from part (c) to write everything with $H$ instead.}
\item Now set up and solve an inequality to determine how much Elmer needs to earn at the hardware store to make at least \$1,200 total per month.
\item If Elmer earns \$8.15 per hour, how many hours does he need to work at the hardware store to make at least  \$1,200 total per month, accounting for his income from the hardware store and his pension check?
\end{enumerate}
 
\item Your local truck rental agency lists what it costs to rent a truck (for one day) based on the number of miles you drive the truck.  They use a linear pricing model.
\begin{center}
\begin{tabular} {|c| |c |c|c|c|} \hline
distance driven (miles) & 50 & 100 & 150 & 200 \\ \hline
rental cost (\$) & 37.50 & 55.00 & 72.50 & 90.00 \\ \hline
\end{tabular}
\end{center}
If you rent a truck and drive it 10 miles, how much do you think it will cost?  As part of your work, name the variables and write a linear equation relating them.

\hfill \emph{Story also appears in 1.2 and 1.3 Exercises}

\item In 2008, the median household income was about \$50,303.  By 2010 it was down to about \$49,445.
\hfill \begin{footnotesize} Source:  U.S. Census Bureau \end{footnotesize}
 % source:  http://www.census.gov/hhes/www/income/data/statemedian/index.html  Access Excel spreadsheet Median Household Income by State - Single-Year Estimates
\begin{enumerate}
\item By how much has it decreased each year, on average? The phrase ``on average'' means that you should assume the decrease is linear.
\item Name the variables and write a linear equation relating them.
\item At this rate when will the median family fall below \$48,000?  Set up and solve an inequality.
\item Graph and check.
\end{enumerate}

\item Buoy instruments in the oceans report changes in the sea level.  In 2005 the sea level (averaged across all the oceans) was 51.7 millimeters above the historical sea level.  In 2012 the sea level was 73.4 millimeters above the historical sea level.  You can assume the increase is linear.
\hfill \begin{footnotesize} Source:  National Aeronautics and Space Administration \end{footnotesize}
\begin{enumerate}
\item Name the variables, including units.
\item Display the information from the story in a table.
\item What is the rate of increase for the sea level?
\item Write an equation relating the variables.
\item In what year will the sea level be 80 millimeters above the historical level?
\end{enumerate} % Assumes an annual growth of 3.1 mm /yr  Source: http://sealevel.colorado.edu/  Cut:  to NASA.

\end{enumerate}