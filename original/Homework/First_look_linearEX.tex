%\section{A first look at linear equations}

\begin{center}
\line(1,0){300} %\line(1,0){250}
\end{center}

\section*{Homework}

\noindent \textbf{Start by doing Practice exercises \#1-4 in the workbook.}

\bigskip

\noindent \textbf{Do you know \ldots}

\begin{itemize} 
\item How to generalize from an example to find an equation? 
\item Where the dependent variable usually is in an equation? 
\item What the slope of a linear function means in the story and what it tells us about the graph? 
\item What the intercept of a linear function means in the story and what it tells us about the graph?  
\item The template for a linear equation? \emph{Ask your instructor if you need to remember the template or if it will be provided during the exam.} 
\item Where the slope and intercept appear in the template for a linear equation?  
\item What makes a function linear? 
\item How to plot negative numbers on a graph? 
\item What the graph of a linear function looks like? 
 \item[~] \textbf{If you're not sure, work the rest of exercises and then return to these questions.  Or, ask your instructor or a classmate for help.} 
\end{itemize}

\subsection*{Exercises}

\begin{enumerate} 
\setcounter{enumi}{4}

\item Plumbers are really expensive, so I'm comparing their rates.  Write an equation for each possibility, using the same variables as our example:  $T$ for the time the plumber takes (in hours) and $P$ for the plumber's total charge (in dollars). 

 \hfill \emph{Story also appears in 4.1 \#4}
\begin{enumerate}
\item James charges \$50 to show up plus \$120 per hour. 
\item Jo the plumber is just getting started in the business.  She charges \$45 to show up plus \$55 per hour.
\item Mario advertises ``no trip charge'' but his hourly rate is \$90 per hour. 
\item Not to be outdone, Luigi offers to unclog any drain for \$150, no matter how long it takes.  (``Wake up, Luigi! The only time plumbers sleep on the job is when we're working by the hour,'' says Mario.)
\end{enumerate} 

\item Abduwali has just opened a restaurant. He spent \$82,500 to get started but hopes to earn back \$6,300 each month.  \hfill \emph{Story also appears in 3.1 Exercises}
\begin{enumerate}
\item If all goes according to plan, will he have made money 10 months from now?
\item Name the variables and write an equation relating them.
\item Identify the slope and intercept, along with their units, and explain what each means in terms of the story. 
\item Make a small tables of values and use it to draw a graph showing Abduwali's profit.
\end{enumerate}

\item When Gretchen walks on her treadmill, she burns 125 calories per mile.

\hfill \emph{Story also appears in 3.1 and 3.2 Exercises}
\begin{enumerate}
\item How many calories will Gretchen burn if she walks 2.3 miles?
\item Name the variables and write an equation relating them.
\item Identify the slope and intercept, along with their units, and explain what each means in terms of the story. 
\item Make a table showing the calories she burns walking 0, 1, 2, 3, or 4 miles.
\end{enumerate} 

\item The local burger restaurant had a promotion this summer.  Typically a bacon double cheeseburger costs \$7.16.  They reduced the price by 2\textcent~for each degree in the daily high temperature. For example, if the high temperature was 80$^\circ$F, they would decrease the price by $0.02 \times 80 = \$1.60$, so the double cheeseburger would cost $7.16-1.60=\$5.56$.  Mmmm.
\hfill \emph{Story also appears in 3.1 \#4} 
\begin{enumerate}
\item Name the variables in the story and write a linear equation relating them.
\item Is the function increasing or decreasing?
\item Make a table showing the price of a bacon cheeseburger when the daily high temperature is 65$^\circ$F, 75$^\circ$F, and 90$^\circ$F.
\item Draw a graph illustrating how the price of a bacon double cheeseburger depends on the temperature.  Start the temperature on your graph at 60$^\circ$F.
\end{enumerate} 

\item A report on health care back in 1975 stated that the U.S. had around 1,466,000 hospital beds and since then the number of beds has declined by around 16,000 beds  per year.   
\hfill \begin{footnotesize} Source:  Center for Disease Control and Prevention \end{footnotesize}
\begin{enumerate}
\item Name the variables, including units and dependence.
\item Write an equation illustrating the function.
\item Is the function increasing or decreasing?
\item Make a table showing the number of hospital beds projected for 1980, 1990, 2000, 2010, and 2020.  
\item  At this rate of decline, in what year will we have only \nicefrac{1}{2} million beds?  First estimate the answer from your table.  Then figure it out, to the nearest year.
\end{enumerate}
%FROM:  http://www.cdc.gov/nchs/data/hus/2010/113.pdf
% 1975 = 1,465,828 then 1980 = 1,364,516 then 1990=1,213,327
% then 1995=1,080,601 then 2000=983,628 then 2007=845,199 then 2008=951,045
% FROM:

\item The stretch of interstate highway through downtown averages 1,450 cars per hour during the morning rush hour.  The economy is improving (finally), but with that the county manager predicts traffic levels with increase around 130 cars per hour more each week for the next couple of years. \hfill \emph{Story also appears in 3.1 Exercises}
\begin{enumerate}
\item Name the variables and write an equation relating them.
\item Make a table showing the number of cars per hour anticipated now and in 2 years, 4 years, 6 years, 8 years, and 10 years.
\item Significant slowdown are expected if traffic levels exceed 2,000 cars per hour.  When do they expect that will happen? Estimate your answer from your table.  (Or, figure it out.)
\item If traffic levels exceed 2,500 the county plans to install control lights at the on ramps.  When is that expected to happen?   
\end{enumerate} 

\end{enumerate}

