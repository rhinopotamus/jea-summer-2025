%\section{Tables and graphs}

\begin{center}
\line(1,0){300} %\line(1,0){250}
\end{center}

\section*{Homework}

\noindent \textbf{Start by doing Practice exercises \#1-4 in the workbook.}

\bigskip

\noindent \textbf{Do you know \ldots}

\begin{itemize}
\item Where the independent and dependent variables appear in a table and in a graph? 
\item How to guess values from a table or from a graph? 
\item How to make a graph from a table?
\item Why we start each axis at 0? 
\item What we mean by scaling an axis evenly? 
\item How to make a table and then a graph from a story? 
\item Why we draw in a smooth line or curve connecting the points? 
\item What type of graphing technology, if any, you're allowed to use?  \emph{Ask your instructor.}
\item[~] \textbf{If you're not sure, work the rest of exercises and then return to these questions.  Or, ask your instructor or a classmate for help.}
\end{itemize}

\subsection*{Exercises} 

\begin{enumerate} 
\setcounter{enumi}{4}

\item The table lists estimates of Earth's population, in billions, for select years since 1800.   
\begin{center}
\begin{tabular} {|l ||c |c |c |c |c |c |c |} \hline
Year & 1800 & 1850 & 1900 & 1950 & 1970 & 1990 & 2000 \\ \hline
Population & .98 & 1.26 & 1.65 & 2.52 & 3.70 & 5.27 & 6.06  \\ \hline
\end{tabular}
\end{center}
\hfill \begin{footnotesize} Source:  ``The World at Six Billion'' United Nations report\end{footnotesize}
%http://www.un.org/esa/population/publications/sixbillion/sixbilpart1.pdf

 \hfill \emph{Story also appears in 1.3 Exercises}
 
Use the table to find or reasonably guess the answers to the following questions.
\begin{enumerate}
\item What was the population of Earth in 1850?
\item What do you think the population of Earth was in 1860?
\item What do you think the population of Earth was in 1960?
\item In what year do you think the population of Earth first exceeded 2 billion?
\item In what year do you think the population of the world will exceed 7 billion?
\item Identify the variables, including units and dependence.
%\item Draw a detailed graph illustrating the dependence based on the points given in the table.  SAVE GRAPH FOR NEXT SECTION
\end{enumerate}  

\item Your local truck rental agency lists what it costs to rent a truck (for one day) based on the number of miles you drive the truck.
\begin{center}
\begin{tabular} {|l ||c |c|c|c|} \hline
Distance driven (miles) & 50 & 100 & 150 & 200 \\ \hline
Rental cost (\$) & 37.50 & 55.00 & 72.50 & 90.00 \\ \hline
\end{tabular}
\end{center}
 \hfill \emph{Story also appears in 1.3 and 4.4 Exercises}
 
Use the table to find or reasonably guess the answers to the following questions.
\begin{enumerate}
\item How much does it cost to rent a truck if you drive it 100 miles?
\item How many miles did you drive a truck costing \$90.00 to rent?
\item If you rent a truck and drive it 75 miles, how much do you think it will cost?
\item If you rent a truck and drive it 10 miles, how much do you think it will cost?
\item If you rent a truck and it costs \$60.00, about how many miles was it driven?
\item Identify the variables, including units, realistic domain, and dependence.
\item Draw a detailed graph illustrating the dependence based on the points given in the table.  Be sure your axes are labeled and evenly scaled.  Sketch in a smooth curve connecting the points.
\item Use your graph to check your answers to the questions.  Modify,  if necessary.
\end{enumerate}  

\item The temperature was 40$^\circ$F at noon yesterday downtown Minneapolis but it dropped 3$^\circ$F an hour in the afternoon.    \hfill \emph{Story also appears in 1.1 and 4.1 Exercises}
\begin{enumerate}
\item Make a table of reasonable values.
\item Draw a graph illustrating the dependence.  Count time in hours after noon.
\item According to your table and graph, when did the temperature drop below freezing (32$^\circ$F)?
\item According to your graph, when did the temperature drop below 0$^\circ$F.  Does that seem realistic?  \emph{Here in the midwest there are no oceans or mountains to moderate large temperature changes.}
\end{enumerate} 

\item Mrs.\ Nystrom's Social Security benefit was \$746.17/month when she retired from teaching in 2009. She had taught in elementary school since I was a girl.   Benefits have increased by 4\% per year.  \hfill \emph{Story also appears in 1.1 and 5.1 Exercises}
\begin{enumerate}
\item Make a table of reasonable values using $N$ for Mrs.\ Nystrom's benefits (in dollars) and $Y$ for time (in years since 2009).  
\item Draw a graph illustrating the dependence.  Scale your graph to show up through the year 2020 and \$\text{1,200}.
\item According to your table and graph, when did her benefit pass \$900/month?
\item If you extend your graph to 2020, what would you estimate Mrs.\ Nystrom's benefit will be then, assuming these increases continue?
\end{enumerate}  

\item The table adapted from shows the ``heat index'' as a function of humidity at an air temperature of 88$^\circ$F. With up to about 40\% humidity, 88$^\circ$F feels like it's 88$^\circ$F. But if the humidity rises to 60\%, then it feels like it is 95$^\circ$F;  that is, the heat index is 95$^\circ$F.  
\begin{center}
\begin{tabular} {|l| |c|c|c|c|c|c|} \hline
Humidity (\%)  & 50 & 60 & 70 & 85 & 90 & 95 \\ \hline
Heat index ($^\circ$F)  & 91 & 95 & 100 & 110 & 113 & 117 \\ \hline
\end{tabular}
\end{center}
\hfill \begin{footnotesize} Source:  National Oceanic and Atmospheric Administration  \end{footnotesize}
% http://www.crh.noaa.gov

All of the following questions refer to situations when the air temperature is 88$^\circ$F.
\begin{enumerate}
\item What is the heat index when the humidity level is 70\%?
\item At what humidity level does it feel more like 98$^\circ$F?
\item Heat exhaustion is likely to occur when the heat index reaches 105$^\circ$F.  At what humidity level will heat exhaustion likely occur?
\item The heat index is considered danger in the range from 105$^\circ$F to 129$^\circ$F.  What range of humidity levels are considered dangerous?
\item What do you think the heat index would be at 99\% humidity?
\item Identify the variables, including units, realistic domain, and dependence.
\item Draw a detailed graph illustrating the dependence based on the points given in the table.  Be sure your axes are labeled and evenly scaled.  Sketch in a smooth curve connecting the points.
\item Use your graph to check your answers to the questions.  Modify, if necessary.
\end{enumerate}
 
\end{enumerate}



