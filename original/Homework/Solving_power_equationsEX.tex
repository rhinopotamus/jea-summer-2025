
%\section{Solving power equations (and roots)}

 \begin{center}
\line(1,0){300} %\line(1,0){250}
\end{center}

\section*{Homework}

\noindent \textbf{Start by doing Practice exercises \#1-4 in the workbook.}

\bigskip

\noindent \textbf{Do you know \ldots}

\begin{itemize} 
\item What a ``power'' equation is? 
\item What we mean by square root, cube root, and $n$th root? 
\item How to calculate square roots, cube roots, and $n$th roots on your calculator? 
\item How to evaluate the \textsc{Root Formula} on your calculator?
\item When to use the \textsc{Root Formula}?  \emph{Ask your instructor if you need to remember the \textsc{Root Formula} or it will be provided during the exam.} 
\item How to solve a power equation? 
\item What the graph of a power function looks like? 
\item[~] \textbf{If you're not sure, work the rest of exercises and then return to these questions.  Or, ask your instructor or a classmate for help.} 
\end{itemize}

\subsection*{Exercises}

\begin{enumerate} 
\setcounter{enumi}{4}

\item Recall our lemon zest formula $Z=0.018C^2$ where $C$ is the circumference of the lemon, in inches, and $Z$ is the amount of lemon zest, in tablespoons.
\begin{enumerate}
\item Use the information we found earlier to draw a graph of the function.  Include values $0 \le C \le 10$. 
\item Set up and solve an equation to find the size lemon needed for 1 tablespoon of zest.
\item Suppose the formula holds for grapefruit too.  I don't know of any recipe that calls for grapefruit zest; it is very bitter!  But grapefruit is notorious for interacting with certain medications, and so we're collecting some zest for an experiment.  Let's say we need \nicefrac{1}{4} cup of zest.  How large a grapefruit will we need?  Set up and solve an equation to answer.  Use that $1 \text{ cup} = 16 \text{ tablespoons}$.
\end{enumerate}

\item Wind turbines are used to generate electricity.  For a particular wind turbine, the equation $$W = 2.4 S^3$$ can be used to calculate the amount of electricity generated ($W$ watts) for a given wind speed ($S$ mph), over a fixed period of time.

\hfill \emph{Story also appears in 1.1, 1.3, and 2.4 Exercises}
\begin{enumerate}
\item Set up and solve an equation to determine the wind speed that will generate 12,500 watts of electricity. 
\item Repeat for 45,000 watts.
\end{enumerate}

\item Mom always said to sit close to the lamp when I was reading.  The intensity of light $L$, measured in percentage (\%) that you see from a lamp depends on your distance from the lamp, $F$ feet as described by the formula $$L=\frac{100}{F^2}$$  
\hfill \emph{Story also appears in 1.1 and 2.3 Exercises}
\begin{enumerate}
\item I am most comfortable reading in good light, say 70\% intensity.  According to the equation, how far away can I sit from the lamp?  Use successive approximation to guess the answer to the nearest \nicefrac{1}{10} foot.  Then set up and solve an equation.   Answer to the nearest inch.  
\item For reading a magazine 35\% intensity is enough light. According to the equation, how far away can I sit from the lamp?  Use successive approximation to guess the answer to the nearest \nicefrac{1}{10} foot.  Then set up and solve an equation.   Answer to the nearest inch.  
\end{enumerate}

\item The lake by Rodney's condo was stocked with bass (fish) 10 years ago.  There were initially 400 bass introduced.  Rodney wonders what the annual percent increase of the bass has been and realizes he can calculate it from the number of fish now.  He will use the equation $$B=400 g^{10}$$
where  $B$ is the number of bass in the lake now and $g$ is the annual growth factor.  For each number of bass, first solve for $g$ using the \textsc{Root Formula}, then calculate $r=g-1$.  The percent increase is $r$ written as a percent.

\hfill \emph{Story also appears in 5.5 Exercises}
\begin{enumerate}
\item Find the annual percent increase if there are $B=3,000$ bass now.
\item Find the annual percent increase if there are $B=4,000$ bass now.
\end{enumerate}

\item If you drop a rock from a high place, it falls $R$ feet in $T$ seconds where 
$$R = 16T^2$$
\begin{enumerate}
\item How far does the rock fall in 2 seconds? In 4 seconds?
\item Is the rock falling faster during the first two seconds ($T=0$ to $T=2$) or during the second two seconds ($T=2$ to $T=4$)?   Calculate the rate of change to decide.
\item Tia dropped a rock from her apartment window that's 
 300 feet above ground. Will the rock have hit the ground by 4 seconds after it was dropped?
\item If you evaluate at $T=5$, what value of $R$ do you get and what does it mean in the story, again assuming the rock is dropped from 300 feet up.
\item When does the rock hit the ground?  Set up and solve an equation.  \emph{Hint: what value of $R$ do you solve for?}
\item Now suppose we have a new variable, $H$, which represents the height of the rock Tia dropped after $T$ seconds, write a new equation for $H$ as a function of $T$.  
\item Show how to set up and solve an equation using this new equation to find when the rock hits the ground.  \emph{Hint:  what value of $H$ do you solve for now?}
\end{enumerate}

\item Wynter has a pretty decent job. He is paid a salary of \$780 per week but his hours vary week-to-week. Even though Wynter is not paid by the hour, he can figure out what his hourly wage would be depending on the number of hours he works using the equation $$E = \frac{780}{H}$$ where if he works $H$ hours, then he's earning the equivalent of \$$E$/hour.

\hfill \emph{Story also appears in 2.3 Exercises}
\begin{enumerate}
\item Make a table showing Wynter's equivalent hourly wage if he works 40, 50, or 60 hours a week.
\item Wynter was complaining that things have been so busy lately at work that he's earning the equivalent of only \$9.25/hr.  How many hours a week does that correspond to?
\item Wynter was hoping to earn the equivalent of \$14/hour.  How many hours a week does that correspond to?  
\item Draw a graph illustrating how Wynter's equivalent hourly wage decreases as a function of the number of hours he works.  Include a few extreme values such as 10 hours and 100 hours to better see the shape of the graph.
\end{enumerate}


\end{enumerate}
