%\section{Solving linear equations}

 \begin{center}
\line(1,0){300} %\line(1,0){250}
\end{center}

\section*{Homework}

\noindent \textbf{Start by doing Practice exercises \#1-4 in the workbook.}

\bigskip

\noindent \textbf{Do you know \ldots}

\begin{itemize} 
\item When you solve an equation, as opposed to just evaluating?  
\item Why we ``do the same thing to each side'' of an equation when solving? 
\item How to solve a linear equation? 
\item The advantages and disadvantages of solving versus successive approximation? 
\item How to check that a solution is correct using the equation? 
\item[~] \textbf{If you're not sure, work the rest of exercises and then return to these questions.  Or, ask your instructor or a classmate for help.}
\end{itemize}

\subsection*{Exercises}

\begin{enumerate} 	
\setcounter{enumi}{4}

\item A charter boat tour costs $\$C$ for $P$ passengers, where
$$C = 135.00 + 11.95P$$ % Did not do variation on main example because we're really bored of the whole plumber thing!
\begin{enumerate}
\item Make a table of values showing the charges for no passengers, 4 passengers, 10 passengers, and 20 passengers. 
\item What does the 135.00 represent and what are its units?
\item What does the 11.95 represent and what are its units? 
\item If Freja was charged $\$ 326.20$ for use of the boat, how many passengers were there? Set up and solve an equation to answer the question.  
\item Graph and check.
\end{enumerate}

\item Abduwali has just opened a restaurant. He spent \$82,500 to get started but hopes to earn back \$6,300 each month.  Earlier we determined that $$A = 6,300M - 82,500$$ describes how Abduwali's profit \$$A$ is a function of how long he works ($M$ months). 

\hfill \emph{Story also appears in 2.1 Exercises}
\begin{enumerate}
\item Set up and solve an equation to determine how long it will take Abduwali to \textbf{break even}, meaning make a profit of \$0?
\item Aduwali will consider the restaurant a success once he's earned \$100,000.  According to our equation, when will that be?
\end{enumerate} 

\item Between e-mail, automatic bill pay, and online banking, it seems like I hardly ever actually mail something.   But for those times, I need postage stamps. The corner store sells as many (or few) stamps as I want for 44\textcent~each but they charge a 75\textcent~convenience fee for the whole purchase.  \hfill \emph{Story also appears in 1.1 Exercises}
\begin{enumerate}
\item Make a table showing the cost to buy 5 stamps, 10 stamps, or 20 stamps from the corner store.
\item Name the variables and write a linear equation showing how the total price depends on the number of stamps I buy.
\item My partner bought postage stamps at the corner store and it cost him \$7.35.  Solve your equation to determine how many stamps she bought. 
\item How many stamps could I buy for \$10?  Solve your equation and check your answer.
\end{enumerate} 

\item When Gretchen walks on her treadmill, she burns 125 calories per mile.  Recall $$C=125M$$ where $C$ is the number of calories Gretchen burns by walking $M$ miles. 

 \hfill \emph{Story also appears in 2.1 and 3.2 Exercises}
\begin{enumerate} 
\item Set up and solve an equation to calculate how far Gretchen has to walk to burn 300 calories. 
\item If Gretchen walks 3.4 miles per hour on her treadmill, how long will it take her to burn those 300 calories?  Report your answer to the nearest minute.
\item Pecan pie? Yum.  Not fitting into your favorite jeans?  No fun.   How far does Gretchen have to walk to burn off the calories from those two slices of pecan pie she ate last night?  Each slice has approximately 456 calories.
\end{enumerate} 

\item The more expensive something is, the less likely we are to buy it.  Well, if we have a choice.  For example, when strawberries are in the peak of season, they cost about \$2.50 per pint at my neighborhood farmer's market and demand is approximately 180 pints.  (That means, people want to buy about 180 pints at that price.) We approximate that the demand, $D$ pints, depends on the price, \$$P$, as described by the equation $$D = 305 - 50P$$
\begin{enumerate}
\item How many pints of strawberries are in demand when the price is \$3.19 per pint?
\item Make a table of values showing the demand for strawberries priced at  \$2.00/pint, \$2.25/pint, \$2.50/pint, \$2.75/pint, \$3.00/pint, \$3.25/pint, \$3.50/pint.
\item Draw a graph illustrating the function.  Start at \$0/pint even though that's not realistic.
\item It's been a great week for strawberries and there are 240 pints to be sold at my neighborhood farmer's. What price should the farmer charge for her strawberries in order to sell them all? Estimate your answer from the graph.  Then set up and solve an equation to answer the question.
\end{enumerate}

\item The stretch of interstate highway through downtown averages 1,450 cars per hour during the morning rush hour.  The economy is improving (finally), but with that the county manager predicts traffic levels with increase around 130 cars per hour more each week for the next couple of years. Earlier we found the equation $$C=1,450 + 130W$$ where $C$ is the number of cars per hour during the morning rush $W$ weeks since the country manage made her projection.   \hfill \emph{Story also appears in 2.1 Exercises}
\begin{enumerate}
\item Significant slowdown are expected if traffic levels exceed 2,000 cars per hour.  When do they expect that will happen? Set up and solve an equation.  Don't forget to check your answer by evaluating.
\item If traffic levels exceed 2,500 the county plans to install control lights at the on ramps.  When is that expected to happen?   Set up and solve an equation.  Don't forget to check your answer by evaluating.
\end{enumerate} 


\end{enumerate}
