%\section{Finance formulas}

\begin{center}
\line(1,0){300} %\line(1,0){250}
\end{center}

\section*{Homework}

\noindent \textbf{Start by doing Practice exercises \#1-4 in the workbook.}

\bigskip

\noindent \textbf{Do you know \ldots}

\begin{itemize} 
\item How to determine which formula to use? \emph{Ask your instructor if you will be told which formula to use during the exam.}  
\item What the quantities $a$, $p$, $y$, and $r$ from the formulas mean in the story? 
\item How to evaluate the formulas on your calculator?  \emph{Ask your instructor which formulas you need to remember, and whether any formulas will be provided during the exam.}
\item Why parentheses are needed around the exponent, numerator, and denominator in most of the formulas? 
\item What APR means, and why it is different from the (nominal) interest rate? 
 \item[~] \textbf{If you're not sure, work the rest of exercises and then return to these questions.  Or, ask your instructor or a classmate for help.} 
\end{itemize}

\subsection*{Exercises}

\begin{enumerate} 
\setcounter{enumi}{4}

\item As we have seen, Hector is trying to figure out his finances.  
\begin{enumerate}
\item Check that if Hector deposits \$\text{1,040} into a certificate of deposit earning 3\% interest compounded monthly, then at the end of three years he will have \$\text{1,137.81}.  Use the \textsc{Compound Interest Formula}.
\item Check that if Hector takes 10 years to pay back his student loan of \$\text{16,700} at 5.75\% interest compounded monthly, then his monthly payment will be  \$183.32.  Use the \textsc{Loan Payment Formula}.
\item Check that if Hector deposits \$816.58 each month into an account earning 4.5\% interest compounded monthly for 3 years, then his balance will be \$\text{31,410}.  Use the \textsc{Future Value Annuity Formula}.
\item What is the equivalent APR of 4.5\% interest compounded monthly?  Use the \textsc{Equivalent APR Formula}.
\end{enumerate}

\item \begin{enumerate}
\item If Ayah invests \$\text{35,000} for three years, how much will she have if her money earns each of the following rates compounded monthly? Use the \textsc{Compound Interest Formula}.
\begin{multicols}{4}
\begin{enumerate}
\item 6\%
\item 11\%
\item 1.9\%
\end{enumerate}
\end{multicols}
\item Name the variables, make a table, and draw a graph showing how her balance after three years is a function of the interest rate.  Include 0\% interest on your graph.
\end{enumerate}

\item Lue's family bought a house three years ago and owes \$\text{192,000} on their mortgage.  In reality, their monthly payment will include taxes, insurance, and money for escrow but let's ignore those amounts for this problem. In each part of this problem, use the \textsc{Loan Payment Formula}. 
\begin{enumerate}
\item They currently owe \$\text{192,000} on their mortgage for the remaining 27 years at 4.5\% compounded monthly.  Calculate their monthly payment. 
\item Lue's family can refinance at 3.5\% compounded monthly on a 30-year mortgage loan.  Rolling in closing costs, their new loan would be for \$\text{195,000}.  Calculate their monthly payment if they refinance.
\item Or, they can refinance to a 15-year mortgage at 3.25\% compounded monthly.  With closing costs, their new loan would be again be for \$\text{195,000}.  Calculate their monthly payment if they refinance this way instead.
\end{enumerate}

\item \begin{enumerate}
\item Make a table showing the balance now, after 1 year, after 5 years, and after 12 years if Kurt invests \$\text{50,000} in a certificate of deposit earning 4.77\% interest compounded monthly. Use the \textsc{Compound Interest Formula}.
\item Name the variables and draw a graph showing how the balance is a function of the time.  
\end{enumerate}

\item Soo Jin is borrowing more money for college. Compare the APR for each choice, using the \textsc{Equivalent APR Formula}.
\begin{enumerate}
\item A nationally subsidized loan at 3.4\% compounded monthly.
\item Her bank's ``college loan'' at 7.9\% compounded monthly.
\item Paying her tuition on her credit card that charges of 19.8\% compounded monthly.
\end{enumerate}

\end{enumerate}
