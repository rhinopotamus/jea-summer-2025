%\section{Solving exponential equations (and logs)}

\begin{center}
\line(1,0){300} %\line(1,0){250}
\end{center}

\section*{Homework}

\noindent \textbf{Start by doing Practice exercises \#1-4 in the workbook.}

\bigskip

\noindent \textbf{Do you know \ldots}

\begin{itemize} 
\item What ``log'' means? 
\item The connection is between logs and scientific notation? 
\item How to evaluate logs on your calculator? 
\item How to evaluate the \textsc{Log Divides Formula} using your calcuator? 
\item When to use the \textsc{Log Divides Formula}?  \emph{Ask your instructor if you need to remember the \textsc{Log Divides Formula} or if it will be provided during the exam.}
\item How to solve an exponential equation? 
 \item[~] \textbf{If you're not sure, work the rest of exercises and then return to these questions.  Or, ask your instructor or a classmate for help.} 
\end{itemize}

\subsection*{Exercises}

\begin{enumerate} 
\setcounter{enumi}{4}

\item The employee-paid cost of health insurance has risen dramatically, increasing by 7\% each year since 2003 when it cost \$420/month.  
\begin{enumerate}
\item Name the variables and write an exponential equation relating them.
\item If this rate of increase continues, when will or did the employee-paid cost pass \$550/month?  Solve your equation.
\item Repeat for \$600/month.
\item Graph the function.  
\end{enumerate}

\item The number of school children in the district from a single parent household has been on the rise.  In one district there were \text{1,290} children from single parent households in 2010 and that number was expected to increase about 3\% per year.  Earlier, we found the equation was $$C = \text{1,290}\ast1.03^Y$$ where $C$ is the number of children and $Y$ is the years since 2010.

\hfill \emph{Story also appears in 2.2 and 5.3 Exercises}
\begin{enumerate}
\item Use successive approximation to determine when there will be over \text{3,000} school children in the district from a single parent household. Display your work in a table.  Round your answer to the nearest year.
\item Show how to solve the equation to calculate when there will be over \text{3,000} school children in the district from a single parent household. Show how you solve the equation.
\item Solve again to determine when there will be over \text{3,500} children. Check your answer.
\end{enumerate}

\item Suppose a special kind of window glass is 1 inch thick and lets through only 75\% of the light.  If we use $W$ inches of window glass, it lets $L\%$ of the light through where $$L = 100\ast .75^W$$
\hfill \emph{Story also appears in 2.4 and 5.3 Exercises}
\begin{enumerate}
\item What thickness glass should be used to let through less than 10\% of the light?  \emph{Set up and solve an equation.}
\item What about 50\%? \emph{Set up and solve an equation.}
\item Check the graph (drawn before) to see if your answers make sense.
\end{enumerate}

\item We saw that poultry population was estimated to grow according to the equation $$P = 78 \ast 1.016^Y$$ where $P$ is the poultry population in million tons and $Y$ is the years starting in 2005.
\hfill \begin{footnotesize} Source:  Worldwatch Institute \end{footnotesize} 
\hfill \emph{Story also appears in 2.2 Exercises}
\begin{enumerate}
\item When will production rise above 95 million tons?  Set up and solve an equation.  Then use some other method to check.
\item Repeat for 120 million tons.
\end{enumerate}

\item Darcy likes to use temporary hair color in wild colors.  Good thing it washes out.  Her best guess is that 8\% of the color washes out each time she washes her hair.  That means the percentage of color remaining, $C$, is a function of the number of times she washes her hair, $W$, according to the equation $$C=100 \ast .92^W$$
\begin{enumerate}
\item When will half the color be gone?  That means find $C=50\%$.  Set up and solve an equation.  Then check some other way.
\item By the time only 10\% of the color remains you really can't tell anymore if it was pink or orange or blue.  So, she might as well switch to a new color then.  How many washes before only 10\% remains?  Again, first solve.  Then check.
\item Draw a graph showing how the color washes out of Darcy's hair.
\end{enumerate}

\end{enumerate}

%%%%%%%%%%%%%%%%%%

