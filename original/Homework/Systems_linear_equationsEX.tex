%\section{Systems of linear equations}

\begin{center}
\line(1,0){300} %\line(1,0){250}
\end{center}

\section*{Homework}

\noindent \textbf{Start by doing Practice exercises \#1-4 in the workbook.}

\bigskip

\noindent \textbf{Do you know \ldots}

\begin{itemize} 
\item How to compare two linear functions using a table? 
\item How to graph two linear functions on the same axes? 
\item What the solution of a linear system means in terms of the story? 
\item Where to look on a graph to see the solution of a linear system? 
\item How to successively approximate the solution of a linear system? 
\item How to solve a linear system? 
\item When to use inequality instead of an equation for a linear system? 
 \item[~] \textbf{If you're not sure, work the rest of exercises and then return to these questions.  Or, ask your instructor or a classmate for help.}  
\end{itemize}

\subsection*{Exercises}

\begin{enumerate} 
\setcounter{enumi}{4}

\item Just when Quia Xun thought she had things figured out, another possible option emerged.  She can outsource her lock production to a different plant at the cost of  \$1.55 per lock.   While that's more expensive per lock, it avoids having to buy a machine at all.  Recall her first two options were
\begin{center}
\begin{tabular} {ll}
$$\textbf{Machine \#1:} & E  =3,200 + 1.25L$$ \\
$$\textbf{Machine \#2:} & E  =5,400 + 0.80L$$ \\ 
\end{tabular}
\end{center}
where $E$ is the total expenditure to produce $L$ locks on that machine.  
\begin{enumerate}
\item  Write an equation showing how $E$ depends on $L$ if Quia Xun decides to outsource lock production.
\item For what number of locks does outsourcing make financial sense versus Machine \#1?  Set up and solve a system of equations.
\item Repeat for Machine \#2.
\end{enumerate}  

\item Don't ask why I know this, but it takes me 80 seconds per square foot to wash a floor using a rag.  If I use a mop, it's slightly quicker at 75 seconds per square foot, but another 3 minutes at the end to wring out the mop.  (The rag just pops in the washing machine.)
\begin{enumerate}
\item Name the variables and write an equation for each option: rag versus mop.  

\emph{Hint:  time is the dependent variable}
\item Set up and solve the system to determine the area of a room where either option takes the same amount of total time.
\item What do you suggest for each room?  (Note: find the area by multiplying the length times the width.)  My bathroom floor is 5'$\times$5'.   My kitchen floor is 8'$\times$10'.  The laundry room is 10'$\times$14'.
\end{enumerate}

\item Maria needed to replace the light bulb in the hallway.  When she went to the home improvement store she was overwhelmed with the choices of light bulbs.  One option is a compact fluorescent light  (CFL) bulb.  A CFL bulb costs \$1.  This fixture will cost \$0.95 per month to run with a CFL bulb.  A different option Maria is considering is a light-emitting diode (LED) instead of a bulb.  A LED costs \$24 but reduces the cost for the fixture to  \$0.60 per month.    
\begin{enumerate}
\item Name the variables and write an equation for each option: LED versus CFL.
\item Compare the total cost for each bulb for 1, 6, 18, and 36 months.
\item Draw a graph showing both functions on the same axes.
\item Set up and solve a system of linear equations to determine the payback time.
\item Based on what you've learned, fill in the blank and circle the correct word.
\begin{quote}
The more expensive LED pays off if Maria is going to use it for \underline{\quad~}  or more months.  
\end{quote}
\end{enumerate}

\item The community center offers exercise classes on a pay-as-you-go basis.  It normally costs \$20 to register and then \$15 per exercise class.  Alternatively, you can pay \$150 to become a member, and then \$10 per exercise class.  If we let $E$ represent the number of exercise classes I attend and $T$ represent the total cost in dollars, then the equations are:
\begin{center}
\begin{tabular} {ll}
\textbf{Pay as-you-go:} & $T = 20 + 15E$ \\
\textbf{Member:} & $T= 150 + 10E$ \\ 
\end{tabular}
\end{center}  
\begin{enumerate}
\item Create a table of reasonable values and draw a graph showing both options.
\item According to your graph, what is the break even point?  
\item Use successive approximations to find the break even point.
\item Set up and solve a system of linear equations to determine the exact break even point
\item Describe in words what you've learned about whether or not to buy membership.
\end{enumerate}

\item The weekly supply $S$ and demand $D$ for corn-on-the-cob (in hundreds) at a local market are given by the equations 
\begin{center}
\begin{tabular} {ll}
\textbf{Supply:} & $S = 1.5P+6$ \\
\textbf{Demand:} & $D= 11 - 0.9P$ \\ 
\end{tabular}
\end{center} 
where $P$ is the price per unit, in dollars per dozen.
\begin{enumerate}
\item Is there a shortage or surplus if corn is priced at \$3.25/dozen?  What if it's priced at \$1.75/dozen?
\item Set up and solve an equation to find the equilibrium price of corn.
\item Make a small table of values and graph.  Does your answer to (b) agree with your graph?  Explain.
\end{enumerate}

\item A solar heating system costs approximately \$30,000 to install and \$150 per year to run.  By comparison, a gas heating system costs approximately \$12,000 to install and \$700 per year to run.   Earlier we wrote equations showing how the total cost \$$T$ depends on the time, $Y$ years.
\begin{center}
\begin{tabular} {ll}
\textbf{Solar:} & $T  =30,000 + 150Y$\\
\textbf{Gas:} & $T  = 12,000 + 700Y$ \\ 
\end{tabular}
\end{center}

\hfill \emph{Story also appears in 4.1 \#1}
\begin{enumerate}
\item Set up an solve a system to determine the payback time of installing the solar heating system.
\item How does the payback time change if the state offers a \$7,000 rebate?  That means, in effect, that the solar system costs \$7,000 less to install. Write a new equation for the solar heating system.  Then set up and solve the new system.
\item How high a rebate would the state have to offer to insure a payback time of 15 years?  \emph{Hint:  compare the costs of gas and solar heating systems at 15 years.}
\end{enumerate}

\end{enumerate}
