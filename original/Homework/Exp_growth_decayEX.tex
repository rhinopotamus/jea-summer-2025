%\section{Exponential growth and decay}

 \begin{center}
\line(1,0){300} %\line(1,0){250}
\end{center}

\section*{Homework}

\noindent \textbf{Start by doing Practice exercises \#1-4 in the workbook.}

\bigskip

\noindent \textbf{Do you know \ldots}

\begin{itemize}
\item How to write an exponential equation given the starting amount and percent decrease? 
\item What ``half-life'' means? 
\item What ``doubling time'' means?     
\item What the graph of exponential growth and exponential decay look like? 
\item Why the rate of change for exponential decay is negative? 
 \item[~] \textbf{If you're not sure, work the rest of exercises and then return to these questions.  Or, ask your instructor or a classmate for help.} 
\end{itemize}

\subsection*{Exercises}

\begin{enumerate} 
\setcounter{enumi}{4}
\item Joe's girlfriend Ceyda starts the day by downing two cans of Red Bull, containing a total of 160 mg of caffeine.  Her body eliminates the caffeine at a slightly slower rate of 12\% each hour.  
\begin{enumerate}
\item Name the variables and write an equation to model this situation.  
\item What's the half-life of caffeine for Ceyda?  Set up and solve an equation.
\item Ceyda heard that drinking a glass of water an hour can help eliminate caffeine faster.  If so, would the half-life be shorter or longer?  Explain.
\end{enumerate}

\item  The population of bacteria in a culture dish begins at 2,000 and will triple every day.  
\begin{enumerate}
\item Name the variables, including units and dependence.
\item Make a table showing the number of bacteria at the start and after 1 day, 2 days, and 3 days.
\item Write an equation illustrating bacterial growth.
\item Your equation should fit the template for an exponential equation. What is the daily growth factor?
\item Use your equation to extend your table to include 10 days, 20 days, and 30 days. Be careful to report the large numbers appropriately.  \emph{Hint: scientific notation}
\item The dish can support around 1 million bacteria.  When does that happen?  Give your answer to the nearest hour.
\item Draw a graph, including a table of reasonable values.  Remember, the dish can only support up to 1 million bacteria, so your graph should go up to 1 million.
\end{enumerate}

\item Tenzin bought a house for \$291,900 but the housing market collapsed and his house value dropped 4.1\% each year. 
 \begin{enumerate}
\item Name the variables and write an equation relating them.
\item At this rate, how many years would it take for the value of Tenzin's house to drop below \$240,000?  Use successive approximation to guess the year.
\item Now set up and solve an equation.
\end{enumerate}

\item One modern technique for cleaning waste water involves the use of constructed (man-made) wetlands.  Wetlands act as a natural biofilter for various contaminants in the waste water. After each month in the wetlands, only about \nicefrac{1}{4} of the contaminants remain in any given sample.  Suppose a sample had 8 grams of contaminants before processed in the constructed wetlands.
\begin{enumerate}
\item How much would remain in the sample after 1 month?  2 months?  3 months?
\item Name the variables and write an equation relating them.
\item Your equation should fit the template for an exponential equation. What is the monthly decay factor?
\item According to your equation, when will the contaminants fall below 1 mg?  1 $\mu$g?  Remember $1 \text{ gram} = \text{1,000 mg}$ and $1 \text{ gram} = \text{1,000,000 } \mu \text{g}$
\item Draw a graph illustrating the waste water treatment process for the first 6 months.
\end{enumerate}  % DATA is wastewater.xlsx

\item Hibbing, Minnesota is the hometown of baseball star Roger Maris, basketball great Kevin McHale, the Greyhound Bus lines, the Hull-Rust-Mahoning Open Pit Iron Mine and, perhaps most famously, the childhood home of songwriter Bob Dylan.
It is not a big town.  In 2000 the population  was reported at  17,071 residents, with an expected decrease of around 0.4\% per year.
\begin{enumerate}
\item What is the annual decay factor?
\item Name the variables and write an equation relating them.
\item Based on these estimates, what was the anticipated population of Hibbing in 2010?
\item The actual 2010 U.S.\ Census estimate of Hibbing's population was 16,361 people.  Was the decrease slower or faster than expected?  Was the decay rate more or less than 0.4\%?  Explain.
\end{enumerate}

\item Donations to a local food shelf have increased 35\% over last year.  There were 3,400 pounds of food donated last year. \hfill \emph{Story also appears in 5.3 \#4}
\begin{enumerate}
\item Name the variables, measuring time since last year.  (Yes, it's a little awkward.)
\item Write an exponential equation relating the variables, assuming the rate of increase continues.
\item How much was donated this year?  How much is expected next year?  The year after?
\item At this rate, what is the doubling time?  That means, how long would it be until twice as much is donated.  Set up and solve an equation to answer.  Report your answer to the nearest month.
\item The agency estimates that around 12,000 pounds of food is needed each year to meet the needs of the people they serve.  In how many years will donations reach that level?  (In your answer, say how many years from now.)
\end{enumerate}


\end{enumerate}