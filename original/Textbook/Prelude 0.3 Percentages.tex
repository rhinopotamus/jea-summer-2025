\section{Prelude: Percentages}

In a recent basketball game against the Dallas Wings, Minnesota Lynx star Naphessa Collier took 16 free throws and made 11 of them. The fraction, or proportion, of free throws she made is $$\frac{11}{16} = 11 \div 16 = 0.6875$$

To calculate Collier's free throw percentage, we need to remember how percents work.  Luckily, the word ``percent'' is very descriptive.  The ``cent'' part means ``hundred,'' like 100 cents in a dollar or 100 years in a century.  And, as usual, ``per'' means ``for each.''  Together, \textbf{percent} means ``per hundred.''  For example, the number 20\% means 20 for each hundred.  Written as a fraction it is $\frac{20}{100}$.  Divide to get the decimal $20 \div 100 = .20.$   
$$\text{Think money:  }20\%\text{ is like }20\text{\textcent , and }.20\text{ is like }\$.20$$  
Bottom line:  20\%, $\frac{20}{100}$, and .20 mean exactly the same number.
$$20\% = \frac{20}{100} = 20 \div 100 = .20$$
Since we divided by 100 to go from percentage to decimal, we can reverse this process and go from decimal to percentage by multiplying by 100.  Check it out:
$$.20 \times 100 = 20\%$$

Back to Collier.  She made 11 out of 16 free throws, which was $ 0.6875$ of her free throws.  Now we know
$$ 0.6875 \times 100 = 68.75 \approx 68.8\%$$
Her free throw percentage that game was 68.8\%.  
Collier's free throw percentage for the season is actually much lower, at 48.1\%. How many free throws out of 16 would we have expected she would make?  We need to figure out what 48.1\% of 16 is.  First, we convert 48.1 to decimal by dividing by 100:
$$48.1 \div 100 = .481 $$
Next, we multiply that proportion by 16:
$$ .481 \times 16 = 7.696$$
We can even do this calculation in just one step:
$$48.1 \div 100 \times 16 = 7.696$$
Either way, would have expected Collier to make 7 or 8 of her free throws.  So, 11 was a really good night. Turns out she scored 29 points that game with 7 rebounds.

Thang attended that Lynx-Wings game and ordered the hot chicken sandwich with pickles for \$14.59.  When she tapped her debit card to pay, the machine offered her three choices for tip: 10\%, 15\% or 20\%. As you can check, the machine calculated that 10\% tip would add \$1.46, the 15\% tip would add \$2.19, and the 20\% tip would add \$2.92.   Knowing that the concession staff that night was fundraising for a group she supports, Thang selected 20\% and her total bill was $14.59 + 2.92 = \$17.51$.  

Thang's ticket was good for 10\% off the purchase of Lynx merchandise from their website and there was free shipping, so she decided to buy a jersey normally priced at \$99. Since 10\% of \$99 is $$.10 \times 99 = 9.9$$ she knows that her discount will be \$9.90.  (I added the extra 0 at the end because \$9.9 looked strange.) The net price for the jersey will be $$\$99 - \$9.90 = 99 - 9.9 = 89.1 = \$89.10$$
With the discount, Thang can buy the jersey for \$89.10.

In general, we can find the result of a \textbf{percent increase}, like Thang's sandwich plus tip, by calculating the percent and adding it to the original amount.
And we can find the result of a \textbf{percent decrease}, like Thang's on-sale jersey, by calculating the percent and subtracting it from the original amount.

%By the way, there is a quicker way to calculate the result of a percent increase.  Thang paid the full \$14.59 plus another 20\% of \$14.59.  That means she paid the full 100\% plus another 20\% more.  So she actually paid $100\% + 20\%$, which is $120\%$ of the stated price.  That works in general. When we increase a number by 20\%, we end up with 120\% of what we started with.  Note that $120\% = 120 \div 100 = 1.20$, so we can just multiply by 1.20.  Check it out:  $$14.59 \times 1.20 = 17.508 \approx \$17.51$$
%
%Similarly, there is also a quicker way to calculate a \textbf{percentage decrease}.  Thang would have paid the full 100\% but they reduced the price by 10\%, so actually Thang paid $100\%-10\% = 90\%$ of the original price.  That works in general.  When we decrease a number by 10\%, we end up with 90\% of what we started with.  Note that $90\% = 90 \div 100 = .9$, so we can just multiply by .9.  Check it out:  $$99 \times .9 = 89.1$$
%
%Feel free to continue to go through the process step-by-step instead of using these quicker method.  Just thought you might like to know, and it will be useful to us later in the course.

 \begin{center}
\line(1,0){300} %\line(1,0){250}
\end{center}

\section*{Homework}

\noindent \textbf{Start by doing Practice exercises \#1-4 in the workbook.}

\bigskip

\noindent \textbf{Do you know \ldots}

\begin{itemize}
\item What the words ``per'' and ``cent'' mean in the word ``percent.''
\item How to convert a fraction or decimal to a percent? 
\item How to convert a percent to a decimal? %\vfill
\item How to calculate a percentage of a number? %\vfill
\item How to calculate the result of a percent increase or a percent decrease? %\vfill
%\item How to use the distributive property to do percent increase or percent decrease using a single multiplication? %\vfill
 \item[~] \textbf{If you're not sure, work the rest of exercises and then return to these questions.  Or, ask your instructor or a classmate for help.} 
\end{itemize}

\noindent \textbf{If you're not sure, work the rest of exercises and then return to these questions afterwards.  Or, ask your instructor or a classmate for help.}

\subsection*{Exercises}

On each problem, write down what you enter into your calculator and don't forget to write the units on your final answer.  You are welcome to calculate the answer step-by-step but challenge yourself to figure out the answer all at once, not hitting $=$ on your calculator until the very end.

\begin{enumerate} 
\setcounter{enumi}{4}

\item \begin{enumerate}
\item Check for yourself that the tip of Thang's \$14.59 sandwich would be \$1.46 if she had chosen 10\%, \$2.19 if she had chosen 15\%, and \$2.92 when she chose 20\%.
\item What would the net cost of Thang's jersey be if she had a 20\% discount (off the \$99 price) instead?
\end{enumerate}

 \item Donations to a local food shelf have increased 35\% over last year.  There were \text{3,400} pounds of food donated last year. How many pounds of food were donated this year? 
 
 \hfill \emph{Story also appears in 5.2 \#9 and 5.3 \#4} 

\item Ceyda starts the day by downing two cans of Red Bull, containing a total of 160 mg of caffeine.  Her body eliminates the caffeine at the rate of 12\% each hour. How much caffeine is left in her blood after 1 hour?  \hfill \emph{Story also appears in 5.2 \#5}

\item Too much salt can be difficult for your body.  The Centers for Disease Control and Prevention (CDC) suggest that adults limit their daily intake of Sodium to a maximum of 2,300 mg per day.
\begin{enumerate}
\item Omer ate a snack sized bag of chili-flavored corn chips containing 9\% of the daily maximum allowance of Sodium.  How many milligrams of Sodium did he eat?
\item Selu ate the lightly salted corn chips instead containing 80 mg Sodium.  What percentage of the daily maximum allowance of Sodium is in the lightly salted corn chips?
\end{enumerate}

\item Tenzin bought a house for \$291,900 but the housing market collapsed and his house value dropped 4.1\% since he bought it.  What is Tenzin's house worth now?

 \hfill \emph{Story also appears in 5.2 \#7}

\item \emph{Story also appears in 2.2 \#5}
 \begin{enumerate}
\item Mai's salary was \$78,000 before she got a 6\% raise. What was her salary after the raise?
\item The following year Mai only got a 1.5\% raise.  What was her salary after this second raise?  Be careful to use your answer to part (a) to compute the 1.5\% increase.
\end{enumerate}

\end{enumerate}

\bigskip
\newpage

\noindent \textbf{When you're done \ldots}

\begin{itemize}
\item Don't forget to check your answers with those in the back of the textbook. 
\item Not sure if your answers are close enough? Compare with a classmate or ask the instructor.  
\item Getting the wrong answers or stuck on a problem?  Re-read the section and try the problem again.   If you're still stuck, work with a classmate or go to your instructor's office hours.
\item It's normal to find some parts of some problems difficult, but if all the problems are giving you grief, be sure to talk with your instructor or advisor about it.  They might be able to suggest strategies or support services that can help you succeed.
\item Make a list of key ideas or processes to remember from the section.  The ``Do you know?'' questions can be a good starting point.
\end{itemize}

