~\vspace{.1in}

\section{Metric prefixes and scientific notation} 

Tara is working on a big project at work.  She wants to back up her files to her online drop box. The site says she has 72 GB of memory remaining.  Tara has about 200 files at an average of 42.3 MB each that she would like to upload.  Will she have room?

To answer Tara's question we need to know that GB is short for ``gigabyte'' and MB is short for ``megabyte.''  A \textbf{byte} is a very small unit of computer memory storage space just enough for about one letter.  You may have heard the word ``mega'' used to mean ``really big.''  There's a reason for that.  \textbf{Mega} is short for 1 million.  That's pretty big.  But \textbf{giga} stands for 1 billion, so that's even bigger.  (Maybe it's time for a gigamall?) 
\begin{center}
\begin{tabular} {lclcr} 
\textbf{mega} &$=$&1 \textbf{ million} &$=$&$ \text{1,000,000}$\\
\textbf{giga} &$=$& 1 \textbf{ billion} &$=$& $\text{1,000,000,000}$\\ 
\end{tabular}
\end{center}

What all this means is Tara has
$$72\text{ GB} = 72 \text{ billion bytes} = \text{72,000,000,000 bytes}$$
of memory remaining.
She would like to save 200 files at 42.3 MB each which comes to
$$200 \times 42.3 = \text{8,460 MB}$$
which is really $$ \text{8,460 MB} = \text{8,460 million bytes} = \text{8,460,000,000} \text{ bytes}$$
Finding it hard to compare all those zeros?  Try this.
$$ \text{8,460 MB} = \text{8,460,000,000 bytes} = 8.46 \text{ GB } < 9 \text{ GB}$$
So Tara wants to store less than 9 GB of information and she has 72 GB remaining.  She has plenty of room.  Save away.

Tara also needs to download about 700 MB of rather high quality photos.  Her computer downloads photos at 187 kbps.  How long will it take?  (And does she have time to run for a coffee?)  The mysterious \textbf{kbps} stands for kilobits (Kb) per second.  Like mega and giga, the word ``kilo'' stands for a large number, in this case \text{1,000}.  
\begin{center}
\begin{tabular} {lclcr} 
\textbf{kilo} &$=$&1 \textbf{ thousand} &$=$&$ \text{1,000}$\\
\end{tabular}
\end{center}
That's the same word ``kilo'' as in kilometer (about \nicefrac{1}{2} mile) or kilogram (about \nicefrac{1}{2} pound) and there's good reason for that as  
\begin{center}
\begin{tabular} {lclcr} 
\textbf{kilometer} &$=$&$ \text{1,000 meters}$\\
\textbf{kilogram} &$=$&$ \text{1,000 gram}$\\
\end{tabular}
\end{center}
Perhaps you've seen the letter \textbf{K} as short for a thousand?  That's where it comes from. 

(Okay, I have to mention something here.  Kilo by itself is pronounced ``KEE-loh,'' but kilogram is pronounced ``KIL-uh gram,'' and kilometer is pronounced ``ki-LOM-i-ter.''   Well, around these parts at least.)

Back to Tara.  Her download speed is 187 kilobits per second. Perhaps this is the right moment to mention that a \textbf{bit} is even smaller than a \textbf{byte}.
$$1 \text{ byte} = 8 \text{ bits}$$
How long will it take Tara to download 700 MB?  We can think of this calculation as a unit conversion by imagining.
$$187 \text{ kilobits} = 1 \text{ second}$$  Watch.
$$ 700 \text{ \cancel{MB}} 
\ast \frac{\text{1,000,000 \cancel{bytes}}}{1 \text{\cancel{MB}}}
\ast \frac{8 \text{ \cancel{bits}}}{1 \text{ \cancel{byte}}} 
\ast \frac{ 1 \text{ \cancel{kilobit}}}{\text{1,000 \cancel{bits}}}
\ast \frac{\text{second}} {187 \text{ \cancel{kilobits}}}
$$
$$= 700 \times \text{1,000,000} \times 8 \div \text{1,000} \div 187
= 29,946.524\ldots\text{ seconds} $$
Let's convert to a more reasonable unit.  
$$\text{29,946.524\ldots \cancel{seconds}} 
\ast \frac{1 \text{ \cancel{minute}}}{60 \text{ \cancel{seconds}}} 
\ast \frac{1 \text{ hour}}{60 \text{ \cancel{minutes}}} $$
$$ = 29,946.524\ldots \div 60 \div 60 = 8.318\ldots \approx 8.32 \text{ hours}$$
It will take Tara over 8 hours to download those photos.  
Perhaps Tara should compress the photos into a zip file or use lower resolution or find a way to download faster.  Or, she can just set it up to download overnight.  

Quick note.  The \textbf{metric system of measurement}, or \textbf{international system of units (ISU)}, is the official system of nearly all countries, the United States being a notable exception.  Science, international trade, and most international sporting events like the Olympics are based in the metric system.  
In the United States system (known officially at the \textbf{British system} or, since the British stopped using it, the \textbf{imperial system of measurement}), we have all sorts of difficult to remember conversions. 
One notable feature of the metric system is that most units come in sizes ranging from small to large:  the \textbf{(metric) prefixes} like kilo, mega, or giga tell us which size.

Really large numbers, like \text{8,460,000,000}, are awkward to read and awkward to work with.  We have seen how metric prefixes  allow us to rewrite these large numbers in a way that's much easier both to read and to work with.  There's another option that's used often in the sciences (and by your calculator).  To explain it we need to understand powers of 10.

 Perhaps you know what happens when we multiply a number by 10, like
$ 5 \times 10 = 50$ or, more appropriate to our example, $$ 8.46 \times 10 = 84.6$$
The effect of multiplying by 10 is to move the decimal point one place to the right.
When we multiply by \text{1,000} we get
$ 5 \times \text{1,000} = \text{5,000} $ or, for our example, $$8.46 \times \text{1,000} = \text{8,460}$$
The effect of multiplying by \text{1,000} is to move the decimal point three places to the right.
The connection is that $$10 \times 10 \times 10  =\text{1,000}$$
Each $\times 10$ has the effect of moving the decimal point one place to the right so $\times \text{1,000}$ has the same effect as multiplying by 10 three times, so the decimal point moves three places to the right.
That means 
\begin{eqnarray*}
\text{8,460,000,000} & = & 8.46 \underbrace{\hbox{$\times 10 \times 10  \times 10  \times 10  \times 10  \times 10  \times 10  \times 10  \times 10 $}}_{\hbox{\begin{tiny} 9 times \end{tiny}}}  \\  %HSPACE
& = & 8.46 \ast 10^ 9 \\  
\end{eqnarray*} 
\vspace{-.5in} %VSPACE

\noindent Since we're multiplying by the same number (10) over and over again, it's easier to use \textbf{exponential notation}. Here 10 is the \textbf{base} and 9 is the \textbf{exponent} (or \textbf{power}).  In this context, the exponent is also called the \textbf{order of magnitude}. % repeat idea in 3.4

The point of this calculation was that $$\text{8,460,000,000} = 8.46\ast10^9$$
This shorthand is called \textbf{scientific notation}.  The base is always 10.  The exponent is always a whole number.  The number out front, like 8.46 in our example, is always between 1 and 10, which means there's exactly one digit before the decimal point (and any others must come afterwards).  It is customary to use $\times$ instead of $\ast$ in scientific notation, so we should write
$$\text{8,460,000,000} = 8.46\times10^9$$
As another example, we saw earlier that $$\text{5,000} = 5 \times \text{1,000} = 5 \times 10^3$$

Most calculators use the $\wedge$ symbol for exponents, as do most computer software packages. Two other notations calculators sometimes use are $y^x$ or $x^y$.  Sometimes that operation is accessible through the 2nd or shift key; something like SHIFT $\times$.  If you're not sure, ask a classmate or your instructor.  For practice, check that
$$5 \times 10^3 = 5 \times 10 \wedge 3 = \text{5,000} \quad \checkmark$$
 Notice that the order of operations is exactly what we wanted here: $5 \times 10 \wedge 3$ first raises 10 to the $3^{\text{rd}}$ power and then multiplies by 5.  So we can enter it all at once without needing parentheses.
 
Here's the full list of the  \textbf{order of operations}, the priority ranking for arithmetic operations.

\bigskip
 \framebox{
 \begin{minipage}[c]{.85\textwidth}  
~ \vspace{.05in} \\  \textsc{Order of operations:}  %VSPACE
\begin{center}
\begin{tabular} {cl}
~\hspace{.35in} ~ & First, calculate anything inside \textbf{P}arentheses. \\ %HSPACE
& Next, calculate \textbf{E}xponents $\wedge$, in order from left to right. \\
& Then, \textbf{M}ultiply $\times$ and \textbf{D}ivide $\div$, in order from left to right.\\
& Last, \textbf{A}dd $+$ and \textbf{S}ubtract $-$, in order from left to right. \\ 
\end{tabular}
\end{center}
\vspace{.05in} %VSPACE
\end{minipage}
}
\bigskip

\noindent We highlighted the letters PEMDAS  which often helps people remember this order. (``Please Excuse My Dear Aunt Sally'' is how I learned it.)  The good news is that your calculator does the operations in exactly this order.  And if you want something in a different order, all you need to do is use parentheses around quantities you want calculated first.
 
Back to our large number.  Enter $$8.46 \times 10 \wedge 9=$$
What do you see?  Some calculators correctly list out $\text{8,460,000,000}$ while others report the number back in scientific notation, which is not particularly useful. (Sigh.)  

Let's try a number so big that (nearly) every calculator will switch to scientific notation.  Enter $$2.7\times 10 \wedge 30=$$
Look carefully at the screen.  Your calculator might display something like 
$$\boxed{~2.70000000 \quad \text{\begin{footnotesize}E \end{footnotesize}}~30~} \text{ \quad or \quad } \boxed{~2.70000000 \quad _{ \times \text{\begin{tiny}10 \end{tiny}}}~30~}$$ 
Whatever shorthand your calculator uses, you should write $$2.7 \times 10^{30}$$

Interested in what that number is in our usual decimal notation?  It's
$$2, \underbrace{ \hbox{$\text{700,000,000,000,000,000,000,000,000,000} $}}_{\hbox{\begin{tiny} decimal point moves 30 places \end{tiny}}}$$

Enough of that. Poor Tara is pulling her hair out over this project.  Well, not literally, but she is quite frustrated over how slowly the project is going.  She wonders: how thick is a human hair?  And, how many hairs would you need to lay out to span an inch?

Turns out that a typical human hair is about $.00012$ meters across.  Very small numbers are also awkward to read and awkward to work with.  In this section, we write  $.000~12$ where the strange-looking space is to help you read the number.  Of course, a better solution is to use metric prefixes to get more appropriate units, just as we did for large numbers.

For example, \textbf{centi} is short for 1 in a hundred, or $.01$.  Not surprising since one cent is \$.01, or one percent is 1\%=.01.  That's the same word ``centi'' as in centimeter (about \nicefrac{1}{2} inch) and there's good reason for that as
\begin{center}
\begin{tabular} {lclcr} 
$ \text{1 meter}$&$=$&100 \textbf{centimeter} \\
\end{tabular}
\end{center}
Similarly, \textbf{milli} is short for 1 in a thousand and \textbf{micro} is short for 1 in a million.
\begin{center}
\begin{tabular} {lclcr} 
\textbf{centi} &$=$&1 in a hundred &$=$&.01\\
\textbf{milli} &$=$&1 in a thousand &$=$&.001\\
\textbf{micro} &$=$&1 in a million &$=$&.000~001\\
\end{tabular}
\end{center}


What about that human hair?  It is convenient to measure in micrometers
using that $$1 \text{ meter} = \text{1,000,000 micrometers} $$
The width of a human hair in micrometers (abbreviated $\mu m$ in the sciences) is
$$.000~12 \text{ \cancel{meters} } \ast \frac{\text{1,000,000 } \mu \text{m}}{1 \text{ \cancel{meter}}} =  .000~12 \times \text{1,000,000} = 120 \mu \text{m}$$
The $\mu$ symbol is the Greek letter {\it mu}, but we'll just read $\mu m$ as micrometers.  

To answer Tara's question about how many hairs in an inch, we recall that 
$$1 \text{ inch} \approx 2.54 \text{ cm}$$
where cm is short for centimeter.  Ready to convert?
$$ 1 \text{ \cancel{inch}} \ast \frac{ 2.54 \text{ \cancel{cm}} }{1 \text{ \cancel{inch}}}
\ast \frac{1 \text{ \cancel{meter}}}{100 \text{ \cancel{cm}}} 
\ast \frac{\text{1,000,000  \cancel{$\mu$m}}}{1 \text{ \cancel{meter}}} 
\ast \frac{1 \text{ hair}}{120 \text{ \cancel{$\mu$m}}}$$ 
$$ = 2.54 \div 100 \times \text{1,000,000} \div 120 = 211.66666\ldots \approx 200 \text{ hairs}$$

We can also describe really small numbers using scientific notation.  Perhaps you know what happens when we divide a number by 10, like
$ 50 \div 10 = 5$ or, more appropriate to our example, $$ 1.2 \div 10 = .12$$
The effect of dividing by 10 is to move the decimal point one place to the left.
If we divide by \text{1,000,000} instead, we get
$$ 1.2 \div \text{1,000,000} = .000~001~2 $$ 
The connection is that $$ \text{1,000,000} = 10 \wedge 6  $$
and so dividing by \text{1,000,000} moves the decimal point six places to the left.  Notice that we have to introduce zeros as placeholders.

The width of a hair was .00012 meters.  To get that number from 1.2, we need to move the decimal point 4 places to the left.  $$1.2 \div 10^4 = 1.2 \div \text{10,000} = .000~12$$
The shorthand for dividing by a power is to use negative exponents.  For example
$$ \div 10^4 = \times 10^{-4}$$
It has nothing to do with negative numbers.  It's just a shorthand.
The point of this calculation was that $$.00012= 1.2\ast10^{-4}$$

Once again we have scientific notation.  The base is still 10.  The exponent is still a whole number, although now it's negative.  The number out front, like 1.2 in our example, is still between 1 and 10, which means there's exactly one digit before the decimal point (and any others must come afterwards).  As before it customary to use $\times$ instead of $\ast$ in scientific notation, so we should write
$$.000~12= 1.2 \times 10^{-4}$$
When you see a number written in scientific notation, the power of 10 tells you a lot.  For example, $6.7 \times 10^4 = 67,000$ and $6.7 \times 10^{-3} = .006~7$.  A positive power of 10 says you have a big number, and a negative power of 10 says you're dealing with a very small number.  

 %\section{Metric prefixes and scientific notation}

\begin{center}
\line(1,0){300} %\line(1,0){250}
\end{center}

\section*{Homework}

\noindent \textbf{Start by doing Practice exercises \#1-4 in the workbook.}

\bigskip

%SU should this be here:% SU -- check DoYouKnows because this is the first occurence of powers.

\noindent \textbf{Do you know \ldots}

\begin{itemize} 
\item How to calculate powers on your calculator?
\item What  million, billion, and trillion mean?  
\item Why metric prefixes are used?  
\item What common metric prefixes (mega, giga, kilo, centi, milli, micro, nano) mean? 

\emph{Ask your instructor which prefixes you need to remember, and whether any prefixes will be provided during the exam.} 
\item Why scientific notation is used?  
\item The standard format for scientific notation?  
\item What kinds of numbers have a positive order of magnitude, and which have a negative order of magnitude?
\item How to convert between decimal notation and scientific notation?  
\item How your calculator reports numbers in scientific notation, and what (might be) different when you're reporting that number?  
%\item How to enter numbers written in scientific notation into your calculator? 
\item The usual order of operations (PEMDAS) and how to use parentheses when you want a different order?
 \item[~] \textbf{If you're not sure, work the rest of exercises and then return to these questions.  Or, ask your instructor or a classmate for help.} 
\end{itemize}

\subsection*{Exercises}

\begin{enumerate} 
\setcounter{enumi}{4}

\item \begin{enumerate}
\item How many files at an average of 42.3 MB each can each gig (1 GB) of computer memory hold?
\item Tara's coworker Brandon has a much faster Internet connection on his computer at \text{1,500} kbps.  How long would it take Brandon to download 700 MB?  
\item At that rate, how much information could Brandon upload in 8 hours?  Express your answer in kilobytes (KB).
\end{enumerate}

\item \begin{enumerate}
\item Convert each of these amounts of time into an understandable unit of time: 1 million seconds, 1 billion seconds, 1 trillion seconds. 
\item Billy Bob wants to throw a party when he turns 1 billion seconds old. About how many years old will he be?
\item \emph{Bonus question:}  On what date were you or will you be 1 billion seconds old?  Don't forget leap years! \hfill \begin{footnotesize} Source:  Mathew Foss, North Hennepin Community College \end{footnotesize} % YES, only one t in Mathew.
\end{enumerate}  

\item  A proton has mass of about $1.67262 \times 10^{-27}$ kg, while an electron has mass of about $9.10938 \times 10^{-31}$ kg. 
\begin{enumerate}
\item Write out the mass of a proton and that of an electron in normal decimal notation.
\item Which is heavier (has greater mass)?
\item How many times heavier is it?  To calculate the answer take the mass of the heavier particle and divide it by the mass of the lighter particle.
\item How many protons would it take to weigh an ounce? Use 
$1 \text{ ounce} \approx 28.3 \text{ grams}$
and, as always, 1 kg = \text{1,000} grams.
\emph{Because $\times$ and $\div$ are at the same level in the order of operations, you should put parentheses around each number in scientific notation before dividing.}
\end{enumerate} % source:wiki.answers.com

\item How many servings are in 
\begin{enumerate}
\item A 2-liter bottle of a soft drink where the serving size is 250 mL?
\item A 750 mL bottle of wine where a serving size is 5 (fluid) ounces?  Use $1 \text{ quart} = 32 \text{ (fluid) ounces}$ and $1 \text{ liter} \approx 1.056 \text{ quarts}$.
\end{enumerate}

\item  Rayka weighs 140 pounds. She would like to approximate how many cells are in her body.  Use the following information: $1 \text{ cell} \approx 1 \times 10^{-15} \text{g}$, $1 \text{ kg} \approx \text{2.2 pounds} $, and, as always, 1 kg = \text{1,000} g.
\begin{enumerate}
\item How many cells are in Rayka's body?  Write your answer in scientific notation.
\item Rewrite your answer in the most appropriate unit:  millions ($10^6$), billions ($10^9$), trillions ($10^{12}$), quadrillions ($10^{15}$), or quintillions ($10^{18}$).
\end{enumerate}

\item  \textbf{Body Mass Index} (BMI) is one indicator of whether a person is a healthy weight.  BMI are between 18.5 and 24.9 are considered ``normal''.  Jared is 6'4" and weighs 200 pounds.  He would like to calculate his BMI from this guide:
$$\text{BMI} = \text{weight in kilograms} \div \text{height in meters} \wedge 2$$
\hfill \begin{footnotesize} Source:  Center for Disease Control and Prevention  \end{footnotesize}
%http://www.cdc.gov/healthyweight/assessing/bmi/adult_bmi/index.html#Interpreted
\begin{enumerate}
\item Check that Jared is around 1.93 meters tall and weighs around 90.91 kilograms.  Use $1 \text{ inch} \approx 2.54 \text{ cm}$ and $1 \text{ kilogram} \approx 2.2 \text{ pounds}$
\item Jared entered the following keystrokes on his calculator: $$90.91 \div 1.93 \wedge 2=$$
and got the answer $$\text{Jared's BMI } = 24.4060243\ldots$$ Is his BMI considered ``normal''?
\item Suppose Jared had rounded off his height to 1.9 meters and his weight to 91 kilograms.  Calculate his BMI by entering the following keystrokes your calculator:  $$91 \div 1.9 \wedge 2=$$
What do you get?  Round your answer to one decimal place.  Is Jared's BMI considered ``normal''?
\item What would you tell Jared?
\end{enumerate}






\end{enumerate}




%\item Here's one for Section 1.4  The artist Jeanne-Claude and Christo's Over the River installation will use 5.9 miles of fabric panels.  WiDTH??  How many yards of x wide is that?  In their 1973 Running fence installation they used approximately \text{200,00} sq meters of nylon.  In 1983 their Biscayne Bay installation in Florida topped that with 603,850 sq m of PINK polypropylene floating on the bay.
%source:  http://en.wikipedia.org/wiki/Christo_and_Jeanne-Claude

%\item Nanotechnology uses objects in the size range of 1 to 100 nanometers.  How small is that?  How many objects 1 nanometer would fit going across a human hair?  Recall a hair is around .00012 meters.







