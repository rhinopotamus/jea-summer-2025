\section{Prelude: Logarithms}

Nearlywed, hellscape, bodycon, and woke.  Just a few new words added to dictionaries in 2023.  People love to create new words and phrases.  These new words spread through social media, music, and word of mouth.

I created a new word ``puzzlaxing'' meaning relaxing by doing puzzles. In one week, maybe 10 people will have heard of it.  After another week, perhaps 100 people.  Then 1,000 the next week, and so on.  How many weeks until 1 million people are have heard of my new word ``puzzlaxing"?

Notice that in 1 week is $10 = 10^1$ people, two weeks is $10^2 = 100$ people, three weeks is $10^3=1000$ people, and so on.  We are looking for a number $W$ where $10^W=\text{1,000,000}$ people.  Aha!  $10^6=\text{1,000,000}$ and so $W=6$ weeks from now.

Suppose we wanted to determine when 5,000 people had heard of ``puzzlaxing''.  That means we want a number $W$ where $10^W = \text{5,000}$.  Now what?  The answer is somewhere between 3 and 4 weeks because $10^3=\text{1,000}$ and $10^4=\text{10,000}$.  That's probably a good enough answer -- between 3 and 4 weeks, but suppose we want the exact moment. 

Let's try guessing.  How about 3.5 weeks? 
$$10^{3.5} =10 \wedge 3.5 = \text{3,162.27...} $$ 
which is much smaller than 5,000.  How about 3.7 weeks? 
$$10^{3.7} =10 \wedge 3.7 = \text{5,011.87..} $$ 
which is slightly bigger than 5,000.  I want to find the answer here so let's try 
$$10^{3.69}=10 \wedge 3.69= \text{4,897.78...}$$ 
or finally 
$$10^{3.699}=10 \wedge 3.699= \text{5,000.34...}$$ 
That's as close as I'm gonna guess. 

Okay, I'm curious. Is there an exact power of 10 that gives 5,000?  Your calculator should have a key that says ``log'' or maybe ``LOG''.  Try typing $$\log(5000)=  3.6989700043...$$
A small note here about parentheses.  Some calculators give the first parenthesis for free when you type log but you have to type the closing parenthesis in yourself.  

Check it out, that's the answer we were looking for 
$$10^{3.6989700043} =10 \wedge 3.6989700043 = \text{4,999.999~999~5853...} \approx \text{5,000}$$
If we had kept more digits it would have been actually 5,000.

What is this log key doing?  First, log is short for logarithm base 10.  There are other bases, but 10 is what we'll focus on in this course.  Try these calculations:

\begin{eqnarray*}
\log (10) & = & 1 \\
\log (100) & = & 2 \\
\log (\text{1,000}) & = & 3 \\
\log (\text{10,000}) & = & 4 \\
\log (\text{100,000}) & = & 5 \\
\log (\text{1,000,000}) & = & 6 \\
\end{eqnarray*}
\vspace{-.5in} %VSPACE

What do you see?  In each case the logarithm is the number of zeros or, equivalently, it's the power of 10.  For example, 10,000 has 4 zeros and $10^4= \text{10,000}$ and $\log(\text{10,000}) = \log(10^4)=4$. In other words, a logarithm is just an exponent. And logarithms help us find the exponent.  Makes sense.

What logs of numbers that aren't just powers of 10? Here are some examples.
\begin{eqnarray*}
\log (25) & = & 1.3979\ldots \\
\log (250) & = & 2.3979\ldots \\
\log (\text{2,500}) & = & 3.3979\ldots \\
\log (\text{25,000}) & = & 4.3979\ldots \\
\end{eqnarray*}
\vspace{-.5in} %VSPACE

To see what's happening we want to involve powers of 10.  Scientific notation will do that for us.  Let's write these numbers in scientific notation and see what we learn.  For example $ \text{25,000} = 2.5 \times 10^4$ and 
$$ \log( \text{25,000})=\log(2.5 \times 10 ^4)=4.3979\ldots \approx 4$$
Before we write down a general rule, let's check more numbers.

$$\log(\text{7,420,000}) = \log(7.42 \times 10^6)=6.870403905\ldots \approx 6$$ 
$$\log (\text{4})=\log(4 \times 10^0)=0.602059991\ldots \approx 0$$ 
$$\log (\text{.00917})= \log(9.17 \times 10^{-3}) = -2.037630664\ldots \approx -3$$
In every case we are rounding down, but it's always the same:

\begin{center}
log(number) $\approx$ power of 10 in the scientific notation for that number.
\end{center}

\noindent \textbf{Do you know \ldots}

\begin{itemize}
\item What a logarithm (base 10) means? 
\item How to evaluate logarithms (base 10) on a calculator? 
\item Which size numbers have a positive log and which have a negative log (base 10)?
\item The connection between logarithms (base 10) and scientific notation.
\item[~] \textbf{If you're not sure, work the rest of exercises and then return to these questions.  Or, ask your instructor or a classmate for help.}
\end{itemize}

\subsection*{Exercises}


%Find power of 10 that equals number when exact + and -
%Find power of 10 that equals number by successive approximations.
%Find power of 10 that equals number using logs
%Explain connection between logs and scientific notation

\begin{enumerate} 
\setcounter{enumi}{4}

\item According to our story, in approximately how many weeks will 30,000 people have heard of ``puzzlelaxing"?
\begin{enumerate}
\item Since 30,000 is between 10,000 and 100,000, what does that tell us about the answer?
\item Guess to try to find the answer, the number $W$ where $10^W = 30,000$.  It's okay to get the answer to one decimal place.
\item Use logs to find an exact answer.
\end{enumerate}

\item The intensity of a sound in decibels is calculated using a logarithm.  For example, a sound 100 times the level humans can hear has an intensity of 
$$10\log (100)= 10 \times \log(100)= 20\text{ decibels}$$
\begin{enumerate}
\item Calculate the intensity, in decibels, of a cat purring which averages about 300 times the level humans can hear using the formula $10\log(\text{300})$.
\item Calculate the intensity, in decibels, of normal conversation which averages about 1 million times the level humans can hear using the formula $10\log(\text{1,000,000})$.
\item Many young adults are at high risk of hearing loss because they crank the volume of music they're listening to, often to 2 trillion times the level humans can hear.  Calculate that intensity in decimals using the formula $10\log(2 \ast10^{12})$. Prolonged sound above 70 decibels may damage your hearing and any loud noise above 120 decibels can cause immediate harm to hearing.
\end{enumerate}

\item In 2021, Arrietty charged \$5,000 on her credit card to help pay tuition.  Her card charges interest at the rate of 20.7\% APR.   We are going to ignore any minimum payments or fees.
\begin{enumerate}
\item Arrietty was hoping to pay the debt back quickly, but in 2023 she had not paid any of the debt.  Calculate the amount due on that original charge using the formula $5000 \ast 1.207^2$.
\item If she continues to leave the debt unpaid, when will her debt pass \$10,000? As we'll see later, the answer is $\displaystyle \frac{\log(2)}{\log(1.207)}$ years after 2020.  Don't forget to the close the parentheses.  
\end{enumerate}

\item Darcy likes to use temporary hair color in wild colors.  Good thing it washes out.  Her best guess is that 8\% of the color washes out each time she shampoos her hair.   That means $100\% - 8\% = 92\%$ of the color remains after each shampoo.

\hfill \emph{Story also appears in 3.4 \#9}
\begin{enumerate}
\item What percentage of the color will be remain after Darcy washes her hair three times?  Calculate the percentage using the formula $100 \ast .92^3$.  
\item After how many shampoos will half of the color be gone?  As we'll see later, the answer is $\displaystyle \frac{\log(.5)}{\log(.92)}$.  Don't forget to the close the parentheses. 
\end{enumerate}
 

\end{enumerate}

\bigskip

\noindent \textbf{When you're done \ldots}

\begin{itemize}
\item Don't forget to check your answers with those in the back of the textbook. 
\item Not sure if your answers are close enough? Compare with a classmate or ask the instructor.  
\item Getting the wrong answers or stuck on a problem?  Re-read the section and try the problem again.   If you're still stuck, work with a classmate or go to your instructor's office hours.
\item It's normal to find some parts of some problems difficult, but if all the problems are giving you grief, be sure to talk with your instructor or advisor about it.  They might be able to suggest strategies or support services that can help you succeed.
\item Make a list of key ideas or processes to remember from the section.  The ``Do you know?'' questions can be a good starting point.
\end{itemize}

