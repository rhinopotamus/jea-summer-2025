\section{Prelude: Powers and Roots}

Noah is very proud of his sobriety.  He credits some of his success to handicrafts, like beading.  He finds the steady, repetitive work of stringing beads one by one (or, actually, Noah prefers two by two) to be a calming practice and he enjoys flexing his artistic creativity.  

Noah's mom also enjoys handicrafts and would like to build a small glass box to hold some of Noah's beads. To keep things simple she has decided that the box will be a cube, meaning it will have the same length, width, and height.  Plus, the cube has long been the symbol of regeneration and stability but also of limitations and boundaries -- a fitting recovery gift.

She could make a small box that's $2 \times 2 \times 2$.  That's pronounced ``2 by 2 by 2'' and means 2 inches long, 2 inches wide, and 2 inches tall.  Such a small box would hold $2 \times 2 \times 2 =  8$ cubic inches of beads, which is around 2/3rd of a cup (not much).

What if Noah's mom made a box that was $5 \times 5 \times 5$ instead?  That box would hold $5 \times 5 \times 5 = 125$ cubic inches of beads, which is just over 2 liters of beads. Similarly a $10 \times 10 \times 10$ box would hold $10 \times 10 \times 10 = 1000$ cubic inches of beads, or about 4 gallons of beads. In case you're curious about the units here, there's about 14.4 cubic inches per cup,  61 cubic inches per liter, and 231 cubic inches per gallon.  

When we multiply a number by itself, like $$\underbrace{2 \times 2 \times 2}_{3 \text{ times}}$$ we say that we are raising 2 \textbf{to the power of} 3.  The number 3, which counts how many times we multiply 2 by itself, is called the \textbf{power} or \textbf{exponent}. 

On a calculator we can use the $\land$ key.  Try for yourself:
$$2 \land 3 = \underbrace{2 \times 2 \times 2}_{3 \text{ times}} = 8$$
$$5 \land 3 = \underbrace{5 \times 5 \times 5}_{3 \text{ times}} = 125$$
$$10 \land 3 = \underbrace{10 \times 10 \times 10}_{3 \text{ times}} = 1000$$
No key marked $\land$? Look for a key marked $x^y$ or $y^x$ instead. Can't find that either?  Ask a classmate or your instructor for help.

We can even do higher powers for practice:

$$3 \land 4 = \underbrace{3 \times 3 \times 3 \times 3}_{4 \text{ times}} = 81$$
$$2 \land 10 = \underbrace{2 \times 2 \times 2 \times 2 \times 2 \times 2 \times 2 \times 2 \times 2 \times 2}_{10 \text{ times}}= 1024$$
Speaking of which,  Noah thought it was fitting it is that his mom was using mathematical higher powers to build the glass box since learning to trust in a spiritual higher power was so important to his recovery.

Noah would like the box to hold 1 gallon of beads which is about 231 cubic inches.  What size cube box should his mom make?  She wants a number, let's call it $n$, so that an $n \times n \times n$ box will hold 231 cubic inches.  That means she needs $$n \times n \times n = 231$$ or, equivalently, $$n \land 3 = 231$$

Noah's mom can try to guess what number $n$ should be.  She wants a box that's bigger than $5  \times 5 \times 5$ but smaller than $10 \times 10 \times 10$.  Her first guess is $n=7$ inches, meaning the box will be $7 \times 7 \times 7$.  That box would hold $$7 \times 7 \times 7 = 7 \land 3 = 343 \text{ cubic inches}$$ which is too big.  How about $n = 6.5$ inches so the box would hold $$6.5 \times 6.5 \times 6.5 = 6.5 \land 3  \approx 274 \text{ cubic inches}$$ which is still too big. After a few more guesses she tries $n = 6.2$ so the box would hold $$6.2 \times 6.2 \times 6.2 = 6.2 \land 3 \approx 238 \text{ cubic inches}$$ Aha!  She will make a box that's approximately $6.2 \times 6.2 \times 6.2$.

That was a lot of guessing.  Turns out there's a name for the answer.  We write $$n = \sqrt[3]231$$ and say that the dimension we are looking for is the 3rd \textbf{root} of 231.  

We can use our calculator to find roots.  For example, if you have a key that says $\sqrt[x]{y}$ then you can enter
$$3 \sqrt[x]{y} ~231 =  6.1357...$$
Different calculators label the roots key differently.  For example, you may have to use one of the 2nd, Shift, or Inv key with the $\land$, $y^x$, or $x^y$ key.  On a graphing calculator, you may have to enter MATH mode.  

For practice, try evaluating the following roots on your calculator. Feel free to ask a classmate or your instructor for help if you can't figure it out.

$$\sqrt[3]{125} = 3 \sqrt[x]{y}~ 125= 5$$
$$\sqrt[3]{1000} = 3 \sqrt[x]{y}~ 1000 = 10$$
$$\sqrt[4]{81} = 4 \sqrt[x]{y}~ 81 = 3$$
$$\sqrt[10]{1024} = 10 \sqrt[x]{y}~ {1024} = 2$$

By the way, $n \land 3$ is sometimes called $n$ \textbf{cubed} and $\sqrt[3]{~}$ is referred to as the \textbf{cube root}.  That terminology comes from the fact that the $n \times n \times n$ cube has volume $n \times n \times n = n \land 3$, as in our example.  Also note that the units on volume which are inches $\times$ inches $\times$ inches are called \textbf{cubic inches}.

Similarly, $n \land 2$ is called $n$ \textbf{squared} and $\sqrt{~}$ (which is shorthand for $\sqrt[2]{~}$) is referred to as the \textbf{square root}.  Some calculators have a separate key for the square root.  That terminology comes from the fact that an $n \times n$ square has area $n \times n = n \land 2$.  And a square that was $n \times n$ in inches would have area measured in inches $\times$ inches, which are called \textbf{square inches}.
%See Sections 2.3, 2.4, and 3.3
%
%For problem 4 in workbook (and one in homework), perhaps use a growth factor formula example???

 \begin{center}
\line(1,0){300} %\line(1,0){250}
\end{center}

\section*{Homework}

\noindent \textbf{Start by doing Practice exercises \#1-4 in the workbook.}

\bigskip

\noindent \textbf{Do you know \ldots}

\begin{itemize}
\item What the square, cube, or higher power of a number means? %\vfill
\item How to calculate powers of a number using a calculator? %\vfill
\item What the square root, cube roots, or higher root of a number means? %\vfill
\item How to calculate roots of a number using a calculator? %\vfill
 \item[~] \textbf{If you're not sure, work the rest of exercises and then return to these questions.  Or, ask your instructor or a classmate for help.} 
\end{itemize}

\subsection*{Exercises}

\begin{enumerate} 
\setcounter{enumi}{4}

\item Remember Noah's mom from our story?  She was making a glass box that is $n \times n \times n$ where the length $n$ is measured in inches.
\begin{enumerate}
\item How many cubic inches of beads would the box hold if it's $8 \times 8 \times 8$? What if the box were $7.5 \times 7.5 \times 7.5$ instead?  You can first find the answers by multiplying on your calculator, but then challenge yourself to use the power key.
\item Use guessing to approximate the dimension of a box that would hold 400 cubic inches of beads. Just guess to the nearest one decimal place. Again, practice using the power key.
\item Use cube roots and your calculator to figure out the dimension of a box that would hold 400 cubic inches of beads, to the nearest one decimal place.  Does your answer agree with part (b)?  (It should.)
\end{enumerate}

\item Creeping Charlie is a low-growing weed that spreads quickly, doubling the area it covers each year.  If there are 7 square feet of Creeping Charlie in my lawn now, how much of my lawn will be covered by Creeping Charlie in 1 year? In 2 years? In 10 years?  (Try to answer that last one using powers)

\item Saboor is working on a needlepoint that will be a 1 foot by 1 foot square.  The mesh grid comes in different sizes.   For example, a 13-count mesh has 13 holes per inch which is $13 \times 12 =  156$ holes per foot. If she uses a 13-count mesh, then the piece will have $156 \times 156 = 156 \land 2 = 24,336$ holes.
\begin{enumerate}
\item How many holes does a 1 foot by 1 foot square mesh grid have if she uses a 10-count mesh instead which has 10 holes per inch? \vfill
\item What count mesh will have 10,000 holes?  First find the holes per foot by calculating $\sqrt{10000}$.  Then divide by 12 to find the holes per inch. \vfill
\end{enumerate}

\item A set of sterling silverware was valued at \$800 in 1920, and the value increased by around 3\% per year thereafter. \emph{Story appears in Section 5.1}
\begin{enumerate}
\item What was the value in 1921?  Remember, to find 3\% of a number we multiply by $3 \div 100 = .03$. %\vfill
\item What was the value in 1922?  Don't forget to use your answer to part (a) to calculate the new increase in value. %\vfill
\item It turns out there's a quicker way to find the answers to parts (a) and (b).  Calculate $800 \times 1.03=$ and $800 \times 1.03 \times 1.03 =$.   Note:  we will discuss this shortcut in greater detail in Section 2.2  %% Section on First look at exponentials
%\vfill
\item In general, we can find the value of the silverware after $Y$ years by calculating $$800 \times \underbrace{1.03 \times 1.03 \times \cdots \times 1.03}_{Y \text{ times}} = 800 \times 1.03 \land Y =$$ 
Use this powers method to find the value of the sterling in 1957.  Hint: $Y = 1957-1920 = 37$ years. %\vfill
\item Use this powers method to find the value of the sterling in 1990. %\vfill
\item Use this powers method to find the value of the sterling in 2023. %\vfill
\end{enumerate}

\end{enumerate}

\bigskip

\noindent \textbf{When you're done \ldots}

\begin{itemize}
\item Don't forget to check your answers with those in the back of the textbook. 
\item Not sure if your answers are close enough? Compare with a classmate or ask the instructor.  
\item Getting the wrong answers or stuck on a problem?  Re-read the section and try the problem again.   If you're still stuck, work with a classmate or go to your instructor's office hours.
\item It's normal to find some parts of some problems difficult, but if all the problems are giving you grief, be sure to talk with your instructor or advisor about it.  They might be able to suggest strategies or support services that can help you succeed.
\item Make a list of key ideas or processes to remember from the section.  The ``Do you know?'' questions can be a good starting point.
\end{itemize}

