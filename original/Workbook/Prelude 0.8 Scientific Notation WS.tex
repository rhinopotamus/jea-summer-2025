
\section{Prelude: Scientific Notation} 

\subsection*{Practice exercises}

\begin{enumerate}

\item In each story, write out the highlighted numbers (with all the zeros).
\begin{enumerate}
\item Melvin was looking populations based on the 2020 Census and saw the population of Saint Paul, MN listed as $\mathbf{3.10942 \times 10^{5}}$ \textbf{people}. Hint:  you can check the answer to this part by evaluating on your calculator. \vfill
\item The gross domestic product (GDP) measures the market value of all final goods and services produced by an economy. The United States GDP is approximately $\mathbf{\$2.332 \times 10^{13}}$.
\hfill \emph{Story also appears in 1.5 \#1} \vfill
\item The Earth weighs approximately $\mathbf{5.972 \times 10^{24}}$ \textbf{kilograms}.

\hfill \emph{Story also appears in 1.5 \#3} \vfill
\end{enumerate}

\item In each story, write out the highlighted numbers (with all the zeros).
\begin{enumerate}
\item Alpaca have very fine hairs (which can be spun into yarn to make very soft sweaters).  The width of an alpaca hair is around $\mathbf{2.5 \times 10^{-7}}$ \textbf{meters}. Hint:  you can check the answer to this part by evaluating on your calculator. \vfill
\item A dust particle weighs approximately $\mathbf{7.53 \times 10^{-10}}$ \textbf{grams}.

\hfill \emph{Story also appears in 1.5 \#2} \vfill
\item  A proton (part of an atom) has mass of about $\mathbf{1.67262 \times 10^{-27}}$ \textbf{kilograms}.

\hfill \emph{Story also appears in 1.5 \#7} \vfill
\end{enumerate}

\newpage %%%%%%

\item In each story, evaluate the number and report your answer in scientific notation.
\begin{enumerate}
\item Bunnies, bunnies, everywhere.  In 2007 there were 1800 and that number was predicted to increase 13\% each year.  I was trying to predict the number of rabbits in 2023 (after 16 years) but I accidentally typed in 166 years by mistake $$1800 \ast 1.13 \wedge 166=$$
Report the answer I got in scientific notation.  (Yes, this is a gigantic number.  The exponential model I used doesn't actually make sense for that many years.)   

\hfill \emph{Story also appears in 2.1 \#2 and 5.1 \#3}\vfill
\item A signal is sent down a fiber optic cable.  Its strength decreases by 2\% each mile it travels. We can calculate the signal strength after 1000 miles by evaluating $$.98 \wedge 1000=$$
Report the answer you get in scientific notation. (Yes, this is a teeny number.  In reality there would be signal booster installed along the route.)

\hfill \emph{Story also appears in 5.2 \#1} \vfill
\end{enumerate}

\item In each story, write out the highlighted number (with all the 0s).  Note that \textbf{million} is short for $\times 10^6$, \textbf{billion} is short for $\times 10^9$, and \textbf{trillion} is short for $\times 10^{12}$.
\begin{enumerate}
\item There are approximately \textbf{1.084 million quarters} in circulation in the United States. \hfill \emph{Story also appears in 0.1 \#4} \vfill
\item The population of the world is approximately  \textbf{8.1 billion people}. 

\hfill \emph{Story also appears in 0.3 \#1} \vfill

\item One way that the United States government borrow money is Treasury bonds (T-bonds).  %For example, you might pay \$600 for a \$1000 T-bond that matures in 10 years.  That means you paid \$600 now and in 10 years the government will pay you \$1000.  
There are approximately 
 \textbf{\$24 trillion} worth of T-bonds currently.  \vfill
\end{enumerate}

\end{enumerate}


\newpage %%%%%


\noindent \textbf{When you're done \ldots}

\begin{itemize}
\item [$\Box$] Check your solutions.  Still confused?  Work with a classmate, instructor, or tutor.
\item [$\Box$] Try the \textbf{Do you know} questions.  Not sure?  Read the textbook and try again.
\item [$\Box$] Make a list of key ideas and process to remember under \textbf{Don't forget!}
\item [$\Box$] Do the textbook exercises and check your answers. Not sure if you are close enough? Compare answers with a classmate or ask your instructor or tutor.  
\item [$\Box$] Getting the wrong answers or stuck?  Re-read the section and try again.   If you are still stuck, work with a classmate or go to your instructor's office hours or tutor hours.
\item [$\Box$] It is normal to find some parts of exercises difficult, but if most of them are a struggle, meet with your instructor or advisor about possible strategies or support services.
\end{itemize}





\bigskip

\noindent \textbf{Do you know \ldots}

\begin{enumerate} [(a)]
\item What million, billion, and trillion mean? \vfill
\item Why scientific notation is used? \vfill 
\item The standard format for scientific notation? \vfill 
\item That a positive exponent corresponds to a big number and a negative exponent corresponds to a tiny number? \vfill
\item How to convert from scientific notation to decimal? \vfill
\item How your calculator reports numbers in scientific notation, and what (might be) different when you're reporting that number? \vfill
\end{enumerate}

\noindent \textbf{Don't forget!}
\vfill \vfill \vfill




