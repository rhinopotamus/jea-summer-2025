
\section{Prelude: Fractions} 

\subsection*{Practice exercises}
%%\emph{Story also appears in 1.2 \#4, 2.1 \#4, and 4.2 \#2}

\begin{enumerate}

\item There are 2,624 students at a local university.  
\begin{enumerate}
\item Of those students, 673 of those students placed into this algebra class.  What fraction of students placed into algebra? 
\vfill
\item The Dean said that approximately 1 in 4 students, or $\frac{1}{4}$ of all students, placed into algebra.  Is that correct?  Check by determining if your answer to part (a) $\approx \frac{1}{4}$ by comparing decimal approximations. 
\vfill
\end{enumerate}

\item Gas mileage is usually rounded down to the nearest one decimal place. Gas mileage is measured in miles per gallon (mpg).
\begin{enumerate}
\item Xu does gig work delivering take-out food from local restaurants.  He started the week with a full tank of gas and drove 319 miles.  When he went to fill the tank, he needed 11.3 gallons. What was Xus gas mileage?
\vfill
\item Margaret and Cathy are on a cross-country trip.  They've driven from Minnesota to Maine (approximately 1,430 miles).  They have bought gas a few times along the way:  12.7 gallons, then 14.0 gallons, then 13.1 gallons, and then 12.4 gallons.  What was Margaret and Cathy's gas mileage? 
\vfill
\item How could you do the calculation in part (b) one line on your calculator by using parentheses?
\vfill
\end{enumerate}

\newpage %%%%%

\item In January 2015, Graham had 47 albums in his vinyl collection.  By September 2023 (that's 8 years, 9 months later), he had 783 albums.  Approximately how many albums per month did Graham buy?
\begin{enumerate}
\item Figure out the answer step by step.
\vfill
\vfill
\item Now try to combine all of your calculations into one line on your calculator.  Hint:  write as a fraction first.
\vfill
\end{enumerate}

\item It took Mariam 3 hours to complete the reading for her Religion class. The reading was 102 pages long.  
\begin{enumerate}
\item How fast did she read measured in pages per hour? Write the answer as a fraction and as a decimal. 
\vfill
\item Reading speed is often measured in words per minute.  Assuming there are approximately 500 words per page, calculate Mariam's reading speed step by step.
\vfill
\vfill
\item How could you do the calculation in part (b) one line on your calculator by using parentheses? Hint: the ``hours'' cancel! 
\vfill 
\end{enumerate}

\end{enumerate}

\newpage %%%%%


\noindent \textbf{When you're done \ldots}

\begin{itemize}
\item [$\Box$] Check your solutions.  Still confused?  Work with a classmate, instructor, or tutor.
\item [$\Box$] Try the \textbf{Do you know} questions.  Not sure?  Read the textbook and try again.
\item [$\Box$] Make a list of key ideas and process to remember under \textbf{Don't forget!}
\item [$\Box$] Do the textbook exercises and check your answers. Not sure if you are close enough? Compare answers with a classmate or ask your instructor or tutor.  
\item [$\Box$] Getting the wrong answers or stuck?  Re-read the section and try again.   If you are still stuck, work with a classmate or go to your instructor's office hours or tutor hours.
\item [$\Box$] It is normal to find some parts of exercises difficult, but if most of them are a struggle, meet with your instructor or advisor about possible strategies or support services.
\end{itemize}





\bigskip

\noindent \textbf{Do you know \ldots}

\begin{enumerate}[(a)]
\item How we represent a part of a whole as a fraction? \vfill
\item How to multiply fractions?  \vfill
\item What``canceling'' a factor means? \vfill
\item How fractions are related to division? \vfill
\item How to calculate the decimal approximation of a fraction? \vfill
\item How to compare two fractions using their decimal approximations? \vfill
\item How the units of a fraction are determined?  \vfill
\item When we need to use parentheses around the top (numerator) and bottom (denominator) to evaluate a fraction? \vfill
\end{enumerate}

\noindent \textbf{Don't forget!}
\vfill \vfill \vfill




