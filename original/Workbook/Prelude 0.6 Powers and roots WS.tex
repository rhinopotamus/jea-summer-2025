
\section{Prelude: Powers and roots} 

\subsection*{Practice exercises}

\begin{enumerate}

\item Jody is using small wooden balls to make noses for her knitted gnomes.  She figured out that she can calculate the weight of each ball (in ounces) as $.2 \times B \land 3$ where $B$ is the diameter of the ball (in inches).\begin{enumerate}
\item 
What does a 2.5 inch diameter wooden ball weigh? \vfill
\item Jody is considering building a giant gnome for her office.  The nose will be a wooden ball weighing 1 pound.  She calculates that the diameter of the ball will be $\sqrt[3]{80}$.  How big is that? \vfill
\end{enumerate}

\item The size of a round pizza is described by its diameter.  It turns out that we can calculate how many people are served by a pizza of diameter $D$ inches as $.015625 \times D \land 2=$.  For example, a 16-inch diameter pizza serves $.015625 \times 16 \land 2 = 4$ people.  (The mysterious number .015625 comes from a little geometry and pizza science.)

\hfill \emph{Story also appears in 2.4 \#1 and 3.3 \#1.}
\begin{enumerate}
\item How many people would be served by a 12-inch pizza? \vfill
\item A personal pizza is designed to serve one person.  It turns out the diameter of a personal pizza is $\sqrt{64}$.  Calculate the diameter of a personal pizza using the square root key (or just the root key) on your calculator. \vfill
\item An extra large pizza serves 6 people.  It turns out the diameter of an extra large pizza is $\sqrt{384}$.  Calculate the diameter of a personal pizza using the square root key (or just the root key) on your calculator. \vfill
\end{enumerate}

\newpage %%%%%

\item A signal sent down a fiber optic cable decreases by 2\% per mile.  That means after $M$ miles, its strength is $\underbrace{.98 \times .98 \times \cdots \times .98}_{M \text{ times}} = .98 \land M$.  What is the signal strength after 10 miles? After 20 miles? Note: your answers should be decimal numbers less than 1. \vfill

\item Otis invested \$500,000 and estimates his investment will double in value every 10 years.  
\begin{enumerate}
\item Calculate the value of Otis's investment after 10, 20, 30, and 40 years.   \vfill
\item If Kricia invested \$230,000 instead, what would her investment be worth after 40 years?  Try to use a power to help answer the question. Hint:  how many times will the value of her investment double? \vfill
\end{enumerate}

\end{enumerate}

\newpage %%%%%


\noindent \textbf{When you're done \ldots}

\begin{itemize}
\item [$\Box$] Check your solutions.  Still confused?  Work with a classmate, instructor, or tutor.
\item [$\Box$] Try the \textbf{Do you know} questions.  Not sure?  Read the textbook and try again.
\item [$\Box$] Make a list of key ideas and process to remember under \textbf{Don't forget!}
\item [$\Box$] Do the textbook exercises and check your answers. Not sure if you are close enough? Compare answers with a classmate or ask your instructor or tutor.  
\item [$\Box$] Getting the wrong answers or stuck?  Re-read the section and try again.   If you are still stuck, work with a classmate or go to your instructor's office hours or tutor hours.
\item [$\Box$] It is normal to find some parts of exercises difficult, but if most of them are a struggle, meet with your instructor or advisor about possible strategies or support services.
\end{itemize}





\bigskip

\noindent \textbf{Do you know \ldots}

\begin{enumerate} [(a)]
\item What the square, cube, or higher power of a number means? \vfill
\item How to calculate powers of a number using a calculator? \vfill
\item What the square root, cube roots, or higher root of a number means? \vfill
\item How to calculate roots of a number using a calculator? \vfill
\end{enumerate}

\noindent \textbf{Don't forget!}
\vfill \vfill \vfill




