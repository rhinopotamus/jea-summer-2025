
\section{Prelude: Approximation and Rounding} 

\subsection*{Practice exercises}

\begin{enumerate}
\item Round each number up, down, or off to the precision indicated.

\emph{This problem also appears in Section 1.1 \#3.}

\begin{enumerate}
\item My calculations show I need a cross brace around 9.388 feet long. I want the board to be long enough, so round up to the nearest foot.
\vfill
\item Gas mileage is usually rounded down to the nearest one decimal place.  What is the gas mileage for a car measured as getting 42.812 miles per gallon?  What about a car getting 23.09 miles per gallon?
\vfill
\item The population estimate was 4.2 million people, but revised estimates suggest 4,908,229 people.  Report the revised estimate rounded appropriately. What if a different estimate was 4,890,225?  Would that change your answer?
\vfill
\end{enumerate}

\newpage 

\item \begin{enumerate}
\item Callista needs \$117 cash for a mani-pedi at the local salon.  The ATM allows her to withdraw multiples of \$20.  How much money should she withdraw and how many \$20 bills is that? Did you round up, down, or off? 
\vfill
\item Bahari is buying some 8-packs of sparkling water for today's community hour. He expects up to 23 people to be there.  He calculates that he will need $23 \div 8 = 2.875$ 8-packs.  How many 8-packs should he bring?  Did you round up, down, or off? 
\vfill
\item Tzuf has \$20 to buy apples for the new year's celebration.  A bag of apples costs \$3.49.  Tsuf calculates that they can afford $20 \div 3.49 = 5.7306...$ bags. How many bags can they buy? Did you round up, down, or off?  
\vfill
\item Eiji read that life expectancy in the United States is 77.28 years whereas in Japan it is 84.62 years.  How might he describe these life expectancies in (whole) years? Did you round up, down, or off? 
\vfill
\end{enumerate}

\newpage

\item Round off the \underline{calculated number(s)} to give an answer that is reasonable and no more precise than the information given.

\begin{enumerate}
\item The snow removal budget for the city is currently at \$8.3 million but the city council is requesting a reduction of \$1.15 million per year.  We calculate that after three years of cuts, the snow removal budget will be \underline{\$4.8079...} million. 
\vfill
\item A cup of cooked red lentils has around 190 calories and 6.4 grams of dietary fiber, while a cup of cooked chickpeas has around 172 calories and 12.0 grams of dietary fiber.  We calculate that lentils provide \underline{0.03368421...} grams per calorie whereas chickpeas provide \underline{0.06976744...} grams per calorie. 
\vfill
\item \begin{quote} Hibbing [Minnesota] is the former boyhood home of Bob Dylan, basketball great Kevin McHale and the location of the Hull-Rust-Mahoning Open Pit Iron Mine, which has the largest iron-ore pit in the world. Hibbing is also the birthplace of [baseball star] Roger Maris.

\hfill \begin{tiny} (source: http://hibbing.areaconnect.com/)\end{tiny}\end{quote}  

In 2000 the population of Hibbing, Minnesota was reported at just over 17,000 residents. Based on a projected 0.4\% decrease per year, the 2010 population was calculated to be \underline{16,332.110...} people. 
%(In case you are curious, the actual 2010 census count was 17,071 people and the actual 2011 census count was 16,361.  Your reported answer should agree.)
\vfill
\end{enumerate}

\newpage

\item It is easiest to compare the size of decimal numbers when they are written the same precision.  For example, \$1.7 million is more money than \$1.34 million because when we write both numbers to two decimal places we see $$1.7 = 1.70 > 1.34$$  The symbol $>$ means ``greater than;'' it points to the smaller number.  Alternatively, when we expand both numbers we see $$1,700,000  > 1,340,000$$  

In each story, write all of the decimal numbers given to the same precision and list the numbers from largest to smallest using $>$ signs.

\begin{enumerate}
\item Dawn tested a water sample from her apartment and found 21.19 ppm of sulfate.  She volunteers at a local soup kitchen where the water sample tested at 21.3 ppm.  (The abbreviation \textbf{ppm} stands for ``parts per million.  Not to worry -- sulfate levels below 250 are considered safe for human consumption.)
\vfill
\item There are approximately 1.084 million quarters in circulation in the United States, compared to 1.786 million dimes, 1.6 million \$5 bills,  and 1.42 million \$10 bills.
\vfill

\end{enumerate}

\end{enumerate} % PAUSE

\newpage


\noindent \textbf{When you're done \ldots}

\begin{itemize}
\item [$\Box$] Check your solutions.  Still confused?  Work with a classmate, instructor, or tutor.
\item [$\Box$] Try the \textbf{Do you know} questions.  Not sure?  Read the textbook and try again.
\item [$\Box$] Make a list of key ideas and process to remember under \textbf{Don't forget!}
\item [$\Box$] Do the textbook exercises and check your answers. Not sure if you are close enough? Compare answers with a classmate or ask your instructor or tutor.  
\item [$\Box$] Getting the wrong answers or stuck?  Re-read the section and try again.   If you are still stuck, work with a classmate or go to your instructor's office hours or tutor hours.
\item [$\Box$] It is normal to find some parts of exercises difficult, but if most of them are a struggle, meet with your instructor or advisor about possible strategies or support services.
\end{itemize}





\bigskip

\noindent \textbf{Do you know \ldots}

\begin{enumerate}[(a)]
\item What the symbol for ``approximately equal to'' is? \vfill
\item Why an approximate answer is often as good as we can get? \vfill
\item What the term ``precisely'' refers to? \vfill
\item What the saying ``I'd rather be approximately right than precisely wrong'' means? \vfill
\item What the difference is between rounding off, rounding up, and rounding down? \vfill
\item When to round your answer, and when to round your answer up or down (instead of off)? \vfill
\item How to round a decimal to the nearest whole number? \vfill  To one decimal place? \vfill  To two decimal places? \vfill
\item How precisely to round an answer? \vfill
\item How to compare sizes of decimal numbers? \vfill
\item What the symbol for ``greater than'' is? \vfill
\end{enumerate}

\noindent \textbf{Don't forget!}
\vfill \vfill \vfill

%%DO I want to add the symbols for less than, greater than or equal to, less than or equal to????



