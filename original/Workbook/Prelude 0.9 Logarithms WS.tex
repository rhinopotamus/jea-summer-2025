
\section{Prelude: Logarithms} 

\subsection*{Practice exercises}

\begin{enumerate}

\item The number of bacteria in a dish increases 10-fold each day.   Note: 10-fold means $\times 10$.  Suppose we had 1 microliter of bacteria at the start of the first day.  That means after $D$ days there will be $10^D$ microliters of bacteria.
\begin{enumerate}
\item How many bacteria (in microliters) will there be after 1 day?  After 2 days? After 3 days?
\vfill
\item In how many days will the bacteria have reached 1 liter, which is million microliters?   \vfill
\item How can we use logs to find the answer?\vfill
\end{enumerate}

\item The problem continues \ldots
\begin{enumerate}
\item How many days(from the start) does it take to reach the 25 millilter capacity of the petri dish, which is 25,000 microliters?  Guess and check to find the answer to 1 decimal place. \vfill
\item How can we use logs to find the answer? \vfill
\item Convert your answer to days \& hours format (meaning d days and h hours). \vfill
\end{enumerate}

%\item The world's population is increasing 10-fold every 250 years.  In 2000 the world's population was approximately 6.1 billion.
%\begin{enumerate}
%\item According to this model, what will the world's population be in the year 2250? That's 250 years later. \vfill
%\item What will the world's population be in the year 2500? That's another 250 years later. \vfill
%\item When will the world's population pass 1 trillion people?  That's 1,000 billion so we're looking for the number $N$ where $6.1 \ast 10^N =\text{1,000}$.  Note: in part (a) we had $N=1$ and in part (b) we had $N=2$. Guess to find the value of $N$ correct to one decimal place. \vfill \vfill
%\item Use logs to find the value of $N$.As we'll see later, the answer is $\displaystyle \log(\frac{1000}{6.1})$ \vfill
%\item What year does that value of $N$ represent?  Recall that $N$ counts the multiple of 250 years, so the year should be $2000 + 250N$. \vfill
%\end{enumerate}

\newpage %%%%%%

\item The equation $pH = -\log(H^+)$ tells us the pH of a substance (on a scale from 0 to 14) based on its molar hydrogen ion concentration $H^+$.  Don't let the notation here scare you: $pH$ is a single quantity and $H^+$ has nothing to do with exponents or adding. 

For example, lemon juice has $H^+= .0025$ and so the pH of lemon juice is 
$$-\log(.0025) = \text{(-)} \log(.0025) = 2.6020599913 \approx 2.6$$

\begin{enumerate}
\item Coca-cola has $H^+= .000~398$.  Find the pH of orange juice. Note: the funny spaces are to help you read the number.\vfill
\item Hair shampoo has $H^+= .000~003~162$  Find the pH of hair shampoo. \vfill
\item Household bleach has $H^+=1.1 \times 10^{-13}$  Find the pH of bleach. \vfill
\item Materials with pH values between 0-5 are \textbf{acidic}, between 9-14 are \textbf{basic}, and between 5-7 are \textbf{neutral}.  Which of the above materials are acid, basic, and neutral?  \vfill
 \end{enumerate}
 

\item In Minneapolis, apartment rent is expected to increase by 16\% next year.  \

\hfill \emph{Story also appears in 0.3 \#2}
\begin{enumerate}
\item Astra lives in a 1-bedroom apartment where they pay \$825 per month in rent.  If their rent increased by 16\% in how many years would their rent be doubled to \$1,650.  As we'll see later, the answer is $\displaystyle \frac{\log(2)}{\log(1.16)}$.  Don't forget to the close the parentheses.  
\vfill
\item Lucky for Astra, their building is subject to rent stabilization laws and so their rent cannot increase by more than 3\%. In how many years would their rent double under this cap?  The answer is $\displaystyle \frac{\log(2)}{\log(1.03)}$.\vfill
\end{enumerate}



\end{enumerate}


\newpage %%%%%


\noindent \textbf{When you're done \ldots}

\begin{itemize}
\item [$\Box$] Check your solutions.  Still confused?  Work with a classmate, instructor, or tutor.
\item [$\Box$] Try the \textbf{Do you know} questions.  Not sure?  Read the textbook and try again.
\item [$\Box$] Make a list of key ideas and process to remember under \textbf{Don't forget!}
\item [$\Box$] Do the textbook exercises and check your answers. Not sure if you are close enough? Compare answers with a classmate or ask your instructor or tutor.  
\item [$\Box$] Getting the wrong answers or stuck?  Re-read the section and try again.   If you are still stuck, work with a classmate or go to your instructor's office hours or tutor hours.
\item [$\Box$] It is normal to find some parts of exercises difficult, but if most of them are a struggle, meet with your instructor or advisor about possible strategies or support services.
\end{itemize}





\bigskip

\noindent \textbf{Do you know \ldots}

\begin{enumerate} [(a)]
\item What a logarithm (base 10) means? \vfill
\item How to evaluate logarithms (base 10) on a calculator? \vfill
\item Which size numbers have a positive log and which have a negative log (base 10)? \vfill
\item The connection between logarithms (base 10) and scientific notation. \vfill
\end{enumerate}

\noindent \textbf{Don't forget!}
\vfill \vfill \vfill




