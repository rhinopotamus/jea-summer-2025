\documentclass[12pt]{article}
\pagestyle{empty}
\setlength{\parskip}{0in}
\setlength{\textwidth}{6.8in}
\setlength{\topmargin}{-.5in}
\setlength{\textheight}{9.3in}
\setlength{\parindent}{0in}
\setlength{\oddsidemargin}{-.7cm}
\setlength{\evensidemargin}{-.7cm}

\usepackage{amsmath}
\usepackage{amsthm}
\usepackage{amstext}

\usepackage{graphicx}

\begin{document}

\emph{Try taking these practice exams under testing conditions:  no book, no notes, no classmate's help, no electronics (computer, cell phone, television). Give yourself one hour to work and wait until you have tried your best on all of the problems before checking any answers.}
\bigskip

\subsection*{Practice exam 3-- version a}
\bigskip
 \emph{Relax.  You have done problems like these before.  Even if these problems look a bit different, just do what you can.  If you're not sure of something, please ask! You may use your calculator.  Please show all of your work and write down as many steps as you can.  Don't spend too much time on any one problem.  Do well.  And remember, ask me if you're not sure about something.}
 
\bigskip
 
\emph{A few formulas from our book:}

\begin{center}

FORMULAS PRINTED ON EXAM GO HERE

\end{center}

\hspace{-.25in} \hrulefill

\begin{enumerate}
 \item  (3.1, 3.2)  Best we can tell, the floor of our front porch was 7'2" above ground when the house was built.  It has been slowly sinking ever since.  The contractor estimated that it's dropped .45 inches per year.
\begin{enumerate}
\item Make a table showing the height of the front porch when the house was built, when it was 20 years old, and when it was 50 years old.  Don't forget to convert 7'2" to inches first.
\item Name the variables and write an equation illustrating the dependence.
\item The floor of our front porch is currently 48 inches above ground.   Estimate how old our house is using successive approximations.
\item Now, solve your equation to figure out how old our house is.  
 \item Draw a graph showing how the height has changed over time.  Extend your graph for another 20 years into the future.
\item When will the height be only 44 inches?  First use successive approximation to find the answer to the nearest year.  Then set up and solve an equation to find the answer to the nearest month.  Remember it's already an old house, so figure out how many years \emph{from now.}
\item Once the porch is below 40 inches, we will have to do something to fix it. Set up and solve an inequality to find the answer to the nearest month.  
\end{enumerate} 

\item (3.2 -- revamp to include both 3.1 and 3.2 parts.  Also appears on Practice Exam 2) The Torkelinsons want to dig a new well for water for their lake cabin.  The company charges \$900 to bring the equipment on site and draw the permit and then \$2 per foot to dig.  Earlier found the equation $$W = 900 + 2F$$ where $W$ is the total cost to dig the well, in dollars, and $F$ is the depth of the well, in feet.
\begin{enumerate}
\item In their neck of the woods, wells often run 200 feet deep.  According to the equation, how much would that cost?
\item The Torkelinsons have budgeted between \$1,500 and \$1,800 for the well.  Set up and solve a chain of inequalities to find the corresponding range of depths.
\item No such luck.  The company had to drill much deeper than hoped to find water.  The final result, wonderfully cold and clear drinking water.  And, a hefty bill for \$2,072.  How deep is their well?  Set up and solve an equation.
\end{enumerate} 

\item (3.3) The amount of snow in a snowball, $C$ cups, depends on the diameter (distance across) of the snowball, $D$ inches according to the equation $$C = 0.036D^3$$
\begin{enumerate}
\item How many cups of snow are needed to make a snowball that's 3 inches across? 5 inches across?
\item How many cups of snow are needed to make a snowball that's 2 feet across?
Give your answer in gallons.  Usel that $1 \text{ gallon} = 4 \text{ quarts}$ and $1 \text{ quart} = 4 \text{ cups}$.
\item  If Karen has a five gallon paint bucket packed with snow and wants to make one giant snowball out of it, how big will the snowball be? 
\end{enumerate}

\item (3.3)  Jorge deposited \$1,500 in an high yield money market account 5 years ago.  The \textbf{return on investment}, or equivalent annual interest rate, $r$ can be calculated from the current value of Jorge's account \$$A$ by solving the equation
$$A = 1,500g^5$$ to find $g$ and then using that $r= g-1$.  The return on investment is $r$ written as a percentage.
\begin{enumerate}
\item Find Jorge's return on investment if the current value of his account is \$1,743.
\item Repeat if the current value is \$1,982 instead.
\end{enumerate}

\item (3.4) Goldie the goldfish, Pinches the lobster, Quackers the duck, Speedy the turtle.  Expectations back in 1994 were that first generation Beanie Babies toys made would increase in value according to the equation $$B = 6(1.08)^Y$$ where $B$ is the value of Beanie Babies (in \$) and Y is the years since 1994.
\begin{enumerate}
\item If expectations were correct, what was a Beanie Baby toy worth in the year 2004?  What was it worth in 2010?
\item According to your equation, when will Beanie Babies made in 1994 be worth over \$100?  Report the actual year.
\item Show how you can solve the equation to find the answer.
\item Because of the recession, the value actually slowed a bit starting in 2004.  The revised equation is $$B = 12.95 \ast 1.072^T$$ where now $T$ is measured from 2004 instead.  What is the revised estimate for 2010 and when will it be worth over \$100?  Solve an equation to answer. Report the actual year.
\end{enumerate} 


\item (3.5) A rabbit jumps so that her height is given by the formula $$R = 17.6S - 22S^2$$ where $R=$ height of rabbit (feet) and $S=$ time (seconds).
\begin{enumerate}
\item At what height did the rabbit start her jump?
\item Can the rabbit jump over a 3 foot fence?  First,  guess the answer by successive approximation.  Next, set up and solve a quadratic equation.  Last, find the maximum height of the rabbit.
\item How long is the rabbit in the air?  Can you guess the answer from your work so far?  Then use the Quadratic Formula to find the answer.  \emph{Hint:  $c=0$}
\end{enumerate}

\item (3.5) Revenue from airline travel decreased after September 11, 2001 but has been increasing since.  The total revenue $R$ made is given by the equation $$R = 22.16Y^2-71.71Y+135.3$$
where $R$ is number of annual revenue, in billions of dollars, and $Y$ is the year since 2000.  SU LOOK AT EXCEL GRAPH!  Not sure if these numbers are correct.  Might want to hide the 9/11 reference for several reasons.  First, being that then you can ask why min was when it was (if it was then, sigh)>

\begin{enumerate}
\item Calculate the missing values in the table.
\begin{center}
\begin{tabular} {|c ||c |c |c |c |c |c |c |} \hline
$Y$ & 0   &1& 2 &3 &4&5 & 6  \\ \hline
$R$ &135.3 &  \quad ~& 80.52 & 119.16 & 203.02 &  \quad ~  & 502.80  \\  \hline
\end{tabular}
\end{center}
\item Draw a graph showing how airline travel revenue has changed over time.
\item According to this equation, in what year was the revenue the smallest?  What was the revenue for that year?  Show how to use the appropriate formula to calculate the revenue, according to the equation.  \emph{Be sure to show some work.}
\end{enumerate}



%\item (REJECT BUT COULD USE)

%The path of shot putter Randy Barnes' record-breaking put in 1990 closely followed the parabolic arch given by the equation \[ H = -0.0149D^{2}+1.043D+6.6 \] where D is the distance travelled horizontally, and H is the height above the ground of the shot, both in feet?
%\begin{enumerate}
%\item How high does the shot get?
%\item How far away did the shot land?
%\item What is the average rate of change of height versus distance over the first inch of horizontal distance?
%\end{enumerate}  % SU would be better as baseball story!  Or football, calculate hang time!

%%% END

\end{enumerate}

SU -- edit and split later

\end{document}

