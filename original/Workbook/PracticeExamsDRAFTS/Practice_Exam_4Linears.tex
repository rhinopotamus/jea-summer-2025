\documentclass[12pt]{article}
\pagestyle{empty}
\setlength{\parskip}{0in}
\setlength{\textwidth}{6.8in}
\setlength{\topmargin}{-.5in}
\setlength{\textheight}{9.3in}
\setlength{\parindent}{0in}
\setlength{\oddsidemargin}{-.7cm}
\setlength{\evensidemargin}{-.7cm}

\usepackage{amsmath}
\usepackage{amsthm}
\usepackage{amstext}

\usepackage{graphicx}

\begin{document}

\emph{Try taking these practice exams under testing conditions:  no book, no notes, no classmate's help, no electronics (computer, cell phone, television). Give yourself one hour to work and wait until you have tried your best on all of the problems before checking any answers.}
\bigskip

\subsection*{Practice exam 4-- version a}
\bigskip
 \emph{Relax.  You have done problems like these before.  Even if these problems look a bit different, just do what you can.  If you're not sure of something, please ask! You may use your calculator.  Please show all of your work and write down as many steps as you can.  Don't spend too much time on any one problem.  Do well.  And remember, ask me if you're not sure about something.}
 
\bigskip
 
\emph{A few formulas from our book:}

\begin{center}

FORMULAS PRINTED ON EXAM GO HERE

\end{center}

\hspace{-.25in} \hrulefill

\begin{enumerate}
\item (4.1) Forrest's candy bucket weighs 4 ounces when empty.  Each Snickers Almond bar he puts in his bucket weighs 1.76 ounces.  (For those who can remember, Snickers Almond is essentially the old Mars Bar renamed.  But, I digress.)  The total weight $T$ ounces of Forrest's candy bucket depends on the number of Snickers Almond bars $B$ in it according to the equation $$T=4+1.76B$$
\begin{enumerate}
\item Make a table of values showing the weight of the bucket if it contains 1, 5, 12, or 20 candy bars.
\item Draw a graph illustrating the dependence.  
\item If Forrest's candy bucket can hold up to 3 pounds (that's 48 ounces), approximately how many candy bars can it hold? 
\emph{Say what the answer is and mark the point on your graph that shows the answer.}
 \item Solve the inequality $4+1.76B \le 48$.  Interpret!
\end{enumerate}

\item (4.1) Kathy is a single mom trying to raise her kids on the salary from her part-time job.  She gets help from the state. Each month they give her \$300 credit on her EFT card that she can use to buy groceries.  Every day after work Kathy stops at the corner store and buys \$10 worth of food, which is deducted from her EFT card.
\begin{enumerate}
\item How much does Kathy have left on her card at the end of the month (30 days)?
\item Name the variables and write an equation relating them.
\item Identify the slope and intercept.  Don't forget the units.
\item Calculate a few more values and use them to draw a graph of Kathy's EFT balance for up to 30 days.  Don't forget to include day 0.
\item How would the equation be different if she began with \$400 in credit and spent \$12 each day?  
\item Draw that line on your graph too.
\end{enumerate}

\item (4.2)  The Vang family want to buy a new washing machine.  The first model costs \$645 and then \$13.29 per month to run or another, more efficient, model costs \$940 and then \$7.82 per month to run.  If $M$ is the number of months and $V$  is the Vang family's total cost  (in \$), then the equations and some comparable values (to the nearest \$) are:

SU REFORMAT TABLE!
 \begin{figure} [h]  
\begin{minipage}{2.5in} 
\begin{center}
\begin{tabular} {ll} \\
\textbf{First model:} &$V = 645 + 13.29M$ \\ 
\textbf{Second model:}  & $V = 940 + 7.82M$ \\ 
\end{tabular}
\end{center}
\end{minipage} 
\begin{minipage}{4 in} 
\begin{center}
\begin{tabular} {|l|r|r|r|} \hline
$W$ & 12 & 36 & 60 \\ \hline
\textbf{First model:} & 804.48  & 1,123.44 &  1,442.40  \\ \hline
\textbf{Second model:} &  1,033.84 &  1,221.52 &  1,409.20 \\ \hline
\end{tabular}
\end{center}
\end{minipage} 
\end{figure} 
\vspace{-.1in}
\begin{enumerate}
\item Draw a graph illustrating both equations. \emph{Be sure to include the intercepts.}
\item What's the  \textbf{payback time} (the number of months for which the total costs of each washing machine are equal)?
\end{enumerate}


\item (4.2)  BUT MAYBE NOT USE?  Nils and Frida want to rent a limo for the prom.  A Cadillac rents for \$50 plus \$6 / hour. A  Lincoln rents for \$25 plus \$12 / hour.
\begin{enumerate}
\item If he wants the limo for 3 hours, how much will it cost for the Cadillac?  for the Lincoln?
\item If he wants the limo for 8 hours, how much will it cost for the Cadillac?  for the Lincoln?
\item At how many hours do the two different types of limo cost the same? Use succesive approximation to find the answer.  Display your work in a table.
\item Now answer the question by solving an equation.
\item Solve an inequality to find the range of times for which the Cadillac costs no more than \$75 to rent.
\item Solve an inequality to find the range of times for which the Cadillac costs less than the Lincoln to rent.
\end{enumerate}

\item (4.2)  Will women every run the marathon as fast as men do?  The world records are getting close.  In 2012 the men's record was 2:03:38 and the women's record was 2:15:25.  (Each time is given in H:M:S, or hours:minutes:seconds, format.)  That's only about 12 minutes apart!  On the other hand, the record is changing very slowly.  Estimates for the men's time shows about 13 second drop per year on average.  For women's that's fast, but still only 26 seconds drop per year on average.

\hfill \begin{footnotesize} Source:  Wikipedia (Marathon World Record Progression) \end{footnotesize}
%Data from wikipedia shows Men's record 2003 = 2:04:55, then 2007 = 2:04:26, then 2008=2:03:59, and then 2011=2:03:38.  For the women it's 199 = 2:20:43, then 2001 = 2:19:46, then 2002 = 2:17:18, and then 2003 = 2:15:25.
\begin{enumerate}
\item Convert the current record time for men and for women into seconds.
\item Write an equation for each function.  Don't forget to name the variables.
\item Set up and solve a system to estimate when the women's record might equal the men's record.
\item Draw a quick graph to check.
\end{enumerate}

\item  (4.3) Arjun just graduated from college but is living with his uncle for the summer to save money.  They agreed that Arjun would do chores and some light renovations instead of paying rent.  Arjun has been doing around 5 hours of work a week for the past 8 weeks, but still owes his uncle another 30 hours of work.
\begin{enumerate}
\item What was the original agreement? That means, how many hours of work did Arjun promise his uncle?
\item Name the variables and write an equation relating them, assuming Arjun does 5 hours a week of work.
\item How many more weeks will it take Arjun to finish the work he promised? 
\end{enumerate}

\item (4.3) An online music club charges a monthly enrollment fee plus \$.95 per album you download.  Last month Andrew downloaded 31 albums for a total cost of \$49.00.  
\begin{enumerate}
\item What is the monthly enrollment fee?
\item Name the variables, including units, and write an equation relating them.
\item How much (total) will it cost  in a month where Andrew downloads 20 albums?
\item If the bill next month is for \$87.95, how many albums did Andrew download? Show how to solve the equation.
\item If it takes Andrew approximately 43 minutes to listen to each album, how long will it take him to listen to those albums he downloads next month (your answer to (d))?  Round your answer to the nearest hour.
\end{enumerate}

\item (4.4) A report  shows September sea-ice declining in the Northern hemisphere. In 1980 the extent of the sea-ice was 3.1 million square miles.  By 2012, the sea-ice extended only 1.7 million square miles.  For this problem, suppose that the area of sea-ice decreases linearly. 
\hfill \begin{footnotesize} Source: National Snow and Ice Data Center \end{footnotesize}
\begin{enumerate}
\item Name the variables, including units.
\item Display the information from the story in a table.
\item What is the rate of sea ice decrease? 
\item Write a linear equation relating your variables.
\item Scientists are concerned that if the September sea-ice falls between 200,000 and 500,000 square miles, then other climate feedbacks will lead to no more sea-ice in September.  According to your equation, in what years is this expected to occur?  Set up and solve an inequality to answer the question.
\item Graph and check.
\end{enumerate}

\item (4.4) The local zoning commission is considering a plan to expand housing in the city, as measured in the number of residential units.  But with more residential units come more shops, offices, schools, recreational facilities, churches, and other commercial property. Currently the city has 3,500 residential units and 1,575 acres of commercial property.  If the propsal is passed in completed, the city will have a new total of 3,600 residential units and 1,620 acres of commercial property.  You can assume this increase is linear.
\begin{enumerate}
\item Name the variables and summarize the given information in a table.
\item Approximately ow many new acres of commercial property are there for each new residential unit built?
\item Write an equation relating the variables.
\item If the city decides to limit the amount of land to 1,600 acres of commercial property, approximately how many residential units can there be?  Use successive approximation. 
\item Now answer the question by setting up and solving an equation.
\end{enumerate}

\item (4.5) As people age, they begin to experience hearing loss.  A study was done to determine the ``comfort level'' of sound for people of different ages, meaning the loudest sound (in decibels) that the person could listen to comfortably.  The data are given in the table below.
\begin{center}
\begin{tabular} {|l |c |c |c |c |c |c |c |c |} \hline
Name & Akbar & Javier & Walter & Xang & Rolf & Derrick & Iago &Raheem \\ \hline
Age & 45 & 45 & 55 & 65 & 75 & 75 & 85 & 85\\ \hline
Comfort level & 58 & 61 & 63 & 71 & 75 & 80 & 82 & 79 \\ \hline
\end{tabular}
\end{center}
\begin{enumerate}
\item Make a scatterplot showing the data.  
\emph{Scale your axes to start at 40 years and start the level at 56 decibels.  \\ Spread out your scale to get a large, detailed graph.}
\item  Draw the line through the points listed for Xang and Rolf.  Explain why that line does not fit the data well.  \emph{Label this line B.}
\item  SU TELL THEM BEST FIT LINE AND HAVE THEM DRAW IN.
\end{enumerate}


\item  (4.5) It's been a long time since anyone broke the record for the men's long jump.  In 1935 Jesse Owens jumped  8.13m.  The record was next broken 25 years later (in 1960) by Ralph Boston who jumped 8.21m. He broke his own record several times over the next few years, including being surpassed briefly by Igor Ter-Ovanesyan.  Ralph's final record was 8.35m in 1965. Not to be outdone, Igor tied the record in 1967. Then in 1968, Bob Beamon jumped 8.90m.  That record held for 23 years, until Mike Powell jumped 8.95m in 1991 (much to Carl Lewis' dismay).  Powell's record still stood 21 years later, in 2012. \hfill \begin{footnotesize} Source:  Wikipedia (Long Jump) \end{footnotesize}
% source:  http://en.wikipedia.org/wiki/Long_jump 
\begin{enumerate}
\item SU GIVE THEM SCATTER PLOT
\item Draw in the line connecting the data from 1935 and 2012.  Use it to predict the long jump record in 2020.
\item Draw in the line connecting the data from 1991 and 2012.  Use it to predict the long jump record in 2020.
\item Which of your lines do you prefer, and why?  SU CHECK IF EITHER IS GOOD.
\end{enumerate}
%%%% END

\end{enumerate}

SU -- edit and split later

\end{document}

