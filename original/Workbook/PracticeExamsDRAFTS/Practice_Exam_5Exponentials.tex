\documentclass[12pt]{article}
\pagestyle{empty}
\setlength{\parskip}{0in}
\setlength{\textwidth}{6.8in}
\setlength{\topmargin}{-.5in}
\setlength{\textheight}{9.3in}
\setlength{\parindent}{0in}
\setlength{\oddsidemargin}{-.7cm}
\setlength{\evensidemargin}{-.7cm}

\usepackage{amsmath}
\usepackage{amsthm}
\usepackage{amstext}

\usepackage{graphicx}

\begin{document}

\emph{Try taking these practice exams under testing conditions:  no book, no notes, no classmate's help, no electronics (computer, cell phone, television). Give yourself one hour to work and wait until you have tried your best on all of the problems before checking any answers.}
\bigskip

\subsection*{Practice exam 5-- version a}
\bigskip
 \emph{Relax.  You have done problems like these before.  Even if these problems look a bit different, just do what you can.  If you're not sure of something, please ask! You may use your calculator.  Please show all of your work and write down as many steps as you can.  Don't spend too much time on any one problem.  Do well.  And remember, ask me if you're not sure about something.}
 
\bigskip
 
\emph{A few formulas from our book:}

\begin{center}

FORMULAS PRINTED ON EXAM GO HERE

\end{center}

\hspace{-.25in} \hrulefill

\begin{enumerate}
\item (5.1, 5.3) The number of school children in the district whose first language in not English has been on the rise.  The equation describing the situation is $$C=673(1.043)^Y$$ where $C$ is the number of school children in the district whose first language is not English, and $Y$ is the number of years (from now).
\begin{enumerate}
\item Make a table showing the number of school children in the district whose first language is not English now, in one year, in two years, and in ten years. \emph{Don't forget now too.}
\item ADD QUESTION what is the percent?
\item Use successive approximation to determine when there will be over 1,700 school children in the district whose first language is not English.  \emph{Display your work in a table.  Round your answer to the nearest year.}
\item Show how to solve the equation to calculate when there will be over 1,700 school children in the district whose first language is not English. \emph{Show how you solve the equation.}  
\item When will the number pass 1,000?  Set up and solve an equation. Then check your answer.
\end{enumerate}

%\item (5.1) How about people who adopt new technology, new fashion, new language, . . .  See follow up in logistic section.  If give equation, can also follow up in 5.3 w/ reading off percent change.

\item (5.1) Something about people who adopt new fashion trends. Leopard print hat. Originally 5 out of 1,000 women shopping at a major retail store even looked twice.  But that number grew and grew, by my estimate around 40\% a week, thanks to carefully placed ads in fashion magazines.
\begin{enumerate}
\item Name the variables and write an equation relating them.  (SU tell them one variable is ``interest in the leopard hat'' measured in women per thousand?)
\item Make a table showing the number of women, per thousand female shoppers, who stop and look at the hat 1 week, 2 weeks, and 3 weeks after it hits the stores.
\item The leopard print hat is considered popular when more than 300 out of 1,000 women try it on.  According to the equation, when will the hat be considered popular?
\item The hat will be considered pass\'e when over 750 out of 1,000 women try it on.  I mean -- everyone's got one!.  According to your equation, when will that happen?
\end{enumerate}

NOTE;  see note about language of "early adopters", etc.

\item (5.2) A patient is given an initial dose of 120 mg of a medicine.  It has now reached full levels, and is expected to drop around 5\% per hour.  What is the half-life of the medicine? As part of your work, name the variables, write an equation, solve your equation, and check your answer.

\item (5.2) SU FINISH.  something about doubling your money by betting on Red/Black at Roulette.  Initial bet of \$300.  How is money a function of consecutive wins, if each win doubles your money?  What's a realistic domain?

%\item (5.3)  \item  One of the toxic radioactive elements produced by nuclear power plants is strontium-90. A large amount of strontium-90 was released in the nuclear accident at Chernobyl in the 1980's.  The clouds carried the strontium-90 great distances. The rain washed it down into the grass, which was eaten by cows. People then drank the milk from the cows.  Unfortunately, strontium-90 causes cancer. Strontium-90 is particularly dangerous because it has a half-life of approximately 28 years, which means that every 28 years half of the existing strontium-90 changes into a safe product (zirconium-90); the other half remains strontium-90. Suppose that a person drank milk containing 100 milligrams of strontium-90.
%SOURCE:  ``Explorations in College Algebra,'' by Kime and Clark 
%\begin{enumerate}
%\item After 28 years, how many milligrams of strontium-90 remains in the person's body?
%\item After 56 years, how many milligrams of strontium-90 remains?
%\item Find the average annual percentage decrease of strontium-90.
%\item Name the variables and write an equation relating them.
%\item Suppose that any amount under 20 milligrams of strontium-90 is considered ``acceptable'' in humans. How long will this person have to wait until their level is acceptable? (Do you think they will be alive?) Display your work in a table.
%\item Draw a graph illustrating the amount of strontium-90 remaining for 100 years after ingestion.
%\item Set up and solve an equation to determine how long this person will have to wait until their level is below 20 milligrams.
%\end{enumerate}

\item  (5.3) The economic recession has a huge impact on the retail sector.  A large, not-to-be-named retail store reported that their total sales have decreased.  In November of 2008, they made \$1.03 billion in sales.  Three months later, sales dropped to \$.61 billion.
\begin{enumerate}
\item Calculate the monthly growth factor, assuming sales have decreased exponentially.
\item On average, by what percentage per month are sales decreasing?
\end{enumerate}

\item (5.3) Revive the jackpot question -- see Excel list from my old final exams, I think.


\item (5.4) The number of geese in the Twin Cities metropolitan area increased from 480 in 1968 to 25,000 in 1994.  Although population is sometimes modeled with exponential models, there are many factors that might make an exponential model inappropriate, such as changes in migration, wetlands, and hunting.
\begin{enumerate}
\item Name the variables and write both a linear equation and an exponential equation modeling the relationship.
\item Compare the models projections for 1968, 1975, 1984, 1994 (as a check), 2000, 2010, and 2020.  Summarize your findings in a table.
\item Graph each function over the period from 1968 to 2020 on the same set of axes.
\item Research indicates that the Twin Cities metropolitan area could support 60,000 geese.  Use your graph to estimate when that will happen.
\item The actual goose population in 2010 was around 50,000.  Which model was closer?
\end{enumerate}  %SOURCE???

\item (5.4) Many different agencies and governments are working to lower infant mortality.  Infant mortality is measured in deaths per 1,000 births.  The world infant mortality rate in 1955 was around 52 (per 1,000 births).  By the year 2000, it was down to around 23.
\hfill \begin{footnotesize}  Source: Wikipedia (Infant Mortality) \end{footnotesize}
\begin{enumerate}
\item Linear
\item Exponential
\item Compare to actual data?  1955 = 52, 1960 = 47, 1965 =43, 1970=40, 1975 =37, 1980=34, 1985=31, 1990=28, 1995=25, 2000=23 Really is averaged over the 5-year periods (1950-1955, . . ., 1995-2000)
\end{enumerate}

\item (5.5) IDEAS:  See wikipedia, models language change.  Also measures adoption of any practice -- new technology, fashion.  Also used more generally in sociology for the ``tipping point.''  Econ calls this ``diffusion of innovation''  See picture from wiki on diffusion of innovation for the language -- innovators, early adopters, early majority, late majority, laggards.  Give credit.  Might be great way to ask questions about points on the graph.

 Bootcut jean -- out.  Skinny jeans -- in.  Once a trend only worn by fashionistas, skinny jeans are now appearing on the legs of soccer moms everywhere.  

Are you an early adopter of new technologies?  Or are you part of the early majority, late majority?  Or perhaps you're a laggard?  When new innovations . . .

How do new words enter our language?  Turns out the answers are quite varied.  At first innovators coin a term.  Early adopters . . .

See wikipedia, growth of tumors.


\item (5.5)  Revisit diet as upside down saturation or logistic?

\item (5.5) SU:  I'm not sure if this is realistic if we're only counting the NEW cases in each round.  I would think that the logistic counts the CUMULATIVE.  Fix or toss. You and two buddies each invite 10 people to ``like'' your online group.  Suppose everyone accepts and then they each invite 10 people.  And then everyone accepts and they each invite 10 people.  And so on. Of course, there is likely to be substantial overlap, but for the moment pretend that there isn't we have -- SU:  state variables and exponential equation here, if needed.  It was $$N=3 \ast 10^R$$
A logistic model is more realistic.  CONSIDER $$ N = \frac{1,000}{1+333 \ast .11^R}$$

SU this story is from practice exercise 5.2 \#4.
\begin{enumerate}
\item Make a table exponential vs. logistic and compare.
\item Graph the logistic.
\item Long term?
%R = 0 & 1 & 2 & 3 & 4 & 5 & 10
% EXP = 3 & 30 & 300 & 3000 & 30000 & 300000 & 3 $\times 10^10$
% LOG = 3, 27, 199, 693, 954, 995, 1000
\end{enumerate} 
%%% END

\end{enumerate}

SU -- edit and split later

\end{document}

