
\section{Solving linear equations }

Your kitchen sink keeps getting clogged.  Very annoying. Last time %(in section 2.1 SU CITE), 
the plumber was able to fix it pretty quickly, well ahead of when your dinner guests were due.  But now the sink is clogged again.  This time when the plumber comes and unclogs the sink, he suggests redoing the trap and a few other things that were causing the problem.  You are pretty tired of it clogging up and tell him to ``go ahead.''  He warns you that the parts will cost around \$20.  While you're glad that the sink works when he's done, you're a bit shocked when his bill arrives a few days later for \$291.70, of which \$12.95 is parts and \$278.75 is labor. Does that seem right?

Remember our plumber charged \$100 for just showing up and then \$75 per hour for the service.  Using the variables  
\begin{center}
\begin{tabular} {l} 
$T=$ time plumber takes (hours) $\sim$ indep \\
$P= $ total plumber's charge (\$) $\sim$ dep \\ 
\end{tabular}
\end{center}
we found that the equation was $$P=100+75T$$  

Let's figure out how many hours of work would add up to a bill of \$278.75.  Our first approach might be to look at a table.  From earlier we had

\begin{center}
\begin{tabular} {|l| |c |c |c |c |c |c  |} \hline
$T$ & 0 & 1/2 & 1 & 2 & 3 \\ \hline
$P$ & 100.00 & 137.50 & 175.00 & 250.00 & 325.00 \\ \hline
\end{tabular}
\end{center}

\noindent Since \$278.75 is between \$250.00 and \$325.00, we see that the time must be between 2 and 3 hours. You remember the plumber being there over 2 hours, so this is certainly a reasonable answer.  Well, a lot of money, but mathematically it makes sense.

Still curious, you'd like to know exactly how many hours and minutes he worked.  We could use successive approximations. %, as we did in section 2.4 (SU CITE).  
For example, for 2.5 hours the bill would have been 
$$P=100 + 75 \ast 2.5 = 100+75\times \underline{2.5}=\$287.50$$ which is more than the charge. Continuing to guess and check, and displaying our work in a table, we get
\begin{center}
\begin{tabular} {|l| |c |c |c |c |c |c |c |c |c |} \hline
$T$ & 2 & 3 & 2.5 & 2.3 & 2.4 & 2.35 & 2.37 & 2.38\\ \hline
$P$ & 250.00 & 325.00 & 287.50 & 272.50 & 280.00 & 276.25 &277.75 &278.50 \\ \hline
vs. $278.75$& low & high & high & low & high & low & low & close enough\\ \hline
\end{tabular}
\end{center}

\noindent The plumber worked  approximately 2.38 hours.  
Converting units we calculate $$.38 \text{ \cancel{hours}} \ast  \frac{60 \text{ minutes}}{1 \text{ \cancel{hour}}} = .38 \times 60 = 22.8 \approx 23 \text{ minutes}$$  The plumber took about 2 hours, 23 minutes.  Thinking back, the plumber had arrived around 10:30 in the morning and stayed past lunch, probably until around 1:00 p.m. That's about right.

Wait a minute!  We could have figured this out much more quickly.  If the labor cost was \$278.75, we know the first \$100 was the trip charge.  That leaves $$\$278.75-\$100.00 = \$178.75$$ in hourly charges.  At \$75 per hour that comes to $$\$178.75\ast  \frac{\text{1 hour}}{\$75}=178.75 \div 75 = 2.3883333\ldots \approx 2.388 \text{ hours}$$ which comes to around 2 hours, 23 minutes as before.  See how we used the \$75/hour as a unit conversion here?  Very clever.

That worked well.  But, can we figure out this sort of calculation in other problems?  What is the general method we're using? Can we write down our method in an organized fashion so that someone else could follow our thinking here?  Turns out there is a formal way to show this calculation, called \textbf{(symbolically) solving} the equation. Officially \emph{any} method of getting a solution to an equation is considered solving the equation, but in the rest of this book, and in most places that use algebra, when we refer to ``solving the equation'' or give the instruction to \textbf{solve}, we mean \emph{symbolically}.

Here's how it works.  We want to figure out when $P=278.75$.  We know from our equation that $P = 100 + 75T$.  So we want to find the time $T$ where  $$100 + 75T=278.75$$  Remember that the equal sign indicates that the two sides are the same number.  On the left-hand side we have $100 + 75T$.  On the right-hand side we have $278.75$.  Looks different, but same thing.  Again, we're looking for the value of $T$ that makes these two sides the same number.  

The first thing we did to figure out the answer was subtract the \$100 trip charge.  In this formal method, we can subtract 100 from each side of our equation.  I mean, if the left-hand side and the right-hand side are the same number, then we sure better get the same answer when we take away 100 from each side, right?  When we subtract 100 from each side we get 
\begin{eqnarray*}
\cancel{100} +75T & = & 278.75\\
-\cancel{100}\hspace{.4in}~& &  -100 \\  %HSPACE
\end{eqnarray*} 
\vspace{-.5in} %VSPACE

\noindent which simplifies to $$75T=278.75-100=178.75$$ because the $+100$ and $-100$ cancelled.  

The next thing we did to figure out the answer was divide by the \$75/hour charge.  In this formal method, we can divide each side of our equation by 75.  Again, if the left-hand side and right-hand side are the same number, then we will definitely get the same answer when we divide by 75.  Here goes.
$$\frac{\cancel{75}T}{\cancel{75}}=\frac{178.75}{75} $$
Notice that we wrote the division in fraction form (instead of using $\div$).  To understand why the 75's cancelled, remember that $75T$ is short for $75\ast T$ and so $$\frac{75T}{75} = \frac{75\ast T}{75} = 75 \times T \div 75=T$$ because the $\times 75$ and $\div 75$ cancelled.  So we have $$T = \frac{178.75}{75} = 178.75 \div 75 = 2.3883333\ldots$$ as before.  Yet again, our answer is around 2 hours, 23 minutes.

Let's practice working with this symbolic way of solving equations.  Suppose instead the plumber went to my neighbor's house and billed her for \$160 in labor costs.  How long did the plumber work at my neighbor's?  As before, we begin with our equation $$P = 100 + 75T$$  and we are looking for $P=160$.  Put these together to get $$100+75T =160$$ 
Subtract 100 from each side to get
\begin{eqnarray*}
\cancel{100} +75T & = & 160\\
-\cancel{100}\hspace{.4in}~& &  -100 \\  %HSPACE
\end{eqnarray*} 
\vspace{-.5in} %VSPACE

\noindent which simplifes to $$75T=160-100=60$$  
Last, divide each side by 75 to get $$\frac{\cancel{75}T}{\cancel{75}}=\frac{60}{75}$$ which simplifies to $$T = \frac{60}{75} = 60\div75=.8 \text{ hours}$$  

We have solved the equation, but it would make more sense to report our answer in minutes so we convert $$.8 \text{ \cancel{hours}} \ast \frac{60 \text{ minutes}}{1 \text{ \cancel{hour}}} = .8 \times 60 = 48 \text{ minutes}$$  The plumber worked for 48 minutes at my neighbor's house.

Let's quick check.  Since $T$ is measured in hours we need to go back and use $T= 0.8$ hours, not 48 which is in minutes.  Evaluating in our original equation we get $$P=100 + 75 \ast .8 = 100+75\times\underline{.8}= 160 \quad \checkmark$$  

You might be wondering how we knew to subtract the 100 first and then later divide by 75.  In this particular situation we had figured it out already and knew it made sense to take the \$100 right off the top.  But, in general, how would we know?  

It turns out that when solving an equation we do the \emph{opposite} operations in the \emph{reverse} order from the usual order of operations for evaluating.  To evaluate a linear equation we would first multiply and then add.  To solve a linear equation we first subtract (that is the opposite of adding) and then we divide (that is the opposite of multiplying).  % This is cited in the power equations section so keep or toss both.

%\newpage
%Add solving inverse proportion and, heck, go ahead and graph it to see nonlinear in the exercises.
% Hey -- do not use fahrenheit vs celsius in the exercises because that is now the narrative example in section 3.2 on linear ineq

%%\section{Solving linear equations}

 \begin{center}
\line(1,0){300} %\line(1,0){250}
\end{center}

\section*{Homework}

\noindent \textbf{Start by doing Practice exercises \#1-4 in the workbook.}

\bigskip

\noindent \textbf{Do you know \ldots}

\begin{itemize} 
\item When you solve an equation, as opposed to just evaluating?  
\item Why we ``do the same thing to each side'' of an equation when solving? 
\item How to solve a linear equation? 
\item The advantages and disadvantages of solving versus successive approximation? 
\item How to check that a solution is correct using the equation? 
\item[~] \textbf{If you're not sure, work the rest of exercises and then return to these questions.  Or, ask your instructor or a classmate for help.}
\end{itemize}

\subsection*{Exercises}

\begin{enumerate} 	
\setcounter{enumi}{4}

\item A charter boat tour costs $\$C$ for $P$ passengers, where
$$C = 135.00 + 11.95P$$ % Did not do variation on main example because we're really bored of the whole plumber thing!
\begin{enumerate}
\item Make a table of values showing the charges for no passengers, 4 passengers, 10 passengers, and 20 passengers. 
\item What does the 135.00 represent and what are its units?
\item What does the 11.95 represent and what are its units? 
\item If Freja was charged $\$ 326.20$ for use of the boat, how many passengers were there? Set up and solve an equation to answer the question.  
\item Graph and check.
\end{enumerate}

\item Abduwali has just opened a restaurant. He spent \$\text{82,500} to get started but hopes to earn back \$\text{6,300} each month.  Earlier we determined that $$A = \text{6,300}M - \text{82,500}$$ describes how Abduwali's profit \$$A$ is a function of how long he works ($M$ months). 

\hfill \emph{Story also appears in 2.1 Exercises}
\begin{enumerate}
\item Set up and solve an equation to determine how long it will take Abduwali to \textbf{break even}, meaning make a profit of \$0?
\item Aduwali will consider the restaurant a success once he's earned \$\text{100,000}.  According to our equation, when will that be?
\end{enumerate} 

\item Between e-mail, automatic bill pay, and online banking, it seems like I hardly ever actually mail something.   But for those times, I need postage stamps. The corner store sells as many (or few) stamps as I want for 44\textcent~each but they charge a 75\textcent~convenience fee for the whole purchase.  \hfill \emph{Story also appears in 1.1 Exercises}
\begin{enumerate}
\item Make a table showing the cost to buy 5 stamps, 10 stamps, or 20 stamps from the corner store.
\item Name the variables and write a linear equation showing how the total price depends on the number of stamps I buy.
\item My partner bought postage stamps at the corner store and it cost him \$7.35.  Solve your equation to determine how many stamps she bought. 
\item How many stamps could I buy for \$10?  Solve your equation and check your answer.
\end{enumerate} 

\item When Gretchen walks on her treadmill, she burns 125 calories per mile.  Recall $$C=125M$$ where $C$ is the number of calories Gretchen burns by walking $M$ miles. 

 \hfill \emph{Story also appears in 2.1 and 3.2 Exercises}
\begin{enumerate} 
\item Set up and solve an equation to calculate how far Gretchen has to walk to burn 300 calories. 
\item If Gretchen walks 3.4 miles per hour on her treadmill, how long will it take her to burn those 300 calories?  Report your answer to the nearest minute.
\item Pecan pie? Yum.  Not fitting into your favorite jeans?  No fun.   How far does Gretchen have to walk to burn off the calories from those two slices of pecan pie she ate last night?  Each slice has approximately 456 calories.
\end{enumerate} 

\item The more expensive something is, the less likely we are to buy it.  Well, if we have a choice.  For example, when strawberries are in the peak of season, they cost about \$2.50 per pint at my neighborhood farmer's market and demand is approximately 180 pints.  (That means, people want to buy about 180 pints at that price.) We approximate that the demand, $D$ pints, depends on the price, \$$P$, as described by the equation $$D = 305 - 50P$$
\begin{enumerate}
\item How many pints of strawberries are in demand when the price is \$3.19 per pint?
\item Make a table of values showing the demand for strawberries priced at  \$2.00/pint, \$2.25/pint, \$2.50/pint, \$2.75/pint, \$3.00/pint, \$3.25/pint, \$3.50/pint.
\item Draw a graph illustrating the function.  Start at \$0/pint even though that's not realistic.
\item It's been a great week for strawberries and there are 240 pints to be sold at my neighborhood farmer's. What price should the farmer charge for her strawberries in order to sell them all? Estimate your answer from the graph.  Then set up and solve an equation to answer the question.
\end{enumerate}

\item The stretch of interstate highway through downtown averages \text{1,450} cars per hour during the morning rush hour.  The economy is improving (finally), but with that the county manager predicts traffic levels with increase around 130 cars per hour more each week for the next couple of years. Earlier we found the equation $$C=\text{1,450} + 130W$$ where $C$ is the number of cars per hour during the morning rush $W$ weeks since the country manage made her projection.   \hfill \emph{Story also appears in 2.1 Exercises}
\begin{enumerate}
\item Significant slowdown are expected if traffic levels exceed \text{2,000} cars per hour.  When do they expect that will happen? Set up and solve an equation.  Don't forget to check your answer by evaluating.
\item If traffic levels exceed \text{2,500} the county plans to install control lights at the on ramps.  When is that expected to happen?   Set up and solve an equation.  Don't forget to check your answer by evaluating.
\end{enumerate} 


\end{enumerate}
