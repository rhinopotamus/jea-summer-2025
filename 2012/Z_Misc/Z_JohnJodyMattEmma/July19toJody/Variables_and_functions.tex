
\section{Variables and functions}

Things change, like the price of gasoline, and just about every day it seems.  What does it mean when the price of a gallon of gas drops from \$3.999/gal to \$3.299/gal?  The symbol / is short for ``per'' or ``for each,''  so that means each gallon costs
$$\$3.999-\$3.299=\$.70 = 70\text{\textcent}$$ less.  Does this 70\textcent \hspace{.05in}truly matter?

Before we answer that question, are you wondering why there's that extra 9 at the end of the price?  We might think a gallon costs \$3.99 but there's really a small 9 following it.  Sometimes that 9 is raised up slightly on the gas station sign.  You have to read the fine print.  What it means is an extra $ \frac{9}{10}$\textcent \hspace{.05in}for each gallon.  So the true price of a gallon gas would be \$3.999.  Gas costs a tiny bit more than you thought.  Good grief.

%For the record, the / symbol is short for ``per''.  So when we write \$3.999/gal we mean \$3.999 per gallon.  
%Sometimes we write this as a fraction instead:  $ \frac{\$3.999}{\text{gallon}}$.

Back to our question.  Does 70\textcent \hspace{.05in}truly matter to us?  Probably not.  Can't even buy a bag of potato chips for 70\textcent.  But, how often do you buy just one gallon of gas?  Typically you might put five, or ten, or even twenty gallons of gas into the tank.  We want to understand how the price of gasoline influences what it really costs us at the pump.  To do that let's compare our costs when we buy ten gallons of gas.  There's no good reason for picking ten; it's just a nice number to work with.

If gas costs \$3.999/gal and we buy 10 gallons, it costs 
$$10 \text{ gallons} \ast \frac{\$3.999}{\text{gallon}} = 10 \times 3.999  = \$39.99$$   
See how we described the computation twice?  First, with units, fractions, and $\ast$ for multiplication in what's sometimes called ``algebraic notation.''  Then, with just numbers and $\times$ for multiplication -- that's what you can type into a calculator.  % SU later, if have the appendices done, reference here.

 If gas drops to \$3.299/gal and we buy 10 gallons, it costs 
$$10 \text{ gallons} \ast \frac{\$3.299}{\text{gallon}} = 10 \times 3.299  = \$32.99$$
That's \$7 less.  For \$7 savings on gas you could buy that bag of potato chips, and an iced tea to go with it, and still have change. That amount matters.  I mean, especially since it's \$7 savings every time you put 10 gallons in the tank.

Gas prices have been changing wildly, and along with them, the price of 10 gallons of gas.  In mathematics, things that change are called \textbf{variables}.  The two variables we're focusing on in this story are
\begin{center}
\begin{tabular} {l} 
$P=$ price of gasoline (\$/gal) \\
$C= $ total cost (\$) \\ 
\end{tabular}
\end{center}
Notice that we gave each variable a letter name.  It is helpful to just use a single letter chosen from the word it stands for.  In our example, $P$ stands for ``price'' and $C$ stands for ``cost''.  In this course we rarely use the letter $X$ simply because so few words begin with $X$.  Whenever we name a variable ($P$) we also describe in words what it represents (the price of gasoline), and we state what units it's measured in (\$/gal).

In talking about the relationship between these variables we might say ``the cost depends on the price of gas,'' so $C$ depends on $P$.  That tells us that $C$ is the \textbf{dependent variable} and $P$ is the \textbf{independent variable}.  In general, the variable we really care about is the dependent variable, in this case $C$ the total amount of money it costs us.   The concept of dependence is so important that there's yet another word for it.  We say that $C$ is a \textbf{function} of $P$, as in  ``cost is a function of price.''  

Knowing which variable is independent or dependent is helpful to us.  To emphasize the dependence, we often make a notation next to the variable name.  
\begin{center}
\begin{tabular} {l} 
$P=$ price of gasoline (\$/gal) $\sim$ indep \\
$C= $ total cost (\$) $\sim$ dep \\ 
\end{tabular}
\end{center}
This labeling is rarely used outside this textbook, so add it in for yourself if you need it.  In some situations dependency can be viewed either way; there might not be one correct way to do it.  Labeling the dependence is extra important then, so anyone reading your work knows which way you are thinking of it.  

Given a choice, we usually assign dependence such that given a value of the independent variable, it is easy to calculate the corresponding value for the dependent variable.  In our example it's easy to use the price per gallon, $P$, to figure out the total cost, $C$.  We can work backwards -- from $C$ to $P$ -- but it's not as easy.  

For example, suppose we buy 10 gallons of gas and it costs \$28.99.  We can figure out that the price per gallon must be
$$P=\frac{\$28.99}{10 \text{ gallons}}= 28.99 \div 10 = \$2.899 \text{/gal}$$  Notice that we use the fraction as part of the algebraic notation, but we use $\div$ to indicate division on the calculator.  Your calculator key for division may be $/$ instead, which we reserve as a shorthand for ``per.''

From our experience we have a sense of what gas might cost.  In my lifetime, I've seen gas prices as low as 35.9\textcent \hspace{.05in}/gallon in the 1960s to a high of \$4.099/gallon recently.  This range of values sounds too specific, so it would sound better to say something general like \begin{center} ``Gas prices are (definitely) between \$0/gal and \$5/gal.'' \end{center} 

The mathematical shorthand for this sentence is $$0 \le P \le 5$$  The inequality symbol $\le$ is pronounced ``less than or equal to''.  Formally, the range of realistic values of the independent variable is called the \textbf{domain} of the function $C$.  In this text, we rarely write the domain because it's usually clear from the story what realistic values would be.  The exercises in this section ask you to do so for practice.

Be aware that there are often many different numbers in a story.  Some numbers are examples of values the variables take on, such as \$3.999/gal or \$39.99 in our example.  Other numbers are \textbf{constants}; they do not change (at least not during the story).  The one constant in our story is that we are always buying 10 gallons of gas.  Occasionally there are other numbers in a story that turn out not to be relevant at all, so be on the lookout.

Back to our story.  A report says that the average price of gasoline in Minnesota was  \$2.900/gal in 2010 and increased approximately 20\% per year for the next several years.  We would like to check what that says about the average price of gasoline in 2011 and 2012, say.  (It is unlikely that the price increase continued much longer at that rate.)

To understand what that report is saying, we need to remember how percents work.  Luckily, the word ``percent'' is very descriptive.  The ``cent'' part means ``hundred,'' like 100 cents in a dollar or 100 years in a century.  And, as usual, ``per'' means ``for each.''  Together, \textbf{percent} means ``per hundred.''  The number 20\% means 20 for each hundred.  Written as a fraction it is $\frac{20}{100}$.  Divide to get the decimal $20 \div 100 = 0.20.$   
$$\text{Think money:  }20\%\text{ is like }20\text{\textcent , and }0.20\text{ is like }\$0.20$$  
Bottom line:  20\%, $\frac{20}{100}$, and 0.20 mean exactly the same number.
$$20\% = \frac{20}{100} = 20 \div 100 = 0.20$$

To calculate the percent of a number we multiply by the decimal version.  For example,
$$20\% \text{ of } \$2.900 = 0.20 \times 2.900 = \$.58$$
The report says the price increased by 20\% each year, so by 2011 the price had increased an average of \$.58.  That's not what gas cost in 2011.  It's how much \emph{more} gas cost in 2011 compared to 2010.  To see what the report projected for the 2011 cost we need to add that increase on to the original 2010 price.
$$\$2.099 + \$0.58= \$3.48\text{ per gallon}$$
Sounds about right.  Expensive, to be sure, but fairly accurate.

For 2012, the price increased by 20\% again.  That means 20\% of what it was in 2011.  We can't just add \$.58 again.  That was 20\% of the 2010 value, and we want 20\% \emph{of the 2011 value}.  Going to have to calculate that.
$$20\% \text{ of } \$3.48 = 0.20 \times 3.48 = \$.696$$
so the projected 2012 value was $$\$3.48 + \$.696= \$4.176\text{ per gallon}$$
Yikes.

One last note.  The number 20\% in the report sounds like a rough approximation.  The report probably means the increase was around 20\%, maybe a little less, maybe a little more.  So our answers of \$3.48/gal and \$4.176/gal could be a little less or a little more too.  But they sound so perfectly correct.  To be safe, we really ought to round off these answers, to something more general like around \$3.50/gal in 2011 or approximately \$4.20/gal in 2012.  % Su cite appendix on rounding off/approx?

When we want someone reading our calculation to know that we mean approximately, not exactly, we use the \textbf{approximately equal to} symbol $\approx$.  We save the equal sign, =, for when we have not rounded off the number at all.  So, according to the report $P \approx \$3.50$/gal in 2011 and $P \approx \$4.20$/gal in 2012.  

A lot of realistic problems involve percentages and so we use them often in this text.

%\newpage

%%\section{Variables and functions}

\begin{center}
\line(1,0){300} %\line(1,0){250}
\end{center}

\section*{Homework}

\noindent \textbf{Start by doing Practice exercises \#1-4 in the workbook.}

\bigskip

\noindent \textbf{Do you know \ldots}

\begin{itemize}
\item The difference between a variable and a constant?
\item The information needed to ``name'' a variable?
\item Which variable is dependent and which variable is independent? 
\item What ``domain'' means?
\item How to calculate percent increase? 
\item[$\star$] The symbol for ``approximately equal to''? 
\item[$\star$] Why an approximate answer is often as good as we can get? 
\item[$\star$] When to round your answer up or down instead of off? 
\item[$\star$] What the term ``precisely'' refers to? 
\item[$\star$] How to decide how precisely to round your answer?

\hfill  $\star$ indicates question based on \emph{Prelude:  approximation}
\item[~] \textbf{If you're not sure, work the rest of exercises and then return to these questions.  Or, ask your instructor or a classmate for help.}
\end{itemize}

\subsection*{Exercises}

\begin{enumerate} 
\setcounter{enumi}{4}

\item  It's about time!  For each story, try to figure out the answer to the question(s).
\begin{enumerate}
\item The Nussbaums planted a walnut tree years ago when they first bought their house.  The tree was 5 feet tall then and has grown around 2 feet a year. The tree is now 40 feet tall.  How long ago did the Nussbaums plant their walnut tree?
\item After his first beer, Stephen's blood alcohol content (BAC) was already 0.04 and as he continued to drink, his BAC level rose 45\% per hour.  Note that $$45\% = \frac{45}{100} = 45 \div 100 = 0.45$$  What was Stephen's BAC after 1 hour?  After 2 hours?

\hfill \emph{Story also appears in 2.4 Exercises and 3.4 \#1}
\item When McKenna drives 60 mph (miles per hour) it takes her 20 minutes on the highway to get between exits, but when traffic is bad it can take her an hour.  How slow is McKenna driving when traffic is bad?  \emph{Hint:  can you figure out the distance between exits?}
\item The sun set at 6:00 p.m.\ today and I heard on the radio that it sets about 2 minutes earlier each day this time of year.  In how many days will the sun set at 4:30 p.m.?
\emph{Bonus question:  in what month is the story set?}

 \hfill \emph{Stories also appear in 1.1 \#4}
\end{enumerate}  

\item The temperature was 40$^\circ$F at noon yesterday downtown Minneapolis but it dropped 3$^\circ$F an hour in the afternoon.   \hfill \emph{Story also appears in 1.2 and 4.1 Exercises}
 \begin{enumerate}
 \item Which number is a constant in this story: the temperature (40) or the rate at which the temperature dropped (3)?
\item Name the variables, including units and dependence. 
\item When did the temperature drop below freezing (32$^\circ$F)?
\end{enumerate}  

\item Mrs.\ Nystrom's Social Security benefit was \$746.17/month when she retired from teaching in 2009. She had taught in elementary school since I was a girl.   Benefits have increased by 4\% per year.   \hfill \emph{Story also appears in 1.2 and 5.1 Exercises} 
\begin{enumerate}
\item Name the variables, including units and dependence. 
\item What was her benefit in 2012?
\item When will her benefit pass \$900/month?  A reasonable guess is fine.  
\end{enumerate}  

\item Between e-mail, automatic bill pay, and online banking, it seems like I hardly ever actually mail something.   But for those times, I need postage stamps. The corner store sells as many (or few) stamps as I want for 44\textcent~each but they charge a 75\textcent~convenience fee for the whole purchase.   \hfill \emph{Story also appears in 3.1 Exercises}
\begin{enumerate}
\item  Identify and name the variables, including the units.
\item Which variable is dependent and which is independent?
\item How many stamps could I buy for \$10?  Try to figure it out from the story.\end{enumerate} 

\item Sof\'ia bought her car new for \$22,500.  Now the car is fairly old and just passed 109,000 miles.  Sof\'ia looked online and estimates the car is still worth \$5,700.   

  \hfill \emph{Story also appears in Section 5.4} 
\begin{enumerate}
\item Identify and name the variables, including the units
\item Explain the dependence using a sentence of the form ``\underline{~\quad} is a function of \underline{~\quad}''
\item What is a realistic number of miles for a car to drive?  Express the domain as an inequality.
\item Sof\'ia  wonders when the car would be practically worthless, meaning under \$500.   Make a reasonable guess.
\end{enumerate} 

\item For each story, name the variables including units and dependence.
\begin{enumerate}
\item The closer you sit to a lamp, the brighter the light is. 

\hfill \emph{Story also appears in 2.3 and 3.3 Exercises.}
\item The thicker the piece of fish, the longer it takes to grill it.

\hfill \emph{Story also appears in 2.3 and 3.5 Exercises.}
\item Wind turbines are used to generate electricity.  The faster the wind, the more power they generate. \hfill \emph{Story also appears in 1.3, 2.4, and 3.3 Exercises.}

\end{enumerate}

\end{enumerate}



  %% ADD EXERCISES BACK IN LATER