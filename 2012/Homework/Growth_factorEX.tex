%\section{Growth factors}

 \begin{center}
\line(1,0){300} %\line(1,0){250}
\end{center}

\section*{Homework}

\noindent \textbf{Start by doing Practice exercises \#1-4 in the workbook.}

\bigskip

\noindent \textbf{Do you know \ldots}

\begin{itemize}
\item How to find the growth/decay factor given the starting amount and another point of information? 
\item How to find the growth/decay factor given the doubling time or half-life? 
\item When we use the \textsc{Percent Change Formula}, and when we use the \textsc{Growth Factor Formula} instead?  \emph{Ask your instructor if you need to remember the \textsc{Percent Change Formula} and \textsc{Growth Factor Formula} or if they will be provided during the exam.}
\item How to evaluate the \textsc{Percent Change Formula} and \textsc{Growth Factor Formula} using your calcuator? 
\item How to read the starting amount and percent increase/decrease from the equation? 
 \item[~] \textbf{If you're not sure, work the rest of exercises and then return to these questions.  Or, ask your instructor or a classmate for help.} 
\end{itemize}

\subsection*{Exercises}

\begin{enumerate} 
\setcounter{enumi}{4}

\item Estimates for childhood obesity for 2010 were revised to 2.1 out of every ten children.  (The 1994 figure of 1.1 out of every ten children remains accurate.)
\begin{enumerate}
\item Calculate the revised growth factor.  What is the revised percent increase?
\item Revise your equation.
\item Use your new equation to project childhood obesity rates for 2020.
\item Graph both the original and revised estimates on the same set of axes.  
\end{enumerate}

\item For each equation, find the growth rate (percent increase or percent decrease) and state the units. (For example, something might ``grow 2\% per year'' while something else might ``drop 7\% per hour'')  
\begin{enumerate} 
\item The light $L\%$ that passes through panes of glass $W$ inches thick is given by the equation
$$L = 100\ast .75^W$$
\hfill \emph{Story also appears in 2.4 and 3.4 Exercises}

\item The population of bacteria ($B$) in a culture dish after $D$ days is given by the equation $$B=\text{2,000}\ast 3^D$$
\hfill \emph{Story also appears in 5.2 Exercises}

\item The remaining contaminants ($C$ grams) in a waste water sample after $M$ months of treatment is given by $$C=8 \ast .25^M$$
\hfill \emph{Story also appears in 5.2 Exercises}
\end{enumerate}

\item Years ago, Whitney bought an antique mahogany table worth \$560.  Now, 30 years later, she had the table appraised for \$\text{3,700}.  
\begin{enumerate}
\item Calculate the annual growth factor, assuming the value of Whitney's table has increased exponentially.
\item What should she expect the set to be worth in another 10 years? As part of your work, name the variables and write an equation relating them.
\end{enumerate}

\item The opiate drug morphine leaves the body quickly.  After 72 hours about 10\% remains.  A patient receives 100 mg of morphine.
\begin{enumerate}
\item How much morphine will remain in the patient's body after 72 hours?
\item Convert 72 hours to days.
\item Find the daily decay factor using the \textsc{Growth Factor Formula}.
\item What is the corresponding percent decrease?
\item Name the variables and write an equation relating them.  Check that 72 hours gives you the same answer as in part (a).
\item What is the half-life of morphine?  Set up and solve an appropriate equation.
\item Draw a graph showing this patient's morphine levels for 10 days following the injection.
\end{enumerate}

\item Unemployment figures were just released.  At last report there were \text{20,517} unemployed adults and now, 10 months later, we have \text{39,061} unemployed adults.  \begin{enumerate}
\item Calculate the monthly growth factor, assuming unemployment increases exponentially.
\item Write an equation relating the variables.
\item According to your equation, what is the expected number of unemployed adults 6 months from now.  \emph{Notice:  the report was issued 10 months ago.}
\item Make a table of values and draw a graph showing the number of unemployed adults for the past 10 months and the next 2 years.
\end{enumerate}

\item Wetlands help support fish populations, various plant and animal populations, control floods and erosion from nearby lakes and streams, filter water, and help preserve our supply of ground water. 
 Minnesota wetlands acreage in 1850 was 18.6 million acres.  By 2003, that number had dropped to 9.3 million acres. 
 
 \hfill \begin{footnotesize} Source:  Minnesota Department of Natural Resources \end{footnotesize}

 \begin{enumerate}
\item Assuming the acreage decreased exponentially, name the variables, find the annual decay factor and  write an exponential equation showing how Minnesota wetlands have decreased.
\item With some effective management, many wetlands have been restored.  By 2012, it's up to about 10.6 million acres.  Assuming acreage has increased exponentially from 2003, name the variables (you may now want to start the years in 2003), find the growth factor and write an exponential equation showing how Minnesota wetlands have been restored.   
\end{enumerate}

\end{enumerate}

