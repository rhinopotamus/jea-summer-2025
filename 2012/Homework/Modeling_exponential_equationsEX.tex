%\section{Modeling with exponential equations}

 \begin{center}
\line(1,0){300} %\line(1,0){250}
\end{center}

\section*{Homework}

\noindent \textbf{Start by doing Practice exercises \#1-4 in the workbook.}

\bigskip
 
\noindent \textbf{Do you know \ldots}

\begin{itemize} 
\item What makes a function exponential? 
\item The template for an exponential equation? \emph{Ask your instructor if you need to remember the template or if it will be provided during the exam.} 
\item How to write an exponential equation given the starting amount and percent increase?     
\item Where the growth factor and starting amount appear in the template of an exponential equation?  
\item What the graph of an exponential function looks like?  
\item How to solve an exponential equation using the \textsc{Log Divides Formula}?   

\emph{Ask your instructor if you need to remember the \textsc{Log Divides Formula} or if it will be provided during the exam.}  
\item How to calculate the rate of change of an exponential function?     
\item Why the rate of change of an exponential function is not constant?     
\end{itemize}

\subsection*{Exercises}

\begin{enumerate} 
\setcounter{enumi}{4}

\item Use the equation from this section for the value of the sterling silverware to determine when the sterling was first worth over \$\text{5,000}.
\begin{enumerate}
\item First, estimate the answer from our table and graph.
\item Next, use successive approximation to refine your answer. Display your work in a table.
\item Last, practice setting up and solving an equation using the \textsc{Log Divides Formula}.
\end{enumerate}

\item Mrs.\ Nystrom's Social Security benefit was \$746.17/month when she retired from teaching in 2009. She had taught in elementary school since I was a girl.   Benefits have increased by 4\% per year.   \hfill \emph{Story also appears in 1.1 and 1.2  Exercises} 
\begin{enumerate}
\item Name the variables and write an equation relating them.
\item Use your equation to estimate her benefit in the year 2020.
\item Set up and solve an equation to determine when her benefit will pass \$900/month.
\item Repeat for \$\text{1,000}/month.
\end{enumerate}  

\item The number of players of a wildly popular mobile app drawing game  has been growing exponentially according to the equation $$N = 2 \ast 1.57^W$$ where $N$ is the number of players (in millions) and $W$ is the number of weeks since it caught on.
\hfill \emph{Story also appears in 5.1 \#3 and 5.3 Exercises}
 % Source: Wikipedia (Draw Something) 5 weeks 20 million, 50 days 50 million
\begin{enumerate}
\item Make a table showing the number of players after 0 weeks, 2 weeks, 4 weeks, and 6 weeks.  
\item Use successive approximation to determine when there will be over 60 million players.  Round your answer to the nearest week.
\item Show how to solve the equation to determine when there will be over 60 million players.  Record your answer to two decimal places.  
\item Use your answer to (a), (b), and (c) to graph the function.
\end{enumerate} 

\item In 2006 there were about 5.2 million people living in the state of Minnesota.  Predicted growth rates vary, perhaps around .5\% per year. % SU in 2010 it was 5.3  % Source MN dept of administration
\begin{enumerate}
\item What is the annual growth factor?  Careful, the growth rate is .5\%
\item Based on these figures, about how many people will be living in the state of Minnesota in 2010?  In 2020?  
\item Write an equation showing how Minnesota's population is a function of the year.  Don't forget to name the variables.
\item Make a table of values showing the projected population every two years from 2006 to 2020.  
\item Draw a graph illustrating the dependence.
\item Set up and solve an equation to determine when Minnesota's population is expected to be double the population from 2006.
\end{enumerate}

\item Um Archivo data consultant group reported earnings of \$42.7 billion in 2012.  At that time executives projected 17\% increase in earnings annually.  Based on that information, we wrote the equation $$U = 42.7 \ast 1.17^Y$$ where $U$ is Um Archivo's reported earnings (in \$billions) and $Y$ is the years since 2012.
\hfill \emph{Story also appears in 2.2 Exercises}
\begin{enumerate}
\item According to your equation, in what year would Um Archivo's reported earnings pass \$60 billion?  Set up and solve an equation.  Then check your answer.
\item Repeat for \$100 billion.
\end{enumerate}  

\item In 1990 it was estimated that 2.5 million households watched reality television at least once a week.  Executives predicted that number would increase by 7.2\% each year.  According to their estimates, how many millions of households watched reality television in 2000?  In 2010?  As part of your work, name the variables, find the annual growth factor, and write an exponential equation modeling reality television viewing.

\hfill \emph{Story also appears in 5.2 \#3}

\end{enumerate}




