%\section{Variables and functions}

\begin{center}
\line(1,0){300} %\line(1,0){250}
\end{center}

\section*{Homework}

\noindent \textbf{Start by doing Practice exercises \#1-4 in the workbook.}

\bigskip

\noindent \textbf{Do you know \ldots}

\begin{itemize}
\item The difference between a variable and a constant?
\item The information needed to ``name'' a variable?
\item Which variable is dependent and which variable is independent? 
\item What ``domain'' means?
\item How to calculate percent increase? 
\item[$\star$] The symbol for ``approximately equal to''? 
\item[$\star$] Why an approximate answer is often as good as we can get? 
\item[$\star$] When to round your answer up or down instead of off? 
\item[$\star$] What the term ``precisely'' refers to? 
\item[$\star$] How to decide how precisely to round your answer?

\hfill  $\star$ indicates question based on \emph{Prelude:  approximation}
\item[~] \textbf{If you're not sure, work the rest of exercises and then return to these questions.  Or, ask your instructor or a classmate for help.}
\end{itemize}

\subsection*{Exercises}

\begin{enumerate} 
\setcounter{enumi}{4}

\item  It's about time!  For each story, try to figure out the answer to the question(s).
\begin{enumerate}
\item The Nussbaums planted a walnut tree years ago when they first bought their house.  The tree was 5 feet tall then and has grown around 2 feet a year. The tree is now 40 feet tall.  How long ago did the Nussbaums plant their walnut tree?
\item After his first beer, Stephen's blood alcohol content (BAC) was already .04 and as he continued to drink, his BAC level rose 45\% per hour.  Note that $$45\% = \frac{45}{100} = 45 \div 100 = .45$$  What was Stephen's BAC after 1 hour?  After 2 hours?

\hfill \emph{Story also appears in 2.4 Exercises and 3.4 \#1}
\item When McKenna drives 60 mph (miles per hour) it takes her 20 minutes on the highway to get between exits, but when traffic is bad it can take her an hour.  How slow is McKenna driving when traffic is bad?  \emph{Hint:  can you figure out the distance between exits?}
\item The sun set at 6:00 p.m.\ today and I heard on the radio that it sets about 2 minutes earlier each day this time of year.  In how many days will the sun set at 4:30 p.m.?
\emph{Bonus question:  in what month is the story set?}

 \hfill \emph{Stories also appear in 1.1 \#4}
\end{enumerate}  

\item The temperature was 40$^\circ$F at noon yesterday downtown Minneapolis but it dropped 3$^\circ$F an hour in the afternoon.   \hfill \emph{Story also appears in 1.2 and 4.1 Exercises}
 \begin{enumerate}
 \item Which number is a constant in this story: the temperature (40) or the rate at which the temperature dropped (3)?
\item Name the variables, including units and dependence. 
\item When did the temperature drop below freezing (32$^\circ$F)?
\end{enumerate}  

\item Mrs.\ Nystrom's Social Security benefit was \$746.17/month when she retired from teaching in 2009. She had taught in elementary school since I was a girl.   Benefits have increased by 4\% per year.   \hfill \emph{Story also appears in 1.2 and 5.1 Exercises} 
\begin{enumerate}
\item Name the variables, including units and dependence. 
\item What was her benefit in 2012?
\item When will her benefit pass \$900/month?  A reasonable guess is fine.  
\end{enumerate}  

\item Between e-mail, automatic bill pay, and online banking, it seems like I hardly ever actually mail something.   But for those times, I need postage stamps. The corner store sells as many (or few) stamps as I want for 44\textcent~each but they charge a 75\textcent~convenience fee for the whole purchase.   \hfill \emph{Story also appears in 3.1 Exercises}
\begin{enumerate}
\item  Identify and name the variables, including the units.
\item Which variable is dependent and which is independent?
\item How many stamps could I buy for \$10?  Try to figure it out from the story.\end{enumerate} 

\item Sof\'ia bought her car new for \$\text{22,500}.  Now the car is fairly old and just passed \text{109,000} miles.  Sof\'ia looked online and estimates the car is still worth \$\text{5,700}.   

  \hfill \emph{Story also appears in Section 5.4} 
\begin{enumerate}
\item Identify and name the variables, including the units
\item Explain the dependence using a sentence of the form ``\underline{~\quad} is a function of \underline{~\quad}''
\item What is a realistic number of miles for a car to drive?  Express the domain as an inequality.
\item Sof\'ia  wonders when the car would be practically worthless, meaning under \$500.   Make a reasonable guess.
\end{enumerate} 

\item For each story, name the variables including units and dependence.
\begin{enumerate}
\item The closer you sit to a lamp, the brighter the light is. 

\hfill \emph{Story also appears in 2.3 and 3.3 Exercises.}
\item The thicker the piece of fish, the longer it takes to grill it.

\hfill \emph{Story also appears in 2.3 and 3.5 Exercises.}
\item Wind turbines are used to generate electricity.  The faster the wind, the more power they generate. \hfill \emph{Story also appears in 1.3, 2.4, and 3.3 Exercises.}

\end{enumerate}

\end{enumerate}



