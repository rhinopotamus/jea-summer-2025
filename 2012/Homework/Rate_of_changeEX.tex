%\section{Rate of change}

 \begin{center}
\line(1,0){300} %\line(1,0){250}
\end{center}

\section*{Homework}

\noindent \textbf{Start by doing Practice exercises \#1-4 in the workbook.}

\bigskip

\noindent \textbf{Do you know \ldots}

\begin{itemize}
\item How to calculate rate of change between two points?   \emph{Ask your instructor if you need to remember the formula or if it will be provided during the exam.} 
\item What the rate of change means in the story?   
\item How we can use the rate of change to estimate values?   
\item When a function is increasing or decreasing, and the connection to the rate of change?   
\item Why the rate of change is zero at the maximum (or minimum) value of a function?   
\item What the connection is between rate of change and the steepness of the graph?   
\item[~] \textbf{If you're not sure, work the rest of exercises and then return to these questions.  Or, ask your instructor or a classmate for help.} 
\end{itemize}

\subsection*{Exercises}

\begin{enumerate} 
\setcounter{enumi}{4}
\item Look back at the springboard diver example in this section. 
\begin{enumerate}
\item  Check the other rates of change given in the table.
\item Approximately how fast is the diver moving as she enters the water?  Use that her height at 1.4 seconds is 1 foot above water (given earlier), but also her height at 1.5 seconds is just .12 feet above water.
\end{enumerate}

\item Your local truck rental agency lists what it costs to rent a truck (for one day) based on the number of miles you drive the truck.  
\begin{center}
\begin{tabular} {|l| |c |c|c|c|} \hline
Distance driven (miles) & 50 & 100 & 150 & 200 \\ \hline
Rental cost (\$) & 37.50 & 55.00 & 72.50 & 90.00 \\ \hline
\end{tabular}
\end{center}
 \hfill \emph{Story also appears in 1.2 and 4.4 Exercises}
\begin{enumerate}
\item Calculate the rate of change for each time period.
\item Can you use figure out what it probably costs to rent a truck to drive 75 miles?
\item There must be some sort of fixed price plus a per mile price.  Can you figure out what that fixed price must be?  
\end{enumerate}

\item Wind turbines are used to generate electricity.  A few values are recorded in the table
\begin{center}
\begin{tabular} {|c| |c |c|c |c|}\hline
wind speed (mph) & 0 & 10 & 20 & 30 \\ \hline
electricity (watts) & 0 & \text{2,400} & \text{19,200} & \text{64,800} \\ \hline
\end{tabular}
\end{center}
\hfill \emph{Story also appears in 1.1, 2.4, and 3.3 Exercises}

\begin{enumerate}
\item Name the variables, including units and dependence.
\item Plot the points from the table on a graph.
\item Calculate the rate of change in electricity as a function of wind speeds from 0 to 10 mph.  Sketch in the line segment connecting those two points on the graph.   
\item Repeat for wind speeds from 10 to 20 mph.  Is the electricity produced increasing faster or increasing slower than for lower wind speeds.
\item Repeat for wind speeds from 20 to 30 mph.  Comment again on how the rate of change compares to earlier rates of change.
\end{enumerate} 

\item The table lists estimates of Earth's population, in billions, for select years since 1800.   
\begin{center}
\begin{tabular} {|l |c |c |c |c |c |c |c |} \hline
year & 1800 & 1850 & 1900 & 1950 & 1970 & 1990 & 2000 \\ \hline
population & .98 & 1.26 & 1.65 & 2.52 & 3.70 & 5.27 & 6.06  \\ \hline
\end{tabular}
\end{center}
\hfill \begin{footnotesize} Source:  ``The World at Six Billion'' United Nations report, 1999\end{footnotesize}
%http://www.un.org/esa/population/publications/sixbillion/sixbilpart1.pdf

\hfill \emph{Story also appears in 1.2 Exercises}

During which period of time was the Earth's population increasing the fastest?  Calculate the rates of change for each time period to decide.  (Or, explain some other way of deciding.)

\item A company produces backpacks.  The more they make, the less it costs for each one.   For example, if they produce 10 backpack it would cost \$39 each.  For 40 backpacks, they would cost \$18 each.  By 70 backpacks, the unit cost is down to \$15 each.  At 100 backpacks, the unit cost is \$30 each. \hfill \emph{Story also appears in 3.5 Exercises}
\begin{enumerate}
\item Name the variables and summarize the information in a table.
\item Calculate the rates of change between 10 and 40 backpacks, between 40 and 70 backpacks, and between 70 and 100 backpacks.
\item For which range of values does the cost per backpack decrease?  
\item Any ideas why the cost might increase?
\item Draw a graph illustrating the dependence.  Try for a nice, smooth curve.
\item Approximately how many backpacks does the company have to make to keep the cost per backpack as small as possible?
\end{enumerate}

\item The public beach near Paloma's house has lost depth (measured from the dunes to the high water mark) due to erosion since they started keeping records 60 years ago.  The table shows a few values. There $D$ is the depth of the beach in feet, and $Y$ is the year, measured since 60 years ago.  
\begin{center}
\begin{tabular} {|c| |c|c |c |c|}\hline
year & 60 years ago & 30 years ago & 10 years ago & ~\quad now \quad ~\\ \hline
$Y$ & 0 & 30 & 50 & 60 \\ \hline
$D$ & 435 & 322.5 & 247.5 & 210 \\ \hline
\end{tabular}
\end{center} 
\hfill \emph{Story also appears in 4.3 Exercises}
  
\begin{enumerate}
\item Calculate the rates of change for each time period.  
\item Explain why the rates of change should be negative.
\item Approximately how many feet a year is the beach eroding?  
\item Draw a graph showing how the beach depth has changed over the past 60 years.
\end{enumerate}

\end{enumerate}

