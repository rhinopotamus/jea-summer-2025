%\section{A first look at exponential equations}

 \begin{center}
\line(1,0){300} %\line(1,0){250}
\end{center}

\section*{Homework}

\noindent \textbf{Start by doing Practice exercises \#1-4 in the workbook.}

\bigskip

\noindent \textbf{Do you know \ldots}

\begin{itemize}  
\item How to find the growth factor if you know the percent increase?    
\item How to calculate percent increase in one step?   
\item What makes a function exponential?   
\item The template for an exponential equation? \emph{Ask your instructor if you need to remember the template or if it will be provided during the exam.} 
\item Where the starting value and growth factor appear in the template for an exponential equation?   
\item What the graph of an exponential function looks like? 
 \item[~] \textbf{If you're not sure, work the rest of exercises and then return to these questions.  Or, ask your instructor or a classmate for help.} 
\end{itemize}

\subsection*{Exercises}

\begin{enumerate} 
\setcounter{enumi}{4}

\item Mai's salary was \$78,000 before she got a 6\% raise.  Now the economy was not doing as well and she got only a 1.5\% raise this year.
\begin{enumerate}
\item What was her salary after the second raise?
\item Her colleague Tom\'a\u s started with a salary of \$78,000 but did not get a raise the first year like Mai did.  What percentage raise would Tom\'a\u s need now in order to have the same final salary as Mai?
\item Would Mai's salary have been the more than, less than, or the same as now if she had received the 1.5\% raise first and then the 6\% raise?
\item Which order would you rather have:  6\% then 1.5\% or 1.5\% then 6\%?  Why?
\end{enumerate}

\item The number of school children in the district from a single parent household has been on the rise.  In one district there were \text{1,290} children from single parent households in 2010 and that number was expected to increase about 3\% per year.

\hfill \emph{Story also appears in 3.4 and 5.3 Exercises}
\begin{enumerate}
\item Calculate the annual growth factor.
\item How many children from single parent households are expected in that district by 2015?
\item Name the variables and write an equation relating them.
\item Make a table showing the number of school children in the district from a single parent household in 2010, 2015, 2020, and 2030.
\item Graph the function.  
\end{enumerate}

\item Um Archivo data consultant group reported earnings of \$42.7 billion in 2012.  At that time executives projected 17\% increase in earnings annually. 

 \hfill \emph{Story also appears in 5.1 Exercises}
\begin{enumerate}
\item Name the variables and find an equation relating them.
\item According to your equation, what would Um Archivo's earnings be in 2020.
\item If Um Archivo reports earnings of \$78.1 billion in 2020, would you say the projected rate of 17\% was too high or too low?  Explain.
\item Draw a graph showing how Um Archivo's profits are expected to increase.
\end{enumerate}  

\item In 2005, poultry production was 78 million tons estimated to be growing at a rate of around 1.6\% per year.   
\hfill \begin{footnotesize} Source:  Worldwatch Institute \end{footnotesize} 

 \hfill \emph{Story also appears in 3.4 Exercises}
\begin{enumerate}
\item Write an equation showing how poultry production is expected to rise. Don't forget to name the variables.
\item Make a table showing the production in 2005, 2010, 2020, and 2050 (at least according to the equation.)
\end{enumerate} 

\item Back in January 2008, e-book sales were averaging \$5.1 million per month and were increasing approximately 6.3\% per month.  (We are ignoring seasonal variation in this problem.)
%By 2010, that had jumped to around \$263.0 million.  
\hfill \begin{footnotesize} Source:  ReadWriteWeb \end{footnotesize} 

\begin{enumerate}
\item Name the variables including units.
\item Calculate the monthly growth factor.
\item Write an exponential equation illustrating this dependence.  
\item By January 2010, monthly sales averaged 21.9 million.  How does that compare to your equation's estimate?
\item What does your equation project for average monthly ebook sales in January 2014?
\end{enumerate}

\end{enumerate}