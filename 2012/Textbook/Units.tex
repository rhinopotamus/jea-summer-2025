~\vspace{.1in}

\section{Units}

We know 5 city blocks and 5 miles are very different lengths to walk; \$5 and 5\textcent \hspace{.01in} are very different values of money; 5 minutes and 5 years are very different amounts of time to wait -- even though all of these quantities are represented by the number 5. Every variable is measured in terms of some unit.  Since there are often several different units available to use it is important when naming a variable to state which units we are choosing to measure it in.  

In the last section we examined the height of a springboard diver and her speed in the air.  But, how high is 3 meters?  How fast is 4.4 meters per second?  

The metric unit of length called a meter is just over 3 feet (a yard).  Let's use 
$$1 \text{ meter} \approx 3.281 \text{ feet}$$  
We can use this conversion to change 3 meters to feet.
$$3 \text{ meters} \ast  \frac{3.281 \text{ feet}}{1 \text{ meter}} = 3 \times 3.281 = 9.843 \text{ feet} \approx 9.8 \text{ feet}$$  
Since our conversion is just approximate, we rounded off our answer too. 

See that fraction?  The 3.281 feet on the top and the 1 meter on the bottom are just two different ways of saying approximately the same distance. In other words, 
$$\frac{3.281 \text{ feet}}{1 \text{ meter}} \approx 1$$ 
 A fraction where the top and bottom are equal quantities expressed in different units is sometimes called a \textbf{unit conversion fraction}.  Because it's equal to 1 (or at least very close to 1), multiplying by the unit conversion fraction doesn't change the value, just the units.

You might wonder how we knew to put the feet on the top and the meters on the bottom.  One reminder for how this works is to think fractions.  It's like the meters on the top and bottom cancel, leaving the units as feet.  
$$\frac{3 \text{ \cancel{meters}}}{1} \ast  \frac{3.281 \text{ feet}}{1 \text{ \cancel{meter}}} = 3 \times 3.281 \approx 9.8 \text{ feet}$$

One more idea to keep in mind when converting units:  a few large things equals a lot of small things.  
Instead of buying a lot of small bags of ice to fill your cooler, you can buy a few larger  bags of ice instead. 
In our example, a meter is much bigger than a foot.  So it makes sense that a small number of meters (3 meters) equalled a larger number of feet (9.8 feet).  That might seem backwards, but that's how it works.  

Of course, 9.8 feet might sound like a funny answer.  We're much more used to a whole number of feet and then the fraction in inches.  It's 9 feet and some number of inches.  To figure out the inches we look at the decimal part $9.8-9=.8$.  That's the part we need to convert to inches.    Since there are 12 inches in a foot, we use the (exact) conversion
$$1 \text{ foot} = 12 \text{ inches}$$ to get
$$.8 \text{ \cancel{feet}} \ast \frac{12 \text{ inches}}{1 \text{ \cancel{foot}}} = .8 \times 12 = 9.6 \text{ inches} \approx 10 \text{ inches}$$  

Quick caution here.  We rounded off 9.843 to get 9.8 and then just used the .8 to find the extra inches.  Maybe we should have used the .843 instead.  Here's what happens.
$$.843 \text{ \cancel{feet}} \ast \frac{12 \text{ inches}}{1 \text{ \cancel{foot}}} = .843 \times 12 = 10.116 \text{ inches} \approx 10 \text{ inches}$$  Phew!  Either way, the board is about 9 feet and 10 inches high.  The common shorthand for this answer is 9'10".  (That's pronounced \emph{9 foot 10}, as in our team's new center is \emph{6 foot 7}.)  The ' symbol indicates feet and " indicates inches.

The highest height we had recorded for the diver was 4.48 meters.  Now we know that's 
$$ 4.48 \text{ \cancel{meters}}  \ast \frac{3.281 \text{ feet}}{1 \text{ \cancel{meter}}} 
=4.48 \times 3.281= 14.69888\ldots \text{ feet} \approx 14.7 \text{ feet}$$  
In feet and inches, that's about 14 feet, 8 inches because 
$$.69888 \text{ \cancel{feet}} \ast \frac{12 \text{ inches}}{1 \text{ \cancel{foot}}} = .69888 \times 12 = 8.38656 \text{ inches} \approx 8 \text{ inches}$$ 
The diver's highest height was around 14'8".  

You might have guessed that 14.7 feet would be 14'7".  I mean, that sort of looks obvious.  The reason it's not is because decimal numbers are based on 10.  The .7 really means $\frac{7}{10}$.  But inches are based on 12.  Seven inches means 
$$7''=\frac{7}{12}=7 \div 12 = .58333\ldots \approx .6$$
We wanted .7 so that's not it.  Our answer of 8" worked just fine since 
$$8" = \frac{8}{12} = 8 \div 12 = .66666\ldots \approx .7$$

What about the diver's speed?   During the first .2 seconds we calculated her speed as 4.4 meters per second.  How fast is that?  We can certainly convert to feet per second.  
$$\frac{4.4 \text{ \cancel{meters}}}{\text{second}} \ast \frac{3.281 \text{ feet}}{1 \text{  \cancel{meter}}} = 4.4 \times 3.281 = \frac{14.4364 \text{ feet}}{\text{second}}$$ 
Does that help us understand how fast she's going?  Maybe a little.  But, we're probably most familiar with speeds measured in miles per hour, that's what \textbf{mph} stands for.

Let's convert to miles per hour.  First, use that
$$1 \text{ minute} = 60 \text{ seconds}$$ to get
$$ \frac{14.4364  \text{ feet}}{\text{ \cancel{second}}} \ast \frac{60 \text{  \cancel{seconds}}}{1\text{ minute}} =  14.4364 \times 60 = \frac{866.184 \text{ feet}}{\text{minute}}$$ 
The larger number makes sense here because she can go more feet in a minute than in just one second.  

Next, use that
$$1 \text{ hour} = 60 \text{ minutes}$$ to get
$$ \frac{866.184 \text{ feet}}{\text{ \cancel{minute}}}\ast  \frac{60 \text{ \cancel{minutes}}}{1\text{ hour}} = 866.184 \times 60 = \frac{ \text{51,971.04 feet}}{\text{hour}}$$  
Again, the larger number makes sense because she can go more feet in an hour than in just one minute.  

Last, we need to convert to miles.  Turns out that 
$$1 \text{ mile} = \text{5,280 feet}$$ and so
$$ \frac{ \text{51,971.04 \cancel{feet}}}{\text{hour}}\ast  \frac{1 \text{ mile}}{\text{5,280 \cancel{feet}}} = 51,971.04 \div \text{5,280} = \frac{9.843 \text{ miles}}{\text{hour}} \approx 10 \text{ mph}.$$  
This time we got a smaller number because she can go a lot fewer miles in an hour compared to feet in an hour.  Notice how we needed to divide by \text{5,280}.  Numbers on top of the fraction multiply. Those on the bottom divide.

We can do this entire calculation all at once.  Notice how all of the units cancel to leave us with miles per hour.  $$ \frac{4.4 \text{ \cancel{meters}}}{\text{\cancel{second}}} \ast \frac{3.281 \text{ \cancel{feet}}}{1 \text{ \cancel{meter}}}  \ast \frac{60 \text{ \cancel{seconds}}}{1 \text{ \cancel{minute}}}\ast  \frac{60 \text{ \cancel{minutes}}}{1\text{ hour}}\ast  \frac{1 \text{ mile}}{5,280\text{ \cancel{feet}}} $$
$$= 4.4 \times 3.281 \times 60 \times 60 \div \text{5,280} = 9.843 \text{ mph} \approx 10 \text{ mph}.$$  

Right before the diver hit the water she was going around 7.25 meters per second.  How fast is that in mph? Ready for it all in one line?  Here it is.

$$ \frac{7.25 \text{ \cancel{meters}}}{\text{\cancel{second}}} \ast \frac{3.281 \text{ \cancel{feet}}}{1 \text{ \cancel{meter}}}  \ast \frac{60 \text{ \cancel{seconds}}}{1 \text{ \cancel{minute}}}\ast  \frac{60 \text{ \cancel{minutes}}}{1\text{ hour}}\ast  \frac{1 \text{ mile}}{5,280\text{ \cancel{feet}}} $$
$$=7.25 \times 3.281 \times 60 \times 60 \div 5,280 = 16.2185\ldots \approx 16 \text{ mph}$$

If you're having trouble setting up unit conversions, remember to write down the units so you can see how they cancel.  If you can't remember a number for a unit conversion, like \text{5,280} feet for one mile, try searching online.

 %\section{Units}

 \begin{center}
\line(1,0){300} %\line(1,0){250}
\end{center}

\section*{Homework}

\noindent \textbf{Start by doing Practice exercises \#1-4 in the workbook.}

\bigskip

\noindent \textbf{Do you know \ldots}

 \begin{itemize}
\item How to convert from one unit of measurement to another?   
\item What a unit conversion fraction is?   
\item Why multiplying by a unit conversion fraction doesn't change the amount, just the units?   
\item How to connect repeated conversions into one calculation?   
\item Why if we convert an amount to a larger unit, we use a smaller number?  
\item How many seconds in a minute, minutes in an hour, hours in a day, days in a year, inches in a foot, feet in a mile, and other common conversions? 

\emph{Ask your instructor which common conversions you need to remember, and whether any conversion formulas will be provided during the exam.}   
\item How to convert between English and metric measurements? 

\emph{Again, ask your instructor which metric conversions you need to remember, and whether any conversion formulas will be provided during the exam.} 
 \item[~] \textbf{If you're not sure, work the rest of exercises and then return to these questions.  Or, ask your instructor or a classmate for help.} 
\end{itemize}

\subsection*{Exercises}

\begin{enumerate} 
\setcounter{enumi}{4}

\item In August 2008, United States swimmer Michael Phelps set the world record for the 200 meter individual medley swimming it in 1 minute, 54.80 seconds. 

 \hfill \begin{footnotesize} Source:  Wikipedia (World record progression 200 metres IM)  \end{footnotesize}
\begin{enumerate}
\item Convert Phelps' time into minutes.  
\item How fast did Phelps' swim, as measured in meters/min? 
\item Convert Phelps' speed to mph.   Use $1 \text{ mile} \approx {1,609} \text{ meters}$.
\end{enumerate}

In August 2012, Phelps improved his time and won Olympic gold, but failed to break the world record his teammate Ryan Lochte has set a year earlier of 1 minute, 54 seconds.
\begin{enumerate}
\item [(d)] Convert Lochte' time into minutes.  
\item [(e)]  How fast did Lochte' swim, as measured in meters/min?  
\item [(f)] Convert Lochte' speed to mph.
\end{enumerate}

\item \begin{enumerate}
\item The typical weight limit for a suitcase on flights within Africa is 20 kg.  How many pounds is that? Use $1 \text{ kilogram} \approx 2.2 \text{ pounds}$.
\item How many servings are in a 20 ounce package of cookies where a serving size is 3 cookies and each cookie weighs 11 grams?  Use $1 \text{ ounce} = 28.3 \text{ grams}$.
\item My corner convenience store sells a ``thirst quencher'' size of soft drink; it holds 64 (fluid) ounces.  If a can of soft drink is 12 (fluid) ounces, how many cans are in the ``thirst quencher''?
\end{enumerate}

\item \begin{enumerate}
\item The football coach wants everyone to sprint three-quarters of a mile, up and back on the field which is labeled in yards.  How many yards are in three-quarters of a mile?
\item The quilt pattern calls for .375 yards of calico fabric. How many feet is .375 yards?
\item The website said that basil plants should be .35 feet tall a month after germinating.  How many inches is .35 feet?
\end{enumerate}

\item Authorities are tracking down the source of a pollution spill on a nearby river.  They suspect that the local plant is inadvertently leaking waste water.  Last week they found 35 minutes of waste water flow on Monday, 1 hour and 11 minutes on Tuesday, 1/4 hour on Wednesday (that's .25 hours in decimal), none on Thursday, and then 98 minutes Friday.
\begin{enumerate}
\item Convert units as needed to complete the following table showing each time in minutes, each time in hours, and each time in hours and minutes (H:MM format). 

\emph{Hint:  15 minutes in H:MM format would be 0:15}
\begin{center}
\begin{tabular} {|l |c|c|c |c|c|} \hline
Day & Mon & Tue & Wed & Thu & Fri \\ \hline
Minutes & 35 & && 0 & 98 \\ \hline
Hours & \hspace{.5in}~ &\hspace{.5in}~ &.25&\hspace{.5in}~ &\hspace{.5in}~  \\ \hline
H:MM & & 1:11 & \hspace{.5in}~   & &  \\ \hline
\end{tabular}
\end{center}
\item Calculate the total waste water flow measured last week. 
\end{enumerate}

\item If your heart beats around 70 times a minute, how many times does it beat in a week?  A year?

\item \begin{enumerate}
\item Harold's Physics textbook says an object is thrown into the air at 36 feet per second.  To understand how fast that is, convert to mph.
\item Harold's History textbook mentions that in 1800 the city encompassed about \text{6,000} acres.  How many square miles is that?  Use $1 \text{ square mile} = 640 \text{ acres}$.
\item Harold's Economics textbooks lists the recent high price of crude oil at \$100 per barrel.  He'd like to know what that means in \$/gallon of gasoline.  It turns out that 1 barrel of crude oil produces about 19.4 gallons of gasoline. 
\end{enumerate}



%\item TEN Baseball player Joe Mauer signed a multi-year contract with the Minnesota Twins for an average of \$23 million per year.  (And that doesn't include the income he gets from endorsements.)
%\begin{enumerate}
%\item What does Mauer's salary come to in dollars per hour?  That means for every hour, waking or sleeping, all year long. 
%\item If Mauer were working a standard 40 hour work week for 50 weeks a year, what would his salary be, again in dollars per hour? \emph{Hint: that's a total of 2,000 hours} 
%\item In a standard 162 game season, averaging about 2 hours and 51 minutes per game, assuming Joe plays every minute of every game, what does his salary come to in dollars per game minute? \emph{Hint:  calculate the total number of minutes} 
%\end{enumerate}


%%\item WHERE? Every morning Jill goes for a 45-minute walk. 
%\begin{enumerate}
%\item Identify and name the variable.  Don't forget the units.
%\item  Which variable is independent?
%\item  If Jill walked 2.5 miles, how fast was she walking?  Don't forget to convert units as needed.
%\end{enumerate}   




\end{enumerate}





