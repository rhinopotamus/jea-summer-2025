%!TEX root =  A_WS.tex

\section{Finance formulas  -- Practice exercises}

 \bigskip
 
\noindent \hrulefill

 \bigskip

\noindent \textsc{Compound Interest Formula:} \quad
$\displaystyle a = p \left( 1 + \frac{r}{12}\right) ^{12y}$ 

\bigskip

\noindent \textsc{Equivalent APR Formula:} \quad 
$\displaystyle \text{APR} = \left(1+\frac{r}{12}\right)^{12}-1$ 

\bigskip

\noindent  \textsc{Future Value Annuity Formula:} \quad
$\displaystyle a = p \ast \frac{\left( 1 + \frac{r}{12}\right) ^{12y}-1}{\frac{r}{12}}$ 

\bigskip

\noindent  \textsc{Loan Payment Formula:} \quad
$\displaystyle p = \frac{a  \ast \frac{r}{12}}{1-\left( 1 + \frac{r}{12}\right) ^{-12y}}$ 

\bigskip

\noindent  where 
\begin{center}
\begin{tabular} {l} 

$a$ = account balance or loan amount (\$) \\
$p$ = initial deposit (principal), regular deposit, or regular payment (\$) \\
$y$ = time invested (years)\\
$r$ = interest rate compounded monthly (as a decimal) \\ 
\end{tabular}
\end{center}

\noindent \hrulefill
\newpage %%%%%%

\begin{enumerate}
\item Use the indicated formulas to help Kiran figure out her finances.
%\item (a) \$3,305.13 \quad (b) 7.23\% \quad (c) \$415,475 \quad (d) \$400.65
\begin{enumerate}
\item Kiran deposited \$\text{2,500} in a money market account that earned 7\% interest compounded monthly.  Use the \textsc{Compound Interest Formula} to calculate her account balance after 4 years. \vfill
\item What is the equivalent APR on Kiran's money market account?  Use the \textsc{Equivalent APR Formula.} \vfill
\item Kiran puts \$400 a month in her retirement account that amazingly also earns 7\% interest compounded monthly.  Use the \textsc{Future Value Annuity Formula} to determine how much Kiran will have in her retirement account in 28 years. \vfill
\item Kiran would really like to buy a new hybrid car that sells for \$\text{23,500}.  Sadly Kiran's credit rating is not very good, so the best the dealership offers is a loan at (you guessed it) 7\% interest compouned monthly.  Use the \textsc{Loan Payment Formula} to calculate her monthly car payments on a six year loan. \vfill
\end{enumerate}

\newpage %%%%%%

\item Tim and Josh are saving for their kids' college in fifteen years. The account pays the equivalent of 5.4\% interest compounded monthly (taking into consideration various tax incentives). 
 \begin{enumerate}
\item Make a table comparing how much they will have after fifteen years if  they contribute  \$100 per month vs.\ \$500 per month vs.\ \$\text{1,000} per month. Use the \textsc{Future Value Annuity Formula}. \vfill  \vfill \vfill
\item Tim's parents decide to put \$\text{15,000} into the account now.  How much will that deposit be worth in fifteen years?  Use the \textsc{Compound Interest Formula}.  \vfill  \vfill
\end{enumerate}
%\item 
%\begin{tabular} {|c |c |c |c|}\hline
%$P$ & 100 & 500 & 1000 \\ \hline
%$A$ & 27,640.60 & 138,203.01 & 276,406.03 \\ \hline
%\end{tabular}

\item Use the \textsc{Equivalent APR Formula} to find the APR for each of the following published interest rates (compounded monthly) offered by recent credit card companies.
%\item (a) 9.38\% \quad (b) 12.58\% \quad (c) 22.17\%
\begin{enumerate}
\item 9\% \vfill
\item 12.8\%   \vfill
\item 20.19\% \vfill
\end{enumerate}

\newpage %%%%%%

\item  Cesar and Eliana are looking at three different houses to buy.  The first is a large new townhouse priced at \$\text{240,000}.  The second is a small but charming bungalow priced at \$\text{260,000}.  The third is a large 2-story house down the block priced at \$\text{280,000}. 
%\item (a) 
%\begin{tabular} {|c |c |c |c|}\hline
%$A$ & 240,000 & 260,000 & 280,000  \\ \hline
%$P$ & 1,077.71 & 1,167.52 & 1,257.33 \\ \hline
%\end{tabular}
%\quad (b) Increases monthly payment by \$89.81
 \begin{enumerate}
\item Calculate the monthly payment for each house for a 30-year mortgage at 3.5\% interest compounded monthly.   Use the \textsc{Loan Payment Formula}. 
\bigskip

Townhouse \vfill

Bungalow \vfill

2-Story \vfill

\item Describe the effect on Cesar and Eliana's monthly payment of each \$\text{20,000} increase in the house price  at this interest rate. \vfill
\end{enumerate}

\end{enumerate}

\newpage


\noindent \textbf{When you're done \ldots}

\begin{itemize}
\item [$\Box$] Check your solutions.  Still confused?  Work with a classmate, instructor, or tutor.
\item [$\Box$] Try the \textbf{Do you know} questions.  Not sure?  Read the textbook and try again.
\item [$\Box$] Make a list of key ideas and process to remember under \textbf{Don't forget!}
\item [$\Box$] Do the textbook exercises and check your answers. Not sure if you are close enough? Compare answers with a classmate or ask your instructor or tutor.  
\item [$\Box$] Getting the wrong answers or stuck?  Re-read the section and try again.   If you are still stuck, work with a classmate or go to your instructor's office hours or tutor hours.
\item [$\Box$] It is normal to find some parts of exercises difficult, but if most of them are a struggle, meet with your instructor or advisor about possible strategies or support services.
\end{itemize}





\bigskip

\noindent \textbf{Do you know \ldots} %SectionName

\begin{enumerate} [(a)]
\item How to determine which formula to use? \emph{Ask your instructor if you will be told which formula to use during the exam.}  
\item What the quantities $a$, $p$, $y$, and $r$ from the formulas mean in the story? 
\item How to evaluate the formulas on your calculator?  \emph{Ask your instructor which formulas you need to remember, and whether any formulas will be provided during the exam.}
\item Why parentheses are needed around the exponent, numerator, and denominator in most of the formulas? 
\item What APR means, and why it is different from the (nominal) interest rate? 
\end{enumerate}

\bigskip

\noindent \textbf{Don't forget!}


