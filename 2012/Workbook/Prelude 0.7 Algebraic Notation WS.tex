
\section{Prelude: Algebraic Notation} 

\subsection*{Practice exercises}

\begin{enumerate}

\item Since she has been pregnant, Zoe has gained the recommended $\frac{1}{2}$ pound per week. She weighted 153 pounds at the start of her pregnancy.  That means when she is $W$ weeks pregnant, that Zoe weighs $$153 + \frac{1}{2}W$$
What does this expression say Zoe will weigh when she's 40 weeks pregnant? 

\hfill \emph{Story also appears in 0.4 \#3  and 4.3 \#3} \vfill

\item Jody is using small wooden balls to make noses for her knitted gnomes.  She figured out that she can calculate the weight of each ball (in ounces) as $.2 \times B \land 3$.  Write this expression in algebraic notation. 

\hfill \emph{Story also appears in 0.6 \#1}
\vfill

\newpage %%%%%

\item Astra lives in a 1-bedroom apartment where they pay \$825 per month in rent. Thanks to new rent stabilization laws, Astra's rent can only increase 3\% per year.  That means after $Y$ years, their rent will be at most $$825(1.03^Y)$$  What does this expression say her rent could be in 5 years? 

\hfill \emph{Story also appears in 0.3 \#2} \vfill

\item  ``Rose gold'' is a mix of gold and copper.  If we mix 2 grams of gold with $C$ grams of copper, the percentage of the resulting alloy that is gold is given by the expression $$\frac{200}{2 +C}$$
What does this expression say the percentage of gold will be if we add 7 grams of copper?

 \hfill \emph{Story also appears in 0.4 \#2, 2.3 \#2, and 4.1 Exercises} \vfill

\end{enumerate}

\newpage %%%%%


\noindent \textbf{When you're done \ldots}

\begin{itemize}
\item [$\Box$] Check your solutions.  Still confused?  Work with a classmate, instructor, or tutor.
\item [$\Box$] Try the \textbf{Do you know} questions.  Not sure?  Read the textbook and try again.
\item [$\Box$] Make a list of key ideas and process to remember under \textbf{Don't forget!}
\item [$\Box$] Do the textbook exercises and check your answers. Not sure if you are close enough? Compare answers with a classmate or ask your instructor or tutor.  
\item [$\Box$] Getting the wrong answers or stuck?  Re-read the section and try again.   If you are still stuck, work with a classmate or go to your instructor's office hours or tutor hours.
\item [$\Box$] It is normal to find some parts of exercises difficult, but if most of them are a struggle, meet with your instructor or advisor about possible strategies or support services.
\end{itemize}





\bigskip

\noindent \textbf{Do you know \ldots}

\begin{enumerate} [(a)]
\item Where multiplication can be hidden in algebraic notation? \vfill
\item How powers are written in algebraic notation? \vfill
\item How division is written in algebraic notation? \vfill
\item What the word evaluate means? \vfill
\item How to evaluate an algebraic expression on your calculator? \vfill
\end{enumerate}

\noindent \textbf{Don't forget!}
\vfill \vfill \vfill




