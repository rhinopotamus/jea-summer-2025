%!TEX root =  A_WS.tex

\section*{Formulas}  
\markright{Formulas}
\addcontentsline{toc}{section}{Formulas} 

Formulas are used for many purposes, including finding constants  or solving specific types of equations.  Various disciplines, such as finance, have specific formulas to evaluate quantitiies.

\vfill
\noindent \hrulefill
\bigskip
 
 \subsection*{Formulas used to find constants:} \bigskip
 
\noindent \textsc{Rate of Change/Slope (of Linear) Formula:} \quad $$ \displaystyle \text{rate of change = }\frac{\text{1st dep - 2nd dep}}{\text{1st indep - 2nd indep}}$$

\bigskip

\noindent \textsc{Intercept (of Linear) Formula:}$$\text{intercept} = \text{dep} -  \text{slope} \ast \text{indep}$$

\bigskip
 
 \noindent \textsc{Percent Change Formula:} 
\begin{itemize}
\item   If a quantity changes by a percentage corresponding to growth rate $r$, then the growth factor is $$\displaystyle g=1+r$$
\item If the growth factor is $g$, then the growth rate is $$r = g-1$$ ~
\end{itemize}
 
\noindent \textsc{Growth Factor Formula:} \bigskip

If a quantity is growing (decaying) exponentially, then the growth (decay) factor is 
$$\displaystyle g = \sqrt[t]{\frac{a}{s}}$$ 
\quad where $s$ is the starting amount and $a$ is the amount after $t$ time periods.

 \bigskip
\noindent \hrulefill
\vfill

\newpage

~\vfill

\noindent \hrulefill

\subsection*{Formulas used to find solve specific types of equations:} \bigskip

\noindent \textsc{Root Formula:} \quad
The equation $C^n=v$ has solution $C= \sqrt[n]{v}$

 \bigskip
 
\noindent \textsc{Log-Divides Formula:} \quad
The equation $g^Y = v$ has solution $\displaystyle Y = \frac{\log (v)}{\log(g)}$

 \bigskip


\noindent \textsc{Quadratic Formula:} \quad The equation $aT^2+bT+c=0$ has solutions \\ $$T = \frac{-b}{2a} \pm \frac{\sqrt{b^2-4ac}}{2a}$$ 

\bigskip

\noindent \hrulefill

\vfill \vfill

\noindent \hrulefill
 
\subsection*{Formulas from finance:} \bigskip

\noindent \textsc{Compound Interest Formula:} \quad
$\displaystyle a = p \left( 1 + \frac{r}{12}\right) ^{12y}$ 

\bigskip

\noindent \textsc{Equivalent APR Formula:} \quad 
$\displaystyle \text{APR} = \left(1+\frac{r}{12}\right)^{12}-1$ 

\bigskip

\noindent  \textsc{Future Value Annuity Formula:} \quad
$\displaystyle a = p \ast \frac{\left( 1 + \frac{r}{12}\right) ^{12y}-1}{\frac{r}{12}}$ 

\bigskip

\noindent  \textsc{Loan Payment Formula:} \quad
$\displaystyle p = \frac{a  \ast \frac{r}{12}}{1-\left( 1 + \frac{r}{12}\right) ^{-12y}}$ 

\bigskip

\noindent  where 
\begin{center}
\begin{tabular} {l} 

$a$ = account balance or loan amount (\$) \\
$p$ = initial deposit (principal), regular deposit, or regular payment (\$) \\
$y$ = time invested (years)\\
$r$ = interest rate compounded monthly (as a decimal) \\ 
\end{tabular}
\end{center}

\noindent \hrulefill

\vfill



